%%%%%%%%%%%%%%%%%%%%%%%%%%%%%%%%%%%%%%%%%%%%%%%%%%%%%%%%%%%%%%%%%%%%%%%%%%%%%%%
%%%%%%%%%%%%%%%%%%%%%%%%%%%%%%%%%%%%%%%%%%%%%%%%%%%%%%%%%%%%%%%%%%%%%%%%%%%%%%%
%%%%%%%%%%%%%%%%%%%%%%%%%%%%%%%%%%%%%%%%%%%%%%%%%%%%%%%%%%%%%%%%%%%%%%%%%%%%%%%
%%%%%%%%%%%%%%%%%%%%%%%%%%% wriTeX ARTICLE TEMPLATE %%%%%%%%%%%%%%%%%%%%%%%%%%%
%%%%%%%%%%%%%%%%%%%%%%%%%%%%%%%%%%%%%%%%%%%%%%%%%%%%%%%%%%%%%%%%%%%%%%%%%%%%%%%
%%%%%%%%%%%%%%%%%%%%%%%%%%%%%%%%%%%%%%%%%%%%%%%%%%%%%%%%%%%%%%%%%%%%%%%%%%%%%%%
%%%%%%%%%%%%%%%%%%%%%%%%%%%%%%%%%%%%%%%%%%%%%%%%%%%%%%%%%%%%%%%%%%%%%%%%%%%%%%%

%%%%%% CHAPEAU/INTRO/STANDFIRST
%%% Factual
%%% Short
%%% Simple
%%% Summary-oriented
Start with \textcolor{wrtxColorSuccess}{\textbf{Chapeau/Intro/Standfirst}}. typically a few sentences that set the stage for the article, giving readers a quick snapshot of what to expect. It is often used to highlight the most relevant or intriguing aspects of the article, nudging readers to continue. It is a  a more factual or summary-oriented introduction. Somewhat equivalent to the \textcolor{wrtxColorSuccess}{\textbf{Abstract}} in academia.


%%%%%% Lead/Lede
%%% Who?
%%% What?
%%% When/Where?
%%% Why ?
%%% How ?
%%% Dynamic and Engaging
%%% May be Anecdotal or even Humorous
%%% Open with HOOK/ACCROCHE
The \textcolor{wrtxColorSuccess}{\textbf{Lead/Lede}} appears at the very start of the story. Its purpose is to \textcolor{wrtxColorSuccess}{\textbf{hook/accroche}} readers and provide the essential \textbf{"who, what, when, where, why, and how"} in an engaging manner. Leads are usually more narrative or conversational, aiming to pull readers into the story rather than simply summarizing it. Leads can be factual, anecdotal, or even humorous, depending on the article's style.


\wrtxArticleSection{Section}
\lipsum[1]

\wrtxArticleSubsection{Subsection}
\lipsum[2]

\wrtxArticleSubsubsection{Subsubsection}
\lipsum[3]
