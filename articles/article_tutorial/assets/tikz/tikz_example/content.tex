
\tikzset{
    % defining an Arrow
    arrow/.style={line width=0.3mm, black!100,
    decoration={markings,mark=at position 1 with {\arrow[scale=1.5,#1]{latex}}},
    postaction={decorate},
    shorten >=3pt, shorten <=3pt,
    align=center},
    arrow/.default=black!100
}
%%% Defining styles
\tikzstyle{exampleStyle}=[
    align=center,
    anchor=center,
    scale=0.3
];
%%% Defining coordinates
% Distances are in cm by default
\coordinate (coordinateA) at (0,0);
\coordinate (coordinateB) at (5,0);

%%% Creating nodes
% \node[<options>] (<id>) at (<coordinates>) {<contents>}
\node[
    exampleStyle
    ]
    (nodeA)
    at (coordinateA)
    {\includegraphics{example-image.jpg}};
%%% Placing a node relative to first one
\node[
    exampleStyle,
    yshift=9cm,
    font={\Huge\bfseries}
    ]
    (nodeB)
    at (nodeA)
    {TEXT CONTENT OF NODE};
%%% Another node
\node[
    exampleStyle,
    rotate=-180,
    opacity=0.65
    ]
    (nodeC)
    at (coordinateB)
    {\includegraphics{example-image.jpg}};
%%% Another node
\node[
    exampleStyle,
    rotate=90,
    scale=0.5,
    xshift=15cm,
    ]
    (nodeD)
    [above right=1.0 and 1.7 of nodeC] % different way to position
    {\includegraphics{example-image.jpg}};

\node[yshift=-15mm] at (coordinateA) {Some more text};

%%% Arrows
\draw [->, line width=1mm] (nodeA) -- (nodeB);
\draw [->, red] (0,1) -- (1,1);
\draw [->, myColorSuccess] (nodeB) --+ (1,1);
\draw[arrow] (nodeB) --+ (0,1);