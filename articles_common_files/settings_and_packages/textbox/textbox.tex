\ifthenelse{\boolean{isSplitInTwoColumns}}
{
  \newcommand{\textboxXmargin}{5pt}
  \newcommand{\textboxwidth}{0.98\linewidth}
}
{
  \newcommand{\textboxXmargin}{35pt}
  \newcommand{\textboxwidth}{0.95\linewidth}
}

% Define default box style
\tcbset{
  myTcbBaseStyle/.style={
    center, % enlarges bounding box on both sides to fill he line completely
    colback=myGrayLight!50, % Background color
    colframe=myGrayMed, % Outline color
    coltitle=myGrayLight, % Title background color
    sharp corners=south, % Square bottom corners
    rounded corners=north, % Rounded top corners
    boxrule=1pt, % Outline thickness
    top=15pt, % Inner top margin
    bottom=20pt, % Inner bottom margin
    left=\textboxXmargin, % Inner left margin
    right=\textboxXmargin, % Inner right margin
    arc=5pt, % Corner rounding radius
    fonttitle=\bfseries, % Title font
    width=\textboxwidth, % Box width
    before skip=20pt, % Vertical space before the box
    after skip=20pt, % Vertical space after the box
    title after break, % Keep title on broken pages
    breakable, % Allow breaking across pages
    enhanced, % Enable advanced TikZ features
    titlerule=0mm, % Optional title line
    % title={\strut}, % Dummy title; replace with actual text
  },
}

\newtcolorbox[]{myTextBoxBase}[2][]{%
myTcbBaseStyle,
% float, % allow to float ("nofloat" also possible)
% floatplacement=tbp, % default positioning: "{t}op", "{b}ottom", "{p}age" or "{h}ere"
drop fuzzy shadow southeast=myGrayDark,
title after break={#2 -- cont.}, % repeat title after each page break
title={#2},
#1% optional arguments
}

%%% List of Textboxes
\newcommand{\listtextboxname}{My list of Textboxes}
\newlistof{textbox}{loTB}{\listtextboxname}

%%% Textboxes command
\newcommand\myTBLinker{textbox:\the\value{totalTextboxesInArticle:\myarticleKeyCore}}% key to identify individual textboxes
\NewDocumentCommand{\myTextBox}
{
  O{}%optional arguments
  m% title
  m% content
}
{%
\ifthenelse{\boolean{isIncludeTextBoxes}}{%
  \ifthenelse{\boolean{isMoveTextBoxesToEndOfArticle}}
  {%%% Move textbox to end
    \moveTextboxToEnd{%
      \textboxFrame{#2}{#3}
      [nofloat]

      %%% List of Textboxes entry:
      \addLineToToTB{#2}
    }
  }
  {%%% ELSE add textbox as a float
      \textboxFrame{#2}{#3}
      [% /!\ NO SPACES HERE
        float, % allow to float ("nofloat" also possible)
        floatplacement=tbp, % default positioning: "{t}op", "{b}ottom", "{p}age" or "{h}ere"
        #1%
      ]%
    %%% List of Textboxes entry:
    \addLineToToTB{#2}
  }
}{}
}

\newcommand{\addLineToToTB}[1]
{%
  \isPortfolio{%
    %%%% PORTFOLIO
    \addcontentsline{loTB}{textbox}
    {%
      \protect\myContentsTextFont% change font
      \protect%
      \hypertarget{\myTBLinker}%
      {\indent\protect\numberline{%
        \thechapter.% Add chapter nb
        \the\value{totalTextboxesInArticle:\myarticleKeyCore}}%
        \ \ \ #1}%
    }
  }
  {%
    %%%% SINGLE ARTICLE
    \addcontentsline{loTB}{textbox}
    {%
      \protect\myContentsTextFont% change font
      \protect%
      \hypertarget{\myTBLinker}%
      {\indent\protect\numberline{%
        \the\value{totalTextboxesInArticle:\myarticleKeyCore}}%
        \ \ \ #1}%
    }
  }%
  \par%
}

\newcommand{\textboxTitle}[1]{%
  \textcolor{myGrayDark}{\TEXTtextbox\ \the\value{totalTextboxesInArticle:\myarticleKeyCore}:} \detokenize{#1}%
}

\NewDocumentCommand{\textboxFrame}
{
  m % textbox title
  m % textbox content
  O{} % optional arguments
}
{
  \stepcounter{totalTextboxesAltogether}%
  \refstepcounter{totalTextboxesInArticle:\myarticleKeyCore}%
  \begin{myTextBoxBase}[#3]
    {% TEXTBOX TITLE
        {\phantomsection%
        \ifthenelse{\boolean{isIncludeLoTB}}
        {% as hyperlink
          \hyperlink{\myTBLinker}{%
            \textboxTitle{#1}%
          }%
        }%
        {% as simple text
          \textboxTitle{#1}%
        }%
        }
    }%
    \label{\myTBLinker}% increment counter and ref it
    \isDraftDebugger{Ref key: \myTBLinker\\}{}%% Print reference key in draft mode
    % TEXTBOX CONTENTS
    % \onehalfspacing
    #2
  \end{myTextBoxBase}
}


\NewDocumentCommand{\myTextboxRef}
{
    m % reference key
    O{\TEXTtextbox} % reference prefix
}{%
  \begingroup%
  \ifthenelse{\boolean{isDraft}}{%
    \color{myColorWarning}%
    \texttt{(textbox:#1)}%print ref for debugging
  }
  {%
    \color{myGrayMed}%
  }
  %Print ref
  #2\ \ref{textbox:#1}%
  \endgroup%
}