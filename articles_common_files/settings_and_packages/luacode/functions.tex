\usepackage{luacode} % for 'luacode' environment and '\luastring' macro

% Lua code to record the start time
\begin{luacode*}
    startTime = os.clock()
    endTime = nil
    elapsedTime = nil
\end{luacode*}


%%% Create variables that will be stored in aux file
\providecommand{\wrtxCompileDuration}{0}%
\providecommand{\elapsedInt}{0}%
\providecommand{\elapsedFrac}{0}%

%%% Macro to write to auxiliary file
\makeatletter
\newcommand{\writeToAux}[2]{%
%%% \immediate\write\@auxout: Writes immediately to the .aux file.
%%% \string: Ensures that \ characters are written verbatim to the .aux file.
%%% \gdef: Makes the definition global, so it will work in subsequent runs.
% \immediate%
\protected@write\@auxout{}{\gdef\string#1{#2}}}
\makeatother

%%% Calculate elapsed time at the end of the document
\AtEndDocument{
    \luaexec{
        %-- beware of special characters! need to escape them%
        %-- beware of paragraphs !!!
        %-- Comment can cause issues, always add "--" and keep in separate lines!
        endTime = os.clock()
        %-- Calculate time difference
        elapsedTimeSeconds=endTime-startTime
        elapsedTimeMilliseconds=(elapsedTimeSeconds\%1)*10
        %-- any higher multiplication (100, 1000) and it fails for unknown reason
        %-- integer part
        elapsedInt=math.floor(elapsedTimeSeconds)
        %-- fractional part
        elapsedFrac=math.floor(elapsedTimeMilliseconds)
        %-- Format string
        elapsedIntFormatted=tostring(elapsedInt)
        elapsedFracFormatted=tostring(elapsedFrac)
        %-- Define the elapsed time in a TeX macro
        %-- tex.print("\\gdef\\wrtxCompileDurationTemp{" .. string.format("\%.2f seconds", elapsedTimeSeconds) .. "}")
        %-- Write to the AUX file directly
        f=io.open("auxiliary_files/\jobname.aux","a")
        f:write("\\gdef\\elapsedInt{"..elapsedIntFormatted.."}")
        f:write("\\gdef\\elapsedFrac{"..elapsedFracFormatted.."}")
    }
    %
    \isDraftDebugger{%
        \luaexec{
        % -- Print the values for debugging at document ned
        tex.print("Start Time: " .. string.format("\%.2f", startTime))
        tex.print(" --- ")
        tex.print("End Time: " .. string.format("\%.2f", endTime))
        tex.print(" --- ")
        tex.print("Elapsed Time: " .. string.format("\%.2f", elapsedTimeSeconds))
        }%
    }{}
    % STORE RESULT FOR SUBSEQUENT RUNS
    % \writeToAux{\wrtxCompileDuration}{wrtxCompileDurationTemp}
    % \elapsedInt
    % \elapsedFrac
}
