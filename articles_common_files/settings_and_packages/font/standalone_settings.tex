% lua function to transform a string into a command
\begin{luacode}
    function fontCommand(word)
        return "\\"..word
    end
\end{luacode}
% Define a wrapper command in LaTeX
\newcommand{\fontCmd}[1]{%
    \directlua{tex.print(fontCommand("#1"))}%
}


\begin{luacode}
    ---------------------------------------------------------------------------
    -- variables to hold included/excluded font families
    fontFamiliesExcluded={}
    fontFamiliesIncluded={}
    ---------------------------------------------------------------------------
    -- functions to print length of included/excluded lists
    function fontFamiliesExcludedLength()
        tex.print(#fontFamiliesExcluded)
    end
    function fontFamiliesIncludedLength()
        tex.print(#fontFamiliesIncluded)
    end
    ---------------------------------------------------------------------------
    -- functions to add into included/excluded lists
    function addToExcluded(s)
        if s ~= "" then -- if not empty
            fontFamiliesExcluded[#fontFamiliesExcluded + 1] = s
        end
    end
    function addToIncluded(s)
        if s ~= "" then -- if not empty
            fontFamiliesIncluded[#fontFamiliesIncluded + 1] = s
        end
    end

    ---------------------------------------------------------------------------
    -- function to loop over a list and add commas in between
    function loopOverList(l)
        list={}
        -- Print the list contents
        for i, item in pairs(l) do
            -- if the current item is empty
            if item == "" then
                break
            end
            -- if the current item is the last element
            if i ~= #l then
                list[#list + 1] = item .. ","
            else
                list[#list + 1] = item
            end
        end
        return list
    end

    ---------------------------------------------------------------------------
    -- functions to print included/excluded lists
    function printFamiliesIncluded()
        local list=loopOverList(fontFamiliesIncluded)
        if #list == 0 then
            tex.print("none")
        else
            tex.print(list)
        end
    end

    function printFamiliesExcluded()
        local list=loopOverList(fontFamiliesExcluded)
        if #list == 0 then
            tex.print("none")
        else
            tex.print(list)
        end
    end

\end{luacode}

% lua function wrappers
\NewDocumentCommand{\addToIncluded}
{m%
}{%
    \directlua{addToIncluded("#1")}%
}

\NewDocumentCommand{\addToExcluded}
{m%
}{%
    \directlua{addToExcluded("#1")}%
}

%%%% Filter font families based on whether they are available on the computer or not
\newcommand{\filterFontFamilies}{%
    \foreach \fam [count=\n] in \FontFamilies {%
        \expandafter\IfFontExistsTF\expandafter{\fam}
        {\addToIncluded{\fam}}
        {\addToExcluded{\fam}}
    }%
}

%%% Function to apply a passed string as a style for the following text
\NewDocumentCommand{\applyProperty}
{
    m% parameter to be applied
}
{%
    % check if if passed parameter is a font. Since the font family has already been filtered, if it is a font, it must exist. If it does not exists, it must be a font size or style
    \IfFontExistsTF{#1}%
    {% if it is a font, set it
        \setmainfont{#1}%
    }%
    {%% if it is not a font, apply it
        \fontCmd{#1}%
    }%
}%

% function to style labels
\newcommand{\wrtxLabel}[1]{\textcolor{black!75}{#1}}%%% styling labels

% function to add horizontal line separators between outer loop turns
\newcommand{\outerSeparator}{%
    \noindent\rule{\textwidth}{.4pt}\\[\dimexpr-\baselineskip+1mm+2pt]
    \rule{\textwidth}{2pt}
}
