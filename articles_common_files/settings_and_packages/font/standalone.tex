%%%%%%
% FILE TO TEST THE FONTS IN ISOLATION FOR COMPILATION SPEED
\documentclass[
    % 12pt, % set in settings with fontsize package
    % landscape % set with geometry package
    % twoside % set with geometry package
    % draft % "draft" compiles much faster than "final" /!\ do not use final in custom draftmode
]{scrartcl}


%%%%%%%%%%%%%%%%%%%%%%%%%%%%%%%%%%%%%%%%%%%%%%%%
% CHOSEN ARTICLE %%%%%%%%%%%%%%%%%%%%%%%%%%%%%%%
\newcommand{\chosenArticle}{article_template} % /!\ UPDATE HERE
%%%%%%%%%%%%%%%%%%%%%%%%%%%%%%%%%%%%%%%%%%%%%%%%

% define relative paths to common files/settings from article to be used by \input command; multiple such paths can be added akin to \graphicspath
\makeatletter
\providecommand*{\input@path}{}
\g@addto@macro\input@path{%
    % existing paths have the advantage of being clickable
    {./../../../articles/\chosenArticle/}
}% append to already existing default paths, so as not to remove them
\makeatother
%
%
%
% INITIAL INPUTS
\usepackage{luacode} % for 'luacode' environment and '\luastring' macro

% Lua code to record the start time
\begin{luacode*}
    startTime = os.clock()
    endTime = nil
    elapsedTime = nil
\end{luacode*}


%%% Create variables that will be stored in aux file
\providecommand{\wrtxCompileDuration}{0}%
\providecommand{\elapsedInt}{0}%
\providecommand{\elapsedFrac}{0}%

%%% Macro to write to auxiliary file
\makeatletter
\newcommand{\writeToAux}[2]{%
%%% \immediate\write\@auxout: Writes immediately to the .aux file.
%%% \string: Ensures that \ characters are written verbatim to the .aux file.
%%% \gdef: Makes the definition global, so it will work in subsequent runs.
% \immediate%
\protected@write\@auxout{}{\gdef\string#1{#2}}}
\makeatother

%%% Calculate elapsed time at the end of the document
\AtEndDocument{
    \luaexec{
        %-- beware of special characters! need to escape them%
        %-- beware of paragraphs !!!
        %-- Comment can cause issues, always add "--" and keep in separate lines!
        endTime = os.clock()
        %-- Calculate time difference
        elapsedTimeSeconds=endTime-startTime
        elapsedTimeMilliseconds=(elapsedTimeSeconds\%1)*10
        %-- any higher multiplication (100, 1000) and it fails for unknown reason
        %-- integer part
        elapsedInt=math.floor(elapsedTimeSeconds)
        %-- fractional part
        elapsedFrac=math.floor(elapsedTimeMilliseconds)
        %-- Format string
        elapsedIntFormatted=tostring(elapsedInt)
        elapsedFracFormatted=tostring(elapsedFrac)
        %-- Define the elapsed time in a TeX macro
        %-- tex.print("\\gdef\\wrtxCompileDurationTemp{" .. string.format("\%.2f seconds", elapsedTimeSeconds) .. "}")
        %-- Write to the AUX file directly
        f=io.open("auxiliary_files/\jobname.aux","a")
        f:write("\\gdef\\elapsedInt{"..elapsedIntFormatted.."}")
        f:write("\\gdef\\elapsedFrac{"..elapsedFracFormatted.."}")
    }
    %
    \isDraftDebugger{%
        \luaexec{
        % -- Print the values for debugging at document ned
        tex.print("Start Time: " .. string.format("\%.2f", startTime))
        tex.print(" --- ")
        tex.print("End Time: " .. string.format("\%.2f", endTime))
        tex.print(" --- ")
        tex.print("Elapsed Time: " .. string.format("\%.2f", elapsedTimeSeconds))
        }%
    }{}
    % STORE RESULT FOR SUBSEQUENT RUNS
    % \writeToAux{\wrtxCompileDuration}{wrtxCompileDurationTemp}
    % \elapsedInt
    % \elapsedFrac
}

% Overwritten in the article
% Do not use underscores



% Overwritten in the article
% Do not use underscores

\newcommand{\myarticleKeyCore}{} % biblatex key for my article (stripped of prefixes and suffixes, e.g. "article_template"). Leave empty here, else risk of side effects.
\newcommand{\myarticleKey}{%
myarticle:\myarticleKeyCore:\myLanguage%
}% biblatex key for my article

\newcommand{\myMainImg}{} % Main image, if any

\newcommand{\mainSourceKey}{bibliography:source_template} % biblatex key for main source citation

\usepackage{hologo}
\usepackage{ifluatex}
\usepackage{ifxetex}
\usepackage{ifvtex}

%%%%%%%%% VERSIONS
%  THESE VALUES ARE LIKELY TO BE OVERWITTEN DOWNSTREAM

\usepackage{etoolbox} %necessary for booleans
%%% Rules:
% Everything cammelcase!
\newcommand{\createAndSetBoolean}[2]{
  % #1: boolean name
  % #2: set to true or false
  \newbool{#1}\setbool{#1}{#2}
}

\newcommand{\myLanguage}{en}
% Language options: en, fr

% Whether PDF should be in draft or final style
\createAndSetBoolean{isDraft}{false} % true or false
% Whether to go into minimal, high speed mode
\createAndSetBoolean{isMinimal}{true} % true or false
% Whether figures should be included or skipped
\createAndSetBoolean{isIncludeFigures}{true} % true or false
% Whether
\createAndSetBoolean{isPrintVersion}{false} % true or false
\createAndSetBoolean{isDrawRibbons}{false}  % true or false
\createAndSetBoolean{isIncludeMeta}{true} % true or false
\createAndSetBoolean{isIncludeArticleCover}{true} % true or false
\createAndSetBoolean{isIncludeToC}{true}  % true or false
\createAndSetBoolean{isIncludeLoF}{true}  % true or false
\createAndSetBoolean{isIncludeLoT}{true}  % true or false
\createAndSetBoolean{isIncludeLoTB}{true}  % true or false
\createAndSetBoolean{isIncludeBiblio}{true}  % true or false
\createAndSetBoolean{isIncludeGlossary}{true}  % true or false
\createAndSetBoolean{isIncludeAbreviations}{true}  % true or false
\createAndSetBoolean{isPrintUnusedGlossary}{true}  % true or false
\createAndSetBoolean{isPrintUnusedAbreviations}{true}  % true or false
\createAndSetBoolean{isHighlightGlossaryAndAbreviations}{true}  % true or false
\createAndSetBoolean{isIncludeMissingBibEntries}{true}  % true or false
\createAndSetBoolean{hidecontentswitch}{true}  % true or false
\createAndSetBoolean{revealhiddenswitch}{false}  % true or false
% Whether footnotes should be included or ignored
\createAndSetBoolean{isIncludeFootnotes}{true}  % true or false
% Whether citations should be included or ignored
\createAndSetBoolean{isIncludeCitations}{true}  % true or false
% Whether citations should be printed in footnotes
\createAndSetBoolean{isIncludeCitationsInFootnotes}{true}  % true or false
% Whether textboxes should be printed or ignored
\createAndSetBoolean{isIncludeTextBoxes}{true}  % true or false
% Whether textboxes should be printed only at the end of article
\createAndSetBoolean{isMoveTextBoxesToEndOfArticle}{false}  % true or false
% Whether to constrain floats to respective sections
\createAndSetBoolean{isConstrainFloats}{true}  % true or false
% Whether lists items should be inline
\createAndSetBoolean{isCollapseLists}{true}  % true or false
% Whether to include cover image in the article body
\createAndSetBoolean{isIncludeArticleCoverImgInBody}{false}  % true or false
% Whether to include cover image in the article body
\createAndSetBoolean{isCreditsInArticleBody}{true}  % true or false
% Whether to include cover image in the article body
\createAndSetBoolean{isSplitInTwoColumns}{false}  % true or false
% Whether to make whole document in landscape mode or not
\createAndSetBoolean{isLandscapeMode}{false}  % true or false
% Whether to include appendix
\createAndSetBoolean{isIncludeAppendix}{false}  % true or false
% Whether to print custom logs
\createAndSetBoolean{isPrintLogs}{true}  % true or false


%%%% PORTOFOLIO SPECIFIC:
% Whether to add a higher level of organisation as parts to group articles
\createAndSetBoolean{isDivideArticlesIntoParts}{true}  % true or false
% Whether each article should have its own sub TOC/LOT/LOT
\createAndSetBoolean{isIncludePerArticleToC}{true}  % true or false
\createAndSetBoolean{isIncludePerArticleLoF}{true}  % true or false
\createAndSetBoolean{isIncludePerArticleLoT}{true}  % true or false
% Whether each article shows its substance in portfolio
\createAndSetBoolean{isIncludePerArticleSubstance}{true}  % true or false
% Whether the bibliography is split between chapters or grouped at end
\createAndSetBoolean{isSplitBibliographyByChapter}{false}  % true or false






\usepackage{xstring} % Needed for string manipulation

\newcommand{\addonlyfiles}[1]{%
  \def\onlyfiles{#1}%
}


% Create new command: \addContent
\NewDocumentCommand\addContent{
  +m % Arg 1 (Mandatory): Section
  +O{} % Arg 2 (Optional): Language
  }{

    \ifthenelse{\equal{#2}{}}%
      { % IF
        \def\fileAddress{elements/#1/#1.tex}%
      }%
      { % ELSE
        \def\fileAddress{elements/#1/#1\_#2.tex}%
      }%
    % \textcolor{green}{\fileAddress}


    % Update ribbons
    \updateRibbons{Article: \textbf{\myCiteEntry{\myarticleKey}{title}}\ribbonSpacer Section : \textbf{#1}}{#2}

    \IfSubStr{\onlyfiles}{#1}{ %%% only if it is part of \addonlyfiles list
      \InputIfFileExists{\fileAddress}
        {%
           % then
          % Add code here (it is run before file)
        }%
        {%
    %     % else
        \textcolor{myColorDanger}{\Huge #1 #2: CONTENT DOES NOT EXIST}
    %     % ...
        }%
    }

}



%%%%%%%%%%%%%%%%%%%%%%%%%%%%%%%%%%%%%%%%%%%%%%%%%%%%%%%%%%%%%%%%%%%%%%%%%%%%%%

%%%%%% SWITCH MODULE
% https://tex.stackexchange.com/questions/87656/turning-parts-of-text-on-and-off

% new environment for switchable areas
\NewDocumentEnvironment{hidecontent}{O{999}}%
% WARNING: if no argument is provided, content must NOT start with an empty paragraph. If argument is provided, it seems ok, but generally it is better to avoid the empty paragraph.
{%
  \ifnum\version<#1%
  \ifbool{hidecontentswitch}{\comment}%
  \ifbool{revealhiddenswitch}{\color{hideEnvColor}%
  }%
  \else%
  \fi%
  }%
{%
  \ifnum\version<#1%
  \ifbool{hidecontentswitch}{\endcomment}%
  \else%
  \fi%
  }%

% without color option (basic)
\NewDocumentEnvironment{hidecontentbasic}{O{999}}%
% WARNING: if no argument is provided, content must NOT start with an empty paragraph. If argument is provided, it seems ok, but generally it is better to avoid the empty paragraph.
{%
  \ifnum\version<#1%
  \ifbool{hidecontentswitch}{\comment}%
  \else%
  \fi%
  }%
{%
  \ifnum\version<#1%
  \ifbool{hidecontentswitch}{\endcomment}%
  \else%
  \fi%
  }%

%that's it ! now I need to use
% \begin{hidecontent}
% \end{hidecontent}


% with color option
\newcommand{\hide}[2][999]{%
    \ifnum\version<#1%
     \ifbool{hidecontentswitch}{}{%
      \ifbool{revealhiddenswitch}{%
      \ignorespaces\textcolor{hideColor}{%
          #2}}{%
          \ignorespaces#2}%
     }%
     \else%
     #2%
    \fi%
}%

% without color option (basic)
\newcommand{\hidebasic}[2][999]{%
    \ifnum\version<#1%
     \ifbool{hidecontentswitch}{}{%
      \ignorespaces#2%
     }%
    \else%
     #2%
    \fi%
}%

%%%% Shorthand to check if portfolio (yes) or article (no) document
\newcommand{\isPortfolio}[2]{%
  \ifthenelse{\equal{\jobname}{\detokenize{portfolio_document}}}
  {#1}
  {#2}
}%

%%%% Shorthand to check if article (yes) or portfolio (no) document
\newcommand{\isArticle}[2]{%
  \ifthenelse{\equal{\jobname}{\detokenize{document}}}
  {#1}
  {#2}
}%

%%%% Shorthand to check if print version
\newcommand{\isPrint}[2]{%
  \ifthenelse{\boolean{isPrintVersion}}%
  {#1}%
  {#2}%
}
%
%
%
%
% ARTICLE MAIN PARAMETERS
\input{./\chosenArticle.tex}
%
%
%
%
% SET BOOLEANS
\setbool{isMinimal}{false} % overwrites most booleans below
\setbool{isDraft}{false} % true or false
\setbool{isPrintVersion}{false} % true or false
%
%
%
%
% LOAD ALL SETTINGS FROM ARTICLE

% \usepackage{ifthen}
\usepackage{xifthen}
\usepackage{lipsum} % Lorem Ipsum



\newcommand{\setMinimalMode}
{
    \isMinimal
    {

        \ifthenelse{\equal{\jobname}{\detokenize{article_portfolio}}}
        {
            \includeonly{ %to run individuals fragments
            % elements/portfolio_meta_data/portfolio_meta_data,
            % elements/portfolio_title_page/portfolio_title_page,
            % elements/preface/preface
            }
        }
        {
            \addonlyfiles{
                % substance,
                % glossary,
                % abreviations,
                body
            }
        }

        % \setbool{isDraft}{true} % true or false
        \setbool{isPrintVersion}{false} % true or false
        \setbool{isDrawRibbons}{false} % true or false
        \setbool{isIncludeMeta}{false} % true or false
        \setbool{isIncludeArticleCover}{false} % true or false
        \setbool{isIncludeToC}{false} % true or false
        \setbool{isIncludeLoF}{false} % true or false
        \setbool{isIncludeLoT}{false} % true or false
        \setbool{isIncludeLoTB}{false} % true or false
        \setbool{isIncludeBiblio}{false} % true or false
        \setbool{isIncludeGlossary}{false} % true or false
        \setbool{isIncludeAbreviations}{false} % true or false
        \setbool{isPrintUnusedGlossary}{false} % true or false
        \setbool{isPrintUnusedAbreviations}{false} % true or false
        \setbool{isHighlightGlossaryAndAbreviations}{false}  % true or false
        \setbool{isIncludeMissingBibEntries}{false} % true or false
        \setbool{isIncludeFootnotes}{false} % true or false
        \setbool{isIncludeCitations}{false} % true or false
        \setbool{isIncludeCitationsInFootnotes}{false} % true or false
        \setbool{isIncludeTextBoxes}{false} % true or false
        \setbool{isMoveTextBoxesToEndOfArticle}{false}  % true or false
        \setbool{isConstrainFloats}{false}  % true or false
        \setbool{isIncludeArticleCoverImgInBody}{false}  % true or false
        \setbool{isCreditsInArticleBody}{false}  % true or false
        \setbool{isSplitInTwoColumns}{false} % true or false
        \setbool{isLandscapeMode}{false} % true or false
        \setbool{isIncludeAppendix}{false} % true or false

        %%%% PORTOFOLIO SPECIFIC:
        \setbool{isIncludePerArticleToC}{false} % true or false
        \setbool{isIncludePerArticleLoF}{false} % true or false
        \setbool{isIncludePerArticleLoT}{false} % true or false
        \setbool{isIncludePerArticleSubstance}{false} % true or false
        \setbool{isSplitBibliographyByChapter}{false}  % true or false

        %%%% OTHER SETTINGS:
        \pagecolor{black}% bg color
        \color{white}% text color
    }
    {}
}

\AfterPreamble{
    \setMinimalMode
}

%%% ===== MACRO TO RUN CONTENTS ONLY IN DRAFT VERSION
\newcommand{\isMinimal}[2]{%
    \ifthenelse{\boolean{isMinimal}}%
    {#1}%
    {#2}%
}

% make sure to update language in document as well
% ENGLISH (EN)
\ifthenelse{\equal{\wrtxLanguage}{en}}{
  \usepackage[english]{babel}
  \usepackage[en-GB]{datetime2}
  \newcommand{\wrtxLanguageLong}{english}
  %
  %
  %
  \newcommand{\TEXTcover}{Cover}
  \newcommand{\TEXTtoc}{Table of Contents}
  \newcommand{\TEXTlof}{List of Figures}
  \newcommand{\TEXTlot}{List of Tables}
  \newcommand{\TEXTpToc}{Partial Table of Contents}
  \newcommand{\TEXTpLof}{Partial List of Figures}
  \newcommand{\TEXTpLot}{Partial List of Tables}
  \newcommand{\TEXTlotb}{List of Textboxes}
  \newcommand{\TEXTmainSource}{Main source}
  \newcommand{\TEXTtextbox}{Textbox}
  \newcommand{\TEXTbibliography}{Bibliography}
  \newcommand{\TEXTchapter}{Article}
  \newcommand{\TEXTacknowledgements}{Acknowledgements}
  \newcommand{\TEXTpreface}{Preface}
  \newcommand{\TEXTpostface}{Postface}
  \newcommand{\TEXTquotation}{Quotation}
  \newcommand{\TEXTby}{by}
}{}%

% FRENCH (FR)
\ifthenelse{\equal{\wrtxLanguage}{fr}}{
  \usepackage[french]{babel}
  \usepackage[french]{datetime2}
  \newcommand{\wrtxLanguageLong}{french}
  %
  %
  %
  \newcommand{\TEXTcover}{Couverture}
  \newcommand{\TEXTtoc}{Table des Matières}
  \newcommand{\TEXTlof}{Liste des Figures}
  \newcommand{\TEXTlot}{Liste des Tableaux}
  \newcommand{\TEXTpToc}{Table Partielle des Matières}
  \newcommand{\TEXTpLof}{Liste Partielle des Figures}
  \newcommand{\TEXTpLot}{Liste Partielle des Tableaux}
  \newcommand{\TEXTlotb}{Liste des Notes}
  \newcommand{\TEXTmainSource}{Source principale}
  \newcommand{\TEXTtextbox}{Encadré}
  \newcommand{\TEXTbibliography}{Bibliographie}
  \newcommand{\TEXTchapter}{Article}
  \newcommand{\TEXTacknowledgements}{Remerciements}
  \newcommand{\TEXTpreface}{Avant-propos}
  \newcommand{\TEXTpostface}{Postface}
  \newcommand{\TEXTquotation}{Quotation}
  \newcommand{\TEXTby}{par}
}{}%

% \DTMlangsetup{ord=raise,showdayofmonth=false}

\usepackage[inline]{enumitem} % to customize lists

% Command to create a hidden item that takes no space in a list
\def\hiddenitem{%
    \item[]%
    \vskip-\baselineskip%
    \vskip-\parsep%
    \vskip-\itemsep%
}

% EXPANDED LIST WITH LINE BREAKS BETWEEN ITEMS
\newlist{wrtxListMeta}{enumerate}{1}
\setlist[wrtxListMeta]{
    label={},
    align=right, % label alignment
    % labelwidth=0.25cm,
    topsep=0pt, % reduce space above list
    itemsep=0pt, % Sets the space between items
    parsep=0pt, % Sets the space between paragraphs within the items
    leftmargin=100pt, % * or 0pt
    labelsep=10pt, % Space between the bullet and the item text
}

% COLLAPSED LIST WITH INLINE ITEMS
\newlist{wrtxListMetaCollapsed}{enumerate*}{1}
\setlist[wrtxListMetaCollapsed]{
% label=(\textbf{\arabic*})
}



\newenvironment{customlist} % Define the new environment
  {
    \ifthenelse{\boolean{isCollapseLists}}% Check if the argument is "enumerate"
    {\begin{wrtxListMetaCollapsed}}% If yes, start
    {\begin{wrtxListMeta}}%
    }%
  {%
  \ifthenelse{\boolean{isCollapseLists}}% Check if the argument is "enumerate"
  {\end{wrtxListMetaCollapsed}}% If yes, end
  {\end{wrtxListMeta}}%
  }%


% LIST STYLES USED INSIDE ARTICLE
% \newlist{<list name>}{<bases>}{<max depth>}
\newlist{wrtxListEnumerate}{enumerate}{1}
\newlist{wrtxListItemize}{itemize}{1}
% Define a shared macro for list settings
\def\wrtxListAlign{right}
\def\wrtxListTopsep{0pt} % space above list
\def\wrtxListItemsep{8pt} % space between items
\def\wrtxListParsep{0pt} % space between paragraphs within the items
\def\wrtxListLeftmargin{10pt} % space on the left. * or 0pt
\def\wrtxListLabelsep{10pt} % space between the bullet and the item text
%
\setlist[wrtxListEnumerate]{
    label=({\arabic*}),
    align=\wrtxListAlign, % label alignment
    % labelwidth=0.25cm,
    topsep=\wrtxListTopsep, % space above list
    itemsep=\wrtxListItemsep, % space between items
    parsep=\wrtxListParsep, % space between paragraphs within the items
    leftmargin=\wrtxListLeftmargin, % space on the left. * or 0pt
    labelsep=\wrtxListLabelsep, % space between the bullet and the item text
}
\setlist[wrtxListItemize]{
    label={--},
    align=\wrtxListAlign, % label alignment
    % labelwidth=0.25cm,
    topsep=\wrtxListTopsep, % space above list
    itemsep=\wrtxListItemsep, % space between items
    parsep=\wrtxListParsep, % space between paragraphs within the items
    leftmargin=\wrtxListLeftmargin, % space on the left. * or 0pt
    labelsep=\wrtxListLabelsep, % space between the bullet and the item text
}

\usepackage[dvipsnames]{xcolor} % to color comments
% Command to set background color
\usepackage[breakable, skins]{tcolorbox}

\usepackage{transparent}


\definecolor{myColorPrimary}{rgb}{0.3,0.2,0.2}
\definecolor{myColorSecondary}{rgb}{0.1,0.3,0.5}
\definecolor{myColorSuccess}{rgb}{0,1,0}
\definecolor{myColorWarning}{rgb}{1,0.5,0}
\definecolor{myColorDanger}{rgb}{1,0,0}
\definecolor{myGrayDark}{gray}{0.25}
\definecolor{myGrayMed}{gray}{0.5}
\definecolor{myGrayLight}{gray}{0.9}
\definecolor{bgColor}{rgb}{1,1,1}
\definecolor{textColor}{rgb}{0,0,0}
\definecolor{draftBgColor}{rgb}{0.1,0.05,0}
\definecolor{draftTextColor}{RGB}{255,255,255}


% Hidden content colors
\definecolor{hideColor}{rgb}{1,0.5,0.5}
\definecolor{hideEnvColor}{rgb}{0.5,1,0.5}


\colorlet{sectionHeaderColor}{myGrayDark}

\definecolor{sectionDraftContentsBgColor}{rgb}{0.9, 0.7, 0.5}
\colorlet{sectionFinalContentsBgColor}{bgColor}

\pagecolor{bgColor} % Set your desired background color here
\color{textColor} % Set your desired default text color here



%%%%%% Font parameters

\usepackage[T1]{fontenc} % for proper enconding of accents that can be copy pasted
\usepackage[utf8]{inputenc}
\usepackage{fontspec} % Allows for custom fonts (compatible with XeLaTeX or LuaLaTeX; but not with pdfLaTeX)
\usepackage[fontsize=12pt]{fontsize} % Set font size
\usepackage{unicode-math} %
% Set main font with fontspec (if all off, default to Computer Modern)
% \setmainfont{Cambria}
% \setmainfont{Arial}
% \setmainfont{Times New Roman}
% \setmainfont{Helvetica}
% \usepackage{helvet} % use helvetica alternative

% WHEN SETTING FONT FROM A FILE, YOU CAN SPECIFY IT LIKE THIS IF IT IS FAILING
% \setmainfont[
%     Path = /your/font/path/,
%   Extension = .otf ,
%   BoldFont = HelveticaNeueLTPro-Md.otf ,
% ]{HelveticaNeueLTPro-Roman.otf}


% \usepackage{ebgaramond} % Use the EB Garamond font (pdfLaTeX compatible)

%%% use sans serif text
% \renewcommand{\familydefault}{\sfdefault}

%%% use use dyslexic friendly font:
% \setmainfont{Open Dyslexic} % must first install on computer to work

%%% STANDARD WAY TO SET FONTS
% \setromanfont{Times New Roman}
% \setsansfont{Arial}
% \setmonofont{Consolas}[Scale=0.9]
% \setmathfont{Latin Modern Roman}


%%% MY WAY TO SET FONTS
\newcommand{\applyFonts}[2]{%
    % #1 first option
    % #2 backup
    \IfFontExistsTF{#1}
    {%
        \setmainfont{#1}%
    }%
    {% Fallback option
        \setmainfont{#2}%
    }%
}

%%%%%% SETTING FONTS WITH BACKUPS IF NOT AVAILABLE
%%% MAIN FONT
\newcommand{\myMainFont}{Helvetica}
\newcommand{\myMainFontBackup}{Arial}
\newcommand{\setMainFont}{%
    %%% Run immediately after \begin{document}, similar to \AtBeginDocument but the latter is outdated
    \isMinimal% check if in minimal mode
    {}% if yes, do not change font from default to speed up compilation
    {%
        \AfterEndPreamble{
            \applyFonts{\myMainFont}{\myMainFontBackup}
        }%
    }%
}%

%%% SET DEFAULT FONT
\setMainFont%



%%% DRAFT MAIN FONT
\newcommand{\myDraftFont}{Open Dyslexic}
\newcommand{\myDraftFontBackup}{Times New Roman}
\newcommand{\setDraftFont}{%
    %%% Run immediately after \begin{document}, similar to \AtBeginDocument but the latter is outdated
    \isMinimal% check if in minimal mode
    {}% if yes, do not change font from default to speed up compilation
    {%
        \AfterEndPreamble{
            \applyFonts{\myDraftFont}{\myDraftFontBackup}
        }%
    }%
}%


%%% TITLE FONT
\newcommand{\myTitleFont}{Times New Roman}
\newcommand{\myTitleFontBackup}{Times New Roman}
\newcommand{\setTitleFont}{%
    \isMinimal% check if in minimal mode
    {}% if yes, do not change font from default to speed up compilation
    {%
        \applyFonts{\myTitleFont}{\myTitleFontBackup}%
    }%
}%

%%% SUBTITLE FONT
\newcommand{\mySubtitleFont}{Times New Roman}
\newcommand{\mySubtitleFontBackup}{Times New Roman}
\newcommand{\setSubtitleFont}{%
    \isMinimal% check if in minimal mode
    {}% if yes, do not change font from default to speed up compilation
    {%
        \applyFonts{\mySubtitleFont}{\mySubtitleFontBackup}%
    }%
}%


%%%% Page measurements and spacing

\usepackage{geometry}

\ifthenelse{\boolean{isLandscapeMode}}
{%
  \geometry{landscape=true}%
}
{%
  \geometry{landscape=false}%
}

\newcommand{\wrtxWidthDigital}{150mm}
\newcommand{\wrtxTopDigital}{25mm}
\newcommand{\wrtxBottomDigital}{25mm}
\newcommand{\wrtxLeftDigital}{25mm}
\newcommand{\wrtxRightDigital}{25mm}
\newcommand{\wrtxBindingOffsetDigital}{0mm}

\newcommand{\wrtxWidthPrint}{150mm}
\newcommand{\wrtxTopPrint}{25mm}
\newcommand{\wrtxBottomPrint}{25mm}
\newcommand{\wrtxLeftPrint}{25mm}
\newcommand{\wrtxRightPrint}{25mm}
\newcommand{\wrtxBindingOffsetPrint}{8mm}
% DIMENSIONS FOR DIGITAL VERSION
\geometry{
  a4paper,
  width=\wrtxWidthDigital,
  top=\wrtxTopDigital,
  bottom=\wrtxBottomDigital,
  left=\wrtxLeftDigital,
  right=\wrtxRightDigital,
  bindingoffset=\wrtxBindingOffsetDigital,
  twoside % must be true for other features
}
\savegeometry{digital}


% DIMENSIONS FOR PRINTED VERSION
\newgeometry{
  a4paper,
  width=\wrtxWidthPrint,
  top=\wrtxTopPrint,
  bottom=\wrtxBottomPrint,
  left=\wrtxLeftPrint,
  right=\wrtxRightPrint,
  bindingoffset=\wrtxBindingOffsetPrint,
  twoside % must be true for other features
}
\savegeometry{print}


\newcommand{\wrtxGeometrySettingsInfo}{%
text width:\isPrint{\wrtxWidthPrint}{\wrtxWidthDigital};
top margin:\isPrint{\wrtxTopPrint}{\wrtxTopDigital};
bottom margin:\isPrint{\wrtxBottomPrint}{\wrtxBottomDigital};
left margin:
\isPrint{\wrtxLeftPrint}{\wrtxLeftDigital};
right margin:
\isPrint{\wrtxRightPrint}{\wrtxRightDigital};
binding offset:
\isPrint{\wrtxBindingOffsetPrint}{\wrtxBindingOffsetDigital}
}

\usepackage{setspace} % to change spacing between lines
% \onehalfspacing % or
%\singlespacing % or
\doublespacing %
\parskip=1em % spacing between paragraphs (0pt plus 1pt default)
\parindent=15pt % indent at start of each paragraph (15 default)
%%%%



\usepackage{scrextend} %%% Add margins to blocks of text

% \pagestyle{empty}
% \pagestyle{headings}

\usepackage{fancyhdr}
% \pagestyle{fancy} % Enable the fancy page style




\fancypagestyle{styleDigital}{
  \fancyhf{}% Clear header/footer
  % O = odd; E = even; L/R/C = left/right/center
  \fancyhead[RO, RE]{\wrtxArticleTitle}
  \fancyhead[LO, LE]{\wrtxPrintOrDigitalMarker}
  \fancyhead[CO, CE]{
    % \verticalCenterIcon{\includegraphics[scale=0.05]{../../articles_common_files/assets/icons/placeholder.png}}
  }
  % \fancyhead[CO, CE]{\textcolor{wrtxColorPrimary}{\hyperlink{toc}{\leftmark}}}
  \fancyfoot[RO, RE]{
    \textcolor{wrtxColorPrimary}{
      % \hyperlink{title}{\wrtxCiteEntry{\wrtxarticleKey}{author}}
    }
    \wrtxAbsolutePagination
  }
  \fancyfoot[CE, CO]{
    \wrtxRelativePagination
  }
}

%%% DEFINE ANOTHER STYLE BASED ON THIS ONE
\fancypagestyle{styleDigital-NoPage}[styleDigital]{
\fancyfoot[]{} % empty footer
\fancyfoot[RO, RE]{\wrtxAbsolutePagination}
\fancyfoot[LO, LE]{\isDraft{\color{wrtxColorWarning}page without pagination in final v.}{}}
}



\fancypagestyle{stylePrint}{
  \fancyhf{}% Clear header/footer
  % O = odd; E = even; L/R/C = left/right/center
  % HEAD
  \fancyhead[RO, LE]{\wrtxArticleTitle}
  % \fancyhead[LO, RE]{\textcolor{wrtxColorPrimary}{\hyperlink{toc}{\leftmark}}}

  % FOOT
  \fancyfoot[LO, RE]{
    % \textcolor{wrtxColorPrimary}{\hyperlink{title}{\wrtxCiteEntry{\wrtxarticleKey}{author}}}
    \wrtxAbsolutePagination
     }
  \fancyfoot[RO, LE]{
    \textcolor{wrtxColorPrimary}{\wrtxRelativePagination}
    }
  \fancyfoot[CO]{
    % \verticalCenterIcon{\includegraphics[scale=0.05]{../../articles_common_files/assets/icons/placeholder.png}}
    \isDraft{\textcolor{wrtxColorWarning}{ODD (RIGHT)}{}}
  }
  \fancyfoot[CE]{
    % \verticalCenterIcon{\includegraphics[scale=0.05]{../../articles_common_files/assets/icons/placeholder.png}}
    \isDraft{\textcolor{wrtxColorWarning}{EVEN (LEFT)}{}}
  }
}

%%% DEFINE ANOTHER STYLE BASED ON THIS ONE
\fancypagestyle{stylePrint-NoPage}[stylePrint]{
  \fancyfoot[LO,RE, RO, LE]{} % empty footer
  \fancyfoot[LO, RE]{\wrtxAbsolutePagination}
  \fancyfoot[RO, LE]{\isDraft{\color{wrtxColorWarning}page without pagination in final v.}{}}
}


\newcommand{\verticalCenterIcon}[1]{
    \raisebox{-0.5\height}{%center vertically in line
    \scalebox{-1}[1]{%horizontal flipping
      #1
    }
  }
}

%%% Relative pagination (within article)
\newcommand{\wrtxRelativePagination}{%
  \isDraft
  {\hyperlink{toc}{\textcolor{wrtxGrayMed}{\thepage}} \textcolor{wrtxColorPrimary}{out of\totalPagesInArticleBody}}%
  {
    \textcolor{wrtxColorPrimary}{\hyperlink{toc}{\thepage}}%
  }
}

%%% Absolute pagination (whole document)
\newcommand{\wrtxAbsolutePagination}{%
  \isDraft{\textcolor{wrtxColorPrimary}{Absolute page: \textcolor{wrtxGrayMed}{\abspagenumber}/\ztotpages}}{}%
}

%%% Article title
\newcommand{\wrtxArticleTitle}{%
  \hyperlink{\wrtxarticleKey}{\textcolor{wrtxColorPrimary}{\setTitleFont\citefield{\wrtxarticleKey}{title}}}%
}

%%% Print/Digital marker
\newcommand{\wrtxPrintOrDigitalMarker}{%
  \isDraft{%
  \color{wrtxColorWarning}\isPrint{PRINT VERSION}{DIGITAL VERSION}
  }{}
}



% to exclude "chapter #" from footer:
% \renewcommand{\chaptermark}[1]{\markboth{\MakeUppercase{#1}}{}} %remove \makeuppercase{} to keep it normal case

% change ruler style and color
\renewcommand{\headrule}{\hbox to\headwidth{\color{wrtxColorPrimary}\leaders\hrule height \headrulewidth\hfill}}
\renewcommand{\footrule}{\hbox to\headwidth{\color{wrtxColorPrimary}\leaders\hrule height \headrulewidth\hfill}}

% set ruler dimensions
\renewcommand{\headrulewidth}{0pt}
\renewcommand{\footrulewidth}{0pt}
\setlength{\headheight}{15pt}


\fancyheadoffset{0cm}


\usepackage{sectsty}
\sectionfont{\centering} % center section titles

% Used in all indivual article sections except the article title
\newcommand{\mySectionTitleLinker}{}
\newcommand*{\mySectionTitle}[1]{
    \renewcommand{\mySectionTitleLinker}{%
        \myarticleKeyCore:\detokenize{#1}%
    }
    \phantomsection % ensures linking with hyperref to exact page
    \hypertarget{\mySectionTitleLinker}{}
    % Create the inline heading
    \section*{%
        \hyperlink{toc}{%
            \setTitleFont\textcolor{myColorPrimary}{#1}%
            }%
        }\label{\mySectionTitleLinker}%
    % Set the mark (for headers/footers)
    \markboth{\uppercase{#1}}{}
    % ADD SPACE ABOVE LINE
    % \addtocontents{toc}{\protect\vspace*{3ex}}
    % ADD TO TOC WITHOUT NUMBER PAGE (do not use addtocontents for this, it will cause issues with partial lists)
    % \cftaddtitleline{toc}{section}{Alternative strategy}{}
    % ADD TO TOC WITH NUMBER PAGE
    \addcontentsline{toc}{section}{\noindent%
                        \hyperref[\mySectionTitleLinker]{\textbf{#1}}%
                        }
}






\usepackage{titlesec} % to control spacing around section titles

\usepackage{titlecaps} % capitalize words

\usepackage{lineno} % to add line numbering for submission

\usepackage{import} % to handle nested imports

\usepackage{xparse}

\usepackage{totcount} % to store counts between runs




\usepackage{dashrule} % for dashed hrules (hdashrule)
\newcommand*{\mydashrule}{
\smallskip
{\color{gray}\hdashrule{\linewidth}{1pt}{1pt}}
\medskip
}

\usepackage[utf8]{inputenc}
\usepackage{graphicx}
\graphicspath{%
  {./}
  % Common files
  {./../articles_common_files/assets/}% Portfolio
  {./../../articles_common_files/assets/}% Article
  % Article files
  {./assets/} % single articles
  {../articles/\myarticleKeyCore/assets/} % portfolio
}

\usepackage{verbatim} %\begin{comment} and end to comment out long sections
\usepackage{amsmath} % for text in math mode.
\usepackage[nointegrals]{wasysym} % diameter symbol. Nointegrals is t avoid incompatibility with amsmath
\usepackage{microtype} %improves justification, change letter spacing
\usepackage{xspace} % to add  space if necessary - seems to cause more trouble than help



%%%%% BIBLIOGRAPHY
%% Troubleshooting: check README
\usepackage{csquotes}

\ifthenelse{\boolean{isIncludeCitationsInFootnotes}}
{%
    \newcommand{\myCiteStyle}{authoryear}%
    \newcommand{\myBibStyle}{authoryear}%
}
{%
    \newcommand{\myCiteStyle}{numeric}%
    \newcommand{\myBibStyle}{numeric}%
}



\usepackage[
    %sorting=none,
    sorting=nty, %name, title, year
    %%% CITATION STYLES: "style applies to both citations in text and display in printed bibliography, citestyle and bibstyles splits
    % Style option examples: numeric, authoryear, authortitle, verbose
    % style=authoryear,
    citestyle=\myCiteStyle,
    bibstyle=\myBibStyle,
    backend=biber,
    datamodel=mydatamodels,
    doi=false,
    isbn=false,
    url=false,
    eprint=false
    % maxcitenames=2
    % maxbibnames=1,
    % minbibnames=3
]{biblatex} %Imports biblatex package
%Import the bibliography file(s)
%%%% bibliographic sources
\newcommand{\loadBibIfExists}[1]%%% load bib file only if it exists
{\IfFileExists{#1}
    {\addbibresource{#1}}%
    {}%
}
\isPortfolio
{
    \loadBibIfExists{../../../articles_common_files/biblatex_files/bibliography.bib}% FOR PORTFOLIO
}
{
    \isArticle{%%% make sure it really is an article, and not e.g. a standalone file
        \loadBibIfExists{../../articles_common_files/biblatex_files/bibliography.bib}% FOR SINGLE ARTICLES
    }{}
}


\DeclareBibliographyCategory{myMediationArticles}
\addtocategory{myMediationArticles}{\myarticleKey}


%%% Include all biblatex items in bibliography, even if not cited in the text
% \nocite{*}

%%% MY OWN TYPE AND RESPECTIVE FIELDS
\input{../../../../articles_common_files/settings_and_packages/biblatex_settings/my_biblatex_types.tex}


\AtDataInput{\stepcounter{%
    totalCitationsAltogether%
    }} % Increment with each citation
% AtDataInput{} is triggered at the instant of each citation, whereas AtEveryBibitem{} counts only after Bib printing




% Command to add bibliography section
\newcommand{\addBibliography}{
    \isPortfolio{%
    \newcommand{\myCitationCounter}{1} %%% Print in ANY case since 1<>0
    }
    {
    \newcommand{\myCitationCounter}{%
        \totvalue%
        {totalCitationsInArticle:\myarticleKeyCore}%
        }
    }

    \ifthenelse{\boolean{isIncludeBiblio} \and \myCitationCounter>0}{
        \newpage
        \begin{SplitColumnsInTwo}%[true]
        \updateRibbons{\textbf{\TEXTbibliography}}{}
        \mySectionTitle{\TEXTbibliography}
        % This document contains \total{totalCitationsInArticle:\myarticleKeyCore}\ citation(s).
        \printbibliography[
            heading=none, % "bibintoc" adds the title to the table of contents. "none" to exclude.
            % title={My bibliography title} % Add title above bibliography
            % type=report,
            notcategory=myMediationArticles% Exclude my articles
            ]
        \end{SplitColumnsInTwo}
    }{}
}




\newcommand{\myCite}[1]{%
    \ifthenelse{\boolean{isIncludeCitations}}{% whether or not to include citations in article
    \stepcounter{%
    totalCitationsInArticle:\myarticleKeyCore% COUNTS REPEATS!
    }%
        \ifthenelse{\boolean{isIncludeCitationsInFootnotes}}{% whether to show citations in footnotes or not
            \footcite{#1}%
        }{%
            \cite{#1}%
        }%
    }{}
}



% raise inline text of different size to be aligned vertically
\newcommand*\raiseup[2]{%
        \begingroup%
        \setbox0\hbox{#1\strut #2}%
        \leavevmode%
        % Change formula to adjust height
        \raise\dimexpr (\ht\strutbox - \ht0)/3 \box0%
        \endgroup%
}

% Change missing reference message for Biblatex
\usepackage{xpatch}
\newcommand{\myUnknownRefSymbol}{????}
\makeatletter%
\def\abx@missing@entry#1{%
\raiseup{\tiny}{%
    \textcolor{myColorDanger}{%
        \abx@missing{[\myUnknownRefSymbol\ #1 \myUnknownRefSymbol]}%
    }%
    }%
}
\makeatother%


%%% Custom cite commands

% command to apply to prenotes and custom inputs
\newcommand{\genericPrenote}[1]
{\textcolor{myGrayMed}{#1\addcolon\space}}
% custom field format
\DeclareFieldFormat{myLabelFormat}{\genericPrenote{\titlecap{#1}}}
%
% field format for prenotes
\DeclareFieldFormat{prenote}
{\genericPrenote{#1}}
%
% custom empty entry format
\DeclareFieldFormat{myEntrymptyEntry}{\textcolor{red}{#1}}
%
% field format for DOI specifically (auto applies)
\DeclareFieldFormat{doi}{%
%   \mkbibacro{DOI} % prints label by default
  \ifhyperref
    {\href{https://doi.org/#1}{\nolinkurl{#1}}}
    {\nolinkurl{#1}}}
%
% field format for URL specifically (auto applies)
\DeclareFieldFormat{url}{%
% \mkbibacro{URL} % prints label by default
\url{#1}}
%
% field format for ISSN specifically (auto applies)
\DeclareFieldFormat{issn}{%
% \mkbibacro{ISSN} % prints label by default
#1}
% Define separator between citation and postnote
\renewcommand{\postnotedelim}{\space}

% \usepackage{natbib}
% \setcitestyle{comma}

% Macro to check if an entry is empty, and print something if TRUE or FALSE
\newcommand{\checkIfNoEntryFound}[2]{%
    \iffieldundef{\myEntry}% IS IT A FIELD ?
    {%
        \ifnameundef{\myEntry}% IS IT A NAME ?
        {%
            \iflistundef{\myEntry}% IS IT A LIST ?
            {#1}%
            {#2}%
        }%
        {#2}%
    }%
    {#2}%
}%
\newcommand{\checkIfNoEntryFoundConditional}[2]{%
    \ifthenelse{\boolean{isIncludeMissingBibEntries}}
    {%
    % \textcolor{myColorSuccess}{TRUE}
    #2%
    }%
    {%
        \checkIfNoEntryFound{#1}{#2}
    }%
}%


\DeclareCiteCommand{\myCiteCommand}%
    {% PRENOTE
    % \textcolor{orange}{\myEntry}
    \checkIfNoEntryFound{%
            % \textcolor{red}{Missing entry!}
        }{%
            \renewcommand{\genericPrenote}[1]{#1}% to remove formatting from prenote
            \usebibmacro{prenote}%
        }%
    }
    {
        \iffieldundef{\myEntry}% IS IT A FIELD ?
        {%
            \ifnameundef{\myEntry}% IS IT A NAME ?
            {%
                \iflistundef{\myEntry}% IS IT A LIST ?
                {%
                    % \ifthenelse{\boolean{isIncludeMissingBibEntries}}{%
                    \isDraftDebugger{
                        \printtext[myEmptyEntry]
                        {\na}
                        }{}%
                        % If it is none of the below
                    % }{}%
                }%
                {%
                    \printlist{\myEntry}% if it is a biblatex list
                }%
            }%
            {%
                \printnames{\myEntry}% if it is a biblatex name
            }%
        }%
        {%
            \printfield{\myEntry}% if it is a biblatex field
        }%
    }
    {}
    {% POSTNOTE
    \checkIfNoEntryFound{%
            % Missing entry!
        }{%
            \usebibmacro{postnote}%
        }%
    }

\DeclareCiteCommand{\myCiteWithLabelCommand}%
    {%
        \checkIfNoEntryFoundConditional{%
            % Missing entry!
        }{%
            \item[%
                \iffieldundef{prenote}%
                {%
                    \printtext[myLabelFormat]{\myEntry:}%
                    % \setunit{\prenotedelim}
                }%
                {%
                    \usebibmacro{prenote}%
                }%
            ]
        }%
    }%
    {%
        \iffieldundef{\myEntry}% IS IT A FIELD ?
        {%
            \ifnameundef{\myEntry}% IS IT A NAME ?
            {%
                \iflistundef{\myEntry}% IS IT A LIST ?
                {%
                    \ifthenelse{\boolean{isIncludeMissingBibEntries}}{%
                        \isDraftDebugger{
                            \printtext[myEmptyEntry]{\na}%
                            }{}%
                    }{}%
                }%
                {%
                    \printlist{\myEntry}% if it is a biblatex list
                }%
            }%
            {%
                \printnames{\myEntry}% if it is a biblatex name
            }%
        }%
        {%
            \printfield{\myEntry}% if it is a biblatex field
        }%
    }%
    {%
        \multicitedelim%
    }%
    {%
        \checkIfNoEntryFoundConditional{%
            % Missing entry!
        }{%
            \usebibmacro{postnote}%
        }%
    }%


% placeholder command to hold entry label
\newcommand{\myEntry}{}
% entrypoint commands for the citation that picks on the above cite command to generalize it
\NewDocumentCommand{\myCiteEntryWithLabel}
{
    m% #1 citation key
    m% #2 citation entry key
    O{}% #3 prenote
    O{}% #4 postnote
}{%
    \renewcommand{\myEntry}{#2}%
    \myCiteWithLabelCommand[#3][#4]{#1}%
}%

% entrypoint command for the citation that picks on the above cite command to generalize it
\NewDocumentCommand{\myCiteEntry}
{
    m% #1 citation key
    m% #2 citation entry key
    O{}% #3 prenote
    O{}% #4 postnote
}{%
    \renewcommand{\myEntry}{#2}%
    % \textcolor{red}{#2}
    \myCiteCommand[#3][#4]{#1}%
}%


\usepackage{tikz} % create images with tikz
\usetikzlibrary{positioning} %tikz positioning
\usetikzlibrary{trees} %more tree options
\usetikzlibrary{shapes.geometric} % add shapes
\usetikzlibrary{intersections} % for pyramid
\usetikzlibrary{calc} % also for pyramid
\usetikzlibrary{tikzmark}
\usetikzlibrary{decorations.text} % add text to curve
\usetikzlibrary{decorations.pathreplacing,calligraphy}
\usetikzlibrary{arrows}
\usetikzlibrary{decorations.markings}
\usetikzlibrary{3d}
\usetikzlibrary{fadings}
\usetikzlibrary {shadows.blur}
\usetikzlibrary{backgrounds}
\usepackage{pgf-pie}


%%%%%%%%%%%%%%%%%%%%%%%%%%%%%%%%%%%%%%%%%%%%%%%%%%%%%%%%%%%%%%%%%%%%%%%%%%%%%%

%%% ===== MACRO TO RUN CONTENTS ONLY IN DRAFT VERSION
\newcommand{\isDraft}[2]{%
    \ifthenelse{\boolean{isDraft}}%
    {#1}%
    {#2}%
}
\newcommand{\draftVersionOnly}[1]{%
    %If document mode is draft...
    \isDraft{%
        #1%
    }
    {}
}

\usepackage{draftwatermark}
\DraftwatermarkOptions{stamp=false} % watermark off

\newcommand{\myWatermark}{
    \DraftwatermarkOptions{stamp=true} % watermark on
    \isPortfolio{ % IF PORTFOLIO
        \SetWatermarkText{PORTFOLIO DRAFT} % use text as watermark
    }{ % IF SINGLE ARTICLE
        \SetWatermarkText{ARTICLE DRAFT} % use text as watermark
    }
    % \SetWatermarkText{\tikz{\node[opacity=0.2]{\includegraphics{example-image-a}};}} % to use image instead
    \SetWatermarkScale{0.5}
    \SetWatermarkColor[gray]{0.9}
    % \SetWatermarkColor[rgb]{0,1,0}
    \SetWatermarkLightness{0.05}
    \SetWatermarkAngle{45}
}

% Change settings in draft mode
\draftVersionOnly{%
    \color{draftTextColor}% text color
    \pagecolor{draftBgColor}% bg color
    \setDraftFont % font
    \onehalfspacing % or
    \doublespacing %
    %%%% Watermark draft
    \myWatermark
    %%% TWO OPTIONS TO HIGHLIGHT LABELS:showkeys and showlabels
    % \usepackage[
    %     % notref,
    %     notcite% to stop printing citation keys
    % ]{showkeys}
    \usepackage[
        % inner, % print keys inside text margins. Other options: outer [default]
        inline % marginal [default]—put notes in the margin
        ]
    {showlabels}% already part of showkeys
    \renewcommand{\showlabelfont}{\slshape\color{myColorSuccess}\tiny}

    %%%% Show structural frame
    \isMinimal
    {}
    {%
        \usepackage{showframe}
        \renewcommand*\ShowFrameColor{\color{myColorSecondary}}
        \renewcommand*\ShowFrameLinethickness{1pt}
        \usepackage{layout} %%% TO GET LAYOUT INFOMATION
        \AtEndDocument{\newpage\updateRibbons{LAYOUT}{}\layout}
    }%
}{}




\NewDocumentCommand{\isDraftDebugger}
{
    m
    m
    O{myColorWarning}
}{%
    \textcolor{#3}{%
        %
            {%
            \isDraft%
            {\texttt{\scriptsize#1}#2}%
            {#2}%
            }%
    }%
}


\NewDocumentCommand{\myBreakMessage}
{
    O{myColorWarning}% #1 decoration color
    O{myColorSuccess}% #2 text color
    m% #3 mandatory message
}
{%
    {%
    \noindent\centering%
    \textcolor{#1}%
    {%
    \noindent\dotfill\\%
    \setstretch{0.3}%
    \noindent\dotfill \textcolor{#2}{#3} \dotfill\\%
    \noindent\dotfill\\%
    }%
    }%
}



\usepackage[section]{placeins} %force image floats to stay in their section
 % ... and to make them stay in their SUBSECTION:


\usepackage{float} % for custom floats (to behave like figures and tables). RUN BEFORE ALL FIGURES AND OTHER FLOATS PACKAGES
\restylefloat{figure}%FIXES CAPTION IN WRAPFIGURE BUG %https://tex.stackexchange.com/questions/128643/cannot-use-caption-in-wrap-figure

\newcommand{\myFloatBarrier}{%
  \ifthenelse{\boolean{isConstrainFloats}}
  {%
  \isDraftDebugger{%
    \noindent\myBreakMessage[yellow][blue]{FLOAT BARRIER}%
  }{}% visually mark float barrier in pdf
  \FloatBarrier%
  }%
  {}
}


\usepackage{soul} % to strikethrough
\usepackage{textcomp}

%%% to get darkmode (already implemented in custom draftmode)
% \usepackage{darkmode}
% \enabledarkmode

%%% LINKING IMAGES/TABLES TO LIST OF IMAGES/TABLES
%https://tex.stackexchange.com/questions/24283/how-to-link-images-to-their-entry-in-list-of-figures



\newcommand{\na}{
  \textcolor{red}{NA} }

  
\newcommand{\colorsubsection}[1]{%
\colorbox{wrtxColorSecondary!30}{\parbox{\dimexpr\textwidth-2\fboxsep}{\hspace{0.5em}#1}}}
\newcommand{\colorsubsubsection}[1]{%
\colorbox{wrtxColorSecondary!10}{\parbox{\dimexpr\textwidth-2\fboxsep}{\hspace{0.5em}#1}}}


\newcommand{\metaInfoSettings}[1]{
    {
    \newgeometry{top=0.3cm,bottom=0.3cm,left=0cm,right=0cm} % Change the margins
    %%% STYLING META INFORMATION SECTION
    %%%%%% META-INFO PAGES PARAMETERS
    \singlespacing %spacing between lines
    \parskip=0em % spacing between paragraphs (0pt plus 1pt default)
    \parindent=0pt % indent at start of each paragraph (15 default)
    % \titlespacing*{\section}{1pt}{1pt}{1pt}
    \titlespacing*{\paragraph}{0pt}{0pt}{0.5em}
    % Format subsection titles for metadata
    %
    %
    %%% Subsection
    \titlespacing*{\subsection}{0pt}{0pt}{5pt}
    \titleformat{\subsection}
    {\vbox{\rule{1\textwidth}{1pt}\vspace{-1.5ex}}\normalsize\color{wrtxGrayMed}\bfseries}
    {\thesubsection}
    {0.1em}
    {\colorsubsection}
    %
    %
    %
    %%% Subsubsection
    \titlespacing*{\subsubsection}{0pt}{0pt}{5pt}
    \titleformat{\subsubsection}
    {\vbox{\rule{1\textwidth}{0.1pt}\vspace{-1.4ex}}\small\color{wrtxGrayMed}}
    {\thesubsubsection}
    {0.1em}
    {\colorsubsubsection} % change subsubsection style

    % Font size in metadata section
    \scriptsize

    %%% CONTENTS
    #1
    \clearpage
    }
}


\newcommand{\mainSourceMetadata}{
    \subsection*{Main source of information}


        \begin{wrtxListMeta}
        %%% START WITH EMPTY ITEM
            \hiddenitem % EMPTY ITEM NECESSARY TO AVOID ERROR
        %%% KEY
            \item[\wrtxListLabelStyle{Main source key}]\ \expandafter\detokenize\expandafter{\mainSourceKey}
        %%% TITLE
            \wrtxCiteEntryWithLabel{\mainSourceKey}{title}[][]%
            \wrtxCite{\mainSourceKey}% Include citation
        %%% DOI
            \wrtxCiteEntryWithLabel{\mainSourceKey}{doi}[DOI][]%
        %%% ISSN
            \wrtxCiteEntryWithLabel{\mainSourceKey}{issn}[ISSN][]%
        %%% url
            \wrtxCiteEntryWithLabel{\mainSourceKey}{url}[URL][]%
        %%% Language
            \wrtxCiteEntryWithLabel{\mainSourceKey}{language}[][]%
        %%% Publication date
            \wrtxCiteEntryWithLabel{\mainSourceKey}{year}[][]%
            \wrtxCiteEntryWithLabel{\mainSourceKey}{month}[][]%
        %%% authors(s)
            \wrtxCiteEntryWithLabel{\mainSourceKey}{author}[][]%
        %%% Journal
            \wrtxCiteEntryWithLabel{\mainSourceKey}{journaltitle}[Journal:][]%
        %%% Volume
            \wrtxCiteEntryWithLabel{\mainSourceKey}{volume}[][]%
        %%% Pages
            \wrtxCiteEntryWithLabel{\mainSourceKey}{pages}[][]%
        %%% Abstract
            \wrtxCiteEntryWithLabel{\mainSourceKey}{abstract}[][]%
        \end{wrtxListMeta}
}


\newcommand{\wrtxArticleMetadata}{
    \subsection*{Mediation paper}

    \begin{addmargin}
        [1em] % LEFT
        {1em} % RIGHT
        To cite: \fullcite{\wrtxarticleKey}
        \\
    \end{addmargin}

    \begin{wrtxListMeta}
        %%% START WITH EMPTY ITEM
        \hiddenitem % EMPTY ITEM NECESSARY TO AVOID ERROR
        %%% TITLE
        \wrtxCiteEntryWithLabel{\wrtxarticleKey}{title}[][]%
        %%% SUBTITLE
        \wrtxCiteEntryWithLabel{\wrtxarticleKey}{subtitle}[][]%
        %%% DISCIPLINE(S)
        \wrtxCiteEntryWithLabel{\wrtxarticleKey}{discipline}[Disciplines:][]%
        %%% ABSTRACT
        \wrtxCiteEntryWithLabel{\wrtxarticleKey}{abstract}[Summary:][]%
        %%% KEYWORD(S)
        \wrtxCiteEntryWithLabel{\wrtxarticleKey}{keywords}[][]%
    \end{wrtxListMeta}

    %%%%% MEDIATION CONTRIBUTOR(S):
    \subsubsection*{Credits  \& Contributions}
    \begin{wrtxListMeta}
        %%% START WITH EMPTY ITEM
        \hiddenitem % EMPTY ITEM NECESSARY TO AVOID ERROR
        %%% AUTHOR(S)
        \wrtxCiteEntryWithLabel{\wrtxarticleKey}{author}[][]%
        %%% ILLUSTRATOR(S)
        \wrtxCiteEntryWithLabel{\wrtxarticleKey}{illustrator}[][]%
        %%% REVIEWER(S)
        \wrtxCiteEntryWithLabel{\wrtxarticleKey}{reviewer}[][]%
        %%% THANKS
        \wrtxCiteEntryWithLabel{\wrtxarticleKey}{thank}[][]%
        %%% COPYRIGHT
        \wrtxCiteEntryWithLabel{\wrtxarticleKey}{copyright}[][]%
    \end{wrtxListMeta}
    %%%%% TARGET:
    \subsubsection*{Target specifications}
    \begin{wrtxListMeta}
        %%% START WITH EMPTY ITEM
        \hiddenitem % EMPTY ITEM NECESSARY TO AVOID ERROR
        %%% TARGET PUBLICATION:
        \wrtxCiteEntryWithLabel{\wrtxarticleKey}{targetPublication}[Target publication: ][]
        %%% AUDIENCE LEVEL:
        \wrtxCiteEntryWithLabel{\wrtxarticleKey}{audienceLevel}[Audience level: ][]
        % environment where entries are always shown:
        \begin{alwaysShowMeta}
            %%% WORD LIMITS:
            \wrtxCiteEntryWithLabel{\wrtxarticleKey}{wordMin}[World limits: ][]
            ---
            \wrtxCiteEntry{\wrtxarticleKey}{wordMax}
            %%% CHARACTER LIMITS:
            \wrtxCiteEntryWithLabel{\wrtxarticleKey}{charMin}[Character limits: ][]
            ---
            \wrtxCiteEntry{\wrtxarticleKey}{charMax}
        \end{alwaysShowMeta}
    \end{wrtxListMeta}
}

%%% Environment where metadata is always shown, regardless of settings
\newenvironment{alwaysShowMeta}{
    \begingroup%
    \booltrue{isIncludeMissingBibEntries} % Set boolean to true at start
}
{%
    \endgroup% Revert to previous value at end
}%


\newcommand{\wrtxLuatexMessage}{
    \ifluatex \directlua{tex.print(status.banner)}
    \else \textcolor{wrtxColorWarning}{This document was not compiled with \textbf{Lua\TeX}}
    \fi
}

\newcommand{\technicalMetadata}{
    \subsection*{Technical data}
    \begin{wrtxListMeta}
        %
        \item[\wrtxListLabelStyle{Compilation date:}]\ \today
        %
        \item[\wrtxListLabelStyle{Compilation time:}]\ \DTMcurrenttime
        %
        \item[\wrtxListLabelStyle{Compilation duration:}] \elapsedInt.\elapsedFrac\ seconds (previous single run)
        %
        \item[\wrtxListLabelStyle{Lua\TeX\ version:}] \wrtxLuatexMessage
        %
        \item[\wrtxListLabelStyle{Paper format:}] \csname Gm@paper\endcsname
        %
        \item[\wrtxListLabelStyle{Geometry:}] \wrtxGeometrySettingsInfo
        %
    \end{wrtxListMeta}
}


\newcommand{\docOptionsItem}[1]{%
    \isPortfolio{}{%
        \item[\wrtxListLabelStyle{#1}]\ \ifthenelse{\boolean{#1}}{\textcolor{wrtxColorSuccess}{true}}{\textcolor{wrtxColorDanger}{false}}%
    }%
}%

\newcommand{\docOptionsMetadata}{
    \subsection*{Document options}
    %%%%%%%%%%%%%%%%%%%%%%%%%%%%%%%%%%%%%%%%%%%%%%%%%%%%%%%%%%%%%%%
    %%%%% FIRST COLUMN
    \begin{minipage}[t]{0.40\textwidth} % Adjust width as needed
        \begin{wrtxListMeta}[
            leftmargin=70pt,
            labelsep=0pt
            ]
            %%% START WITH EMPTY ITEM
            \hiddenitem % EMPTY ITEM NECESSARY TO AVOID ERROR
            %
            \item[\wrtxListLabelStyle{Article key (core)}]\ \wrtxarticleKeyCore
            \item[\wrtxListLabelStyle{Article key (full)}]\ wrtxarticle:\wrtxarticleKeyCore:\wrtxLanguage
            \item[\wrtxListLabelStyle{Language}]\ \wrtxLanguageLong (\wrtxLanguage)
            \item[] % empty gap
            %
            \docOptionsItem{isMinimal}
            %
            \docOptionsItem{isDraft}
            %
            \docOptionsItem{isLandscapeMode}
            %
            \docOptionsItem{isSplitInTwoColumns}
            %
            \docOptionsItem{isPrintVersion}
            %
            \docOptionsItem{isDrawRibbons}
            %
            \docOptionsItem{isIncludeMeta}
            %
            \docOptionsItem{isIncludeArticleCover}
            %
        \end{wrtxListMeta}
    \end{minipage}
    % \hfill % This adds space between the minipages
    %%%%%%%%%%%%%%%%%%%%%%%%%%%%%%%%%%%%%%%%%%%%%%%%%%%%%%%%%%%%%%%
    %%%%% SECOND COLUMN
    \begin{minipage}[t]{0.30\textwidth}
        \begin{wrtxListMeta}[
            leftmargin=40pt,
            labelsep=0pt
            ]
            %%% START WITH EMPTY ITEM
            \hiddenitem % EMPTY ITEM NECESSARY TO AVOID ERROR
            %
            \docOptionsItem{isIncludeToC}
            %
            \docOptionsItem{isIncludeLoF}
            %
            \docOptionsItem{isIncludeLoT}
            %
            \docOptionsItem{isIncludeLoTB}
            %
            \docOptionsItem{isIncludeBiblio}
            %
            \docOptionsItem{isIncludeGlossary}
            %
            \docOptionsItem{isIncludeAbreviations}
            %
            \docOptionsItem{isPrintUnusedGlossary}
            %
            \docOptionsItem{isPrintUnusedAbreviations}
            %
            \docOptionsItem{isHighlightGlossaryAndAbreviations}
            %
            \docOptionsItem{isIncludeMissingBibEntries}
            %
        \end{wrtxListMeta}
    \end{minipage}
        %%%%%%%%%%%%%%%%%%%%%%%%%%%%%%%%%%%%%%%%%%%%%%%%%%%%%%%%%%%%%%%
    %%%%% THIRD COLUMN
    \begin{minipage}[t]{0.30\textwidth}
        \begin{wrtxListMeta}[
            leftmargin=40pt,
            labelsep=0pt
            ]
            %%% START WITH EMPTY ITEM
            \hiddenitem % EMPTY ITEM NECESSARY TO AVOID ERROR
            %
            \docOptionsItem{isIncludeFootnotes}
            %
            \docOptionsItem{isIncludeCitations}
            %
            \docOptionsItem{isIncludeCitationsInFootnotes}
            %
            \docOptionsItem{isIncludeTextBoxes}
            %
            \docOptionsItem{isMoveTextBoxesToEndOfArticle}
            %
            \docOptionsItem{isConstrainFloats}
            %
            \docOptionsItem{isIncludeArticleCoverImgInBody}
            %
            \docOptionsItem{isCreditsInArticleBody}
            %
            \docOptionsItem{hidecontentswitch}
            %
            \docOptionsItem{revealhiddenswitch}
            %
            \isPortfolio{}{\item[\wrtxListLabelStyle{Contents version}]\ \version}
        \end{wrtxListMeta}
    \end{minipage}


    \isPortfolio{}% do not include if portfolio
    {
        %%%%%%%%%%%%%%%%%%%%%%%%%%%%%%%%%%%%%%%%%%%%%%%%%%%%%%%%%%%%%%%%%%%%%%%%%%%
        %%% INCLUDED SECTIONS
        \subsubsection*{Components included}
        %
        \begin{wrtxListMeta}
            %%% START WITH EMPTY ITEM
            \hiddenitem % EMPTY ITEM NECESSARY TO AVOID ERROR
            \foreach \file in \filesToAdd {
                \item             \expandafter\detokenize\expandafter{\file}%
            }%
        \end{wrtxListMeta}%
        \ifthenelse{\equal{\filesToAdd}{}}{\textcolor{wrtxColorDanger}{No components included}}{}%
        %%%%%%%%%%%%%%%%%%%%%%%%%%%%%%%%%%%%%%%%%%%%%%%%%%%%%%%%%%%%%%%%%%%%%%%%%%%
        %%% PORTFOLIO APPENDIX ITEMS
        \subsubsection*{Appendix entries\ %
        \ifthenelse{\boolean{isIncludeAppendix}}{}{\textsuperscript{\textcolor{wrtxColorDanger}{\tiny APPENDIX EXCLUDED}}}
        }
        \ifthenelse{\not\equal{\appendixList}{}}% if appendix is empty, return false}
        {%
            \begin{wrtxListMeta}
                \foreach \entries [count=\n] in \appendixList {
                    \item[\wrtxListLabelStyle{\makeAlph{\n}}] %
                    \hyperlink{appendix:target:\n}{\expandafter\detokenize\expandafter{\entries}}%
                }
            \end{wrtxListMeta}
        }
        {%
            \hspace{1cm}\textcolor{wrtxColorDanger}{No appendix entries selected}
        }
    }
}


\newcommand{\addMetadata}{
    \ifthenelse{\boolean{isIncludeMeta}}%
    {%
    \updateRibbons{Article: \textbf{\wrtxCiteEntry{\wrtxarticleKey}{title}}\ribbonSpacer Meta-information}{META-INFORMATION}
        \metaInfoSettings{
            {
                \wrtxSectionTitle{Meta-info}

                %%%%%%
                %%%%%% TECHNICAL:
                \isPortfolio{}{\technicalMetadata}
                %%%%%%
                %%%%%% DOCUMENT OPTIONS:
                \docOptionsMetadata
                %%%%%%
                %%%%%% MAIN SOURCE INFO:
                \mainSourceMetadata
                %%%%%%
                %%%%%% MEDIATION PAPER:
                \wrtxArticleMetadata
                %%%%%%
                %%%%%% COUNTERS:
                \countersList
            }
        }%
    }%
}
  \NewDocumentEnvironment{substanceEnvironment}
{
    O{substance}
}
{
    \IfSubStr{#1}{substance}{
        % env beginning code
        % Change the margins
        %%% SUBSTANCE INFORMATION SECTION
        %%%%%% SUBSTANCE PAGES PARAMETERS
        \onehalfspacing %spacing between lines
        \parskip=0em % spacing between paragraphs (0pt plus 1pt default)
        \parindent=0pt % indent at start of each paragraph (15 default)
        % Font size in substance section
        \small
        %%%%%%
        %%% Subsubsection
        \titlespacing*{\subsubsection}{0pt}{0pt}{5pt}
        \titleformat{\subsubsection}
        {\vbox{\rule{1\textwidth}{0.1pt}\vspace{-0.5ex}}\small\color{myGrayMed}}
        {\thesubsubsection}
        {0.1em}
        {\hspace{2em}} % change subsubsection style
        %%%%%% SUBSTANCE:
        \mySectionTitle{Substance}
    }
    {%
        % No substance included
    }%
}
{
    % env ending code
}

\BeforeBeginEnvironment{substanceEnvironment}{
    \newgeometry{top=0.3cm,bottom=0.3cm,left=0.5cm,right=0.5cm}
}
\AfterEndEnvironment{substanceEnvironment}{\restoregeometry}


%%% To optionally add substance to each article in the portfolio
\newcommand{\addSubstanceToPortfolio}[1]
{%
    \ifthenelse{\boolean{isIncludePerArticleSubstance}}%
    {%
        \begin{substanceEnvironment}%
            #1%
        \end{substanceEnvironment}%
    }%
    {}
}%


\BeforeBeginEnvironment{substanceEnvironmentPORTFOLIO}{
    \newgeometry{top=0.3cm,bottom=0.3cm,left=0.5cm,right=0.5cm}
}
\AfterEndEnvironment{substanceEnvironmentPORTFOLIO}{\restoregeometry}




% Substance list
% EXPANDED LIST WITH LINE BREAKS BETWEEN ITEMS
\newlist{substanceList}{enumerate}{1}
\setlist[substanceList]{
    label={},
    align=right, % label alignment
    % labelwidth=0.25cm,
    topsep=0pt, % reduce space above list
    itemsep=0pt, % Sets the space between items
    parsep=0pt, % Sets the space between paragraphs within the items
    leftmargin=0pt, % * or 0pt
    labelsep=0pt, % Space between the bullet and the item text
}
  %%% Article cover page


%%%%%%% DESIGN OF TITLE PAGE
\newcommand{\wrtxTitlePage}{%
    \begin{titlepage}
        \atBeginTitlePage% Important set important functionalities
        \pagecolor{wrtxGrayLight}
        \centering
        % \wrtxVfills{2}
        %
        \begin{wrtxTitlesBox}
            %
            %%%%%%%%%%%%%%%%%%%%%%%% TITLE
            {%
                \color{wrtxColorSecondary}% Color
                \setTitleFont% Font family
                \HUGE% Font size
                \bfseries% Bold
                \scshape% Small caps
                \lsstyle% letter spacing
                % \itshape% Italics
                \wrtxCiteEntry{\wrtxarticleKey}{title}%
            }%
            % \tcblower%%% adds separator in colorbox between upper and lower half
            %
            %%%%%%%%%%%%%%%%%%%%%%%% SUBTITLE
            {%
                \color{wrtxGrayDark}\large \wrtxCiteEntry{\wrtxarticleKey}{subtitle}
                [\\\vspace{0.35cm}]
                [\vspace{0.35cm}]
            }%
        \end{wrtxTitlesBox}
        %
        \titlePageItem
        {0}% space units before
        [\horizontalDeco{Written by:}]% Prenote
        {author}
        [\\]% Postnote
        {0}% space units above
        %
        \titlePageItem
        {0}% space units before
        [\horizontalDeco{Illustrated by:}]% Prenote
        {illustrator}
        [\\]% Postnote
        {0}% space units above
        %
        \titlePageItem
        {0}% space units before
        [\horizontalDeco{Reviewed by:}]% Prenote
        {reviewer}
        [\\]% Postnote
        {0}% space units above
        %
        \titlePageItem
        {0}% space units before
        [\horizontalDeco{Translated by:}]% Prenote
        {translator}
        [\\]% Postnote
        {0}% space units above
        %
        % Institution
        % {\normalsize Institution Name}
        % \wrtxVfills{1}
        %
        % Optional: cover image
        \addCoverImg{\wrtxMainImg}
        \wrtxVfills{1}
        % Optional: logo
        % \includegraphics[width=0.2\textwidth]{icons/placeholder.png}
        %
        % Date
        {\color{wrtxGrayDark}\normalsize \today}
        \wrtxVfills{1}
        %
    \end{titlepage}
}



\newcounter{titlepagenumber}% Store current page to recover it after titlepage (which would reset it)

\newcommand{\addCover}{%
    \ifthenelse{\boolean{isIncludeArticleCover}}%
    {%
        \updateRibbons{Article: \textbf{\wrtxCiteEntry{\wrtxarticleKey}{title}}\ribbonSpacer Cover}{COVER}
        \setcounter{titlepagenumber}{\value{page}}% recover page number
        % Create the cover page
        \begin{samepage}
            \wrtxTitlePage%
        \end{samepage}
        % revert bg color
        \isDraft{%
            \pagecolor{draftBgColor}%
        }%
        {%
            \pagecolor{bgColor}%
        }%
    }
    {%
    % else clause
    }%
}

\NewDocumentCommand{\addCoverImg}
{
    m% image filename
    O{0.5\textwidth}% image size
}{%
    \isDraftDebugger{Cover image:\ #1\\}{}%
    \ifthenelse{\equal{#1}{}}%
    {}%
    {%
    \begin{tcolorbox}
        [
        center,
        colframe=wrtxColorPrimary,
        % colback=orange!50,
        boxsep=0px,
        left=0pt,right=0pt,top=0pt,bottom=0pt,
        boxrule=3px,
        arc=0px,
        outer arc=0px,
        hbox
        ]
         \includegraphics[width=#2]{#1}%\\%
    \end{tcolorbox}%
    }%
}




\newcommand{\atBeginTitlePage}
% if i use atBeginEnv{titlepage}, i get an undesired page break
{%
    \setcounter{page}{\thetitlepagenumber}%
    %%% reference to cover
    \isPortfolio{}{\hypertarget{start}{}}% generic reference to cover, used by ribbon link
    \newcommand{\articleCoverRef}{\wrtxarticleKeyCore:cover}%
    \isPortfolio{}{\hypertarget{\articleCoverRef}{}} % reference to this specific cover
    \isDraftDebugger{specific reference: \articleCoverRef; generic reference: start}{}%
    \phantomsection % ensures linking with hyperref to exact page
    \hypertarget{\wrtxarticleKeyCore:cover}{}%
    \label{\wrtxarticleKeyCore:cover} % Add label here
    % ADD TO TOC WITHOUT NUMBER PAGE
    \isPortfolio
    {%
        \addcontentsline{toc}{section}{%
            \noindent%
            \protect\hyperref[\wrtxarticleKeyCore:cover]{\textbf{\TEXTcover}}%
        }%
    }
    {%
        \addtocontents{toc}{%
            \protect\wrtxContentsTextFont% change font
            \noindent%
            \hyperref[\wrtxarticleKeyCore:cover]{\textbf{\TEXTcover}}%
            \par% this paragraph can cause issues, use \par, not \\
        }%
    }%
}


% Macro add vertical space in proportion
\NewDocumentCommand{\wrtxVfills}
{%
    m% how many
}
{%
    % \par%
    \ifthenelse{#1>0}%
    {%
        \foreach \n in {1,...,#1}{%
            \vfill%
            \isDraftDebugger%
            {%
                \textcolor{wrtxColorWarning}{\tiny\n}%
                \vfill%
            }{}%
        }%
    }%
    {%
        \isDraftDebugger
        {\textcolor{wrtxColorDanger}{\tiny DO NOT ADD SPACE}}%
        {}%
    }% if 0, do add anything
}


\NewDocumentCommand{\titlePageItem}
{%
    m% space added at the beginning
    O{}% Optional prenote
    m% cite key to desired field
    O{}% Optional postnote
    m% space added at the end
}
{%
    {%
        % Styling entry
        \normalsize\color{wrtxGrayDark}%
        %
        \wrtxCiteEntry{\wrtxarticleKey}{#3}%
        %%% Prenote
        [{%
            \wrtxVfills{#1}% SPACE
            \normalsize\color{wrtxGrayMed}% STYLE
            #2% CONTENT
        }]%
        %%% Postnote
        [{%
            \normalsize\color{wrtxGrayMed}% STYLE
            #4% CONTENT
            \wrtxVfills{#5}% SPACE
        }]%%%adding space if present
    }%
}

\newtcolorbox{wrtxTitlesBox}
{
    % draft,% to see measurements
    enhanced,
    center,
    halign=center,
    % halign lower=center,
    valign upper=center,
    % valign lower=center,
    % lower separated=false,% make separation invisible
    width={1\pagewidth},
    % text width=0.8\linewidth,
    height=5cm,
    boxsep=5mm,
    boxrule=0.1mm,
    leftrule=0.25mm,
    rightrule=0.25mm,
    arc is angular,% box shape
    arc=1mm,
    outer arc=0mm,
    colback=wrtxColorPrimary!3!white,colframe=wrtxColorPrimary!95!black,
    % TITLE OF TEXTBOX
    title={Mediation Article},
    fonttitle=\Large,
    coltitle=white,
    % colbacktitle=red,
    toptitle=3mm, bottomtitle=3mm,
    halign title=flush center,
    attach boxed title to top center={
        xshift=0cm,
        yshift= -3.5mm, % What do I put here? I'd like to have something like:
%       yshift= -0.5\titleboxheight
    }
}


\NewDocumentCommand{\horizontalDeco}
{m}{%
    \tikzset{
    wrtxline/.style={
        line width=0.1ex,
        line cap=round,
        wrtxColorPrimary
    }
    }
    \noindent\tikz{%
        %%% CONTENT
        \path (0,0) -- node[inner xsep=1em] (content) {#1} ++ (\linewidth,0);
        % LEFT LINE
        \draw[wrtxline]  (0,0) -- (content);
        % RIGHT LINE
        \draw[wrtxline]  (content) -- (\linewidth,0);
    }%
}
  
\usepackage{titletoc} % to create partial tables of contents.

\newcommand{\myContentsTextFont}{\setTitleFont}

%Add dots for Sections in TOC
\usepackage{tocloft}
\renewcommand{\cftsecdotsep}{\cftdotsep}
\setcounter{secnumdepth}{0} % to remove numbering from all (sub)sections while keeping it in the ToC
\renewcommand{\cftsecindent}{0em}
\renewcommand{\cftsecnumwidth}{2.4em}
\renewcommand{\cftsubsecindent}{2.4em} %subsec indent
\renewcommand{\cftsubsecnumwidth}{3.0em}
\renewcommand{\cftsubsubsecindent}{4.7em}
\renewcommand{\cftsubsubsecnumwidth}{4.7cm}
% TOC text style
\renewcommand{\cftpartfont}{\LARGE\bfseries\color{black}\myContentsTextFont}
\isPortfolio{\renewcommand{\cftchapfont}{\large\color{black}\myContentsTextFont\bfseries}}{}
\renewcommand{\cftsecfont}{\color{black}\myContentsTextFont}
\renewcommand{\cftsubsecfont}{\color{myGrayDark}\myContentsTextFont}
\renewcommand{\cftsubsubsecfont}{\color{myGrayDark!95}\myContentsTextFont}
\renewcommand{\cftparafont}{\color{myGrayDark!90}\myContentsTextFont}
\renewcommand{\cftsubparafont}{\color{myGrayDark!85}\myContentsTextFont}
% TOC numbering style
% \renewcommand{\cftpartpagefont}{\setMainFont}
\isPortfolio{\renewcommand{\cftchappagefont}{\setMainFont}}{}
\renewcommand{\cftsecpagefont}{\setMainFont}
\renewcommand{\cftsubsecpagefont}{\setMainFont}
\renewcommand{\cftsubsubsecpagefont}{\setMainFont}
\renewcommand{\cftparapagefont}{\setMainFont}
\renewcommand{\cftsubparapagefont}{\setMainFont}
% LOF text style
\renewcommand{\cftfigfont}{\color{black}\myContentsTextFont}
% LOF numbering style
\renewcommand{\cftfigpagefont}{\setMainFont}
% LOT text style
\renewcommand{\cfttabfont}{\color{black}\myContentsTextFont}
% LOT numbering style
\renewcommand{\cfttabpagefont}{\setMainFont}
% LOTB text style
% \renewcommand{\cfttextboxfont}{\color{red}\myContentsTextFont}% not working as expected, changes done direcly in addcontentsline command
% LOTB numbering style
% \renewcommand{\cfttextboxpagefont}{\setMainFont}% not working as expected, changes done direcly in addcontentsline command


\setlength{\cftfigindent}{0pt}  % remove indentation from figures in lof
\setlength{\cfttabindent}{0pt}  % remove indentation from tables in lot
%%% Write something below list titles
\newcommand{\listFirstLine}{%
% \hfill\null\\%
\null\hfill\textmd{%
    \color{myGrayMed}{Page}%
    }%
}
%%% TOC
\renewcommand\cftaftertoctitle{\listFirstLine}
%%% LOF
\renewcommand\cftafterloftitle{\listFirstLine}
%%% LOT
\renewcommand\cftafterlottitle{\listFirstLine}
%%% LOTB: list of textboxes; use alternative solution
\AfterPreamble{%
    %how to use:  \cftaddtitleline{hfilei}{hkindi}{htitlei}{hpagei}
    \cftaddtitleline{loTB}{textbox*}{%
    \listFirstLine%
    \\% add extra space to compensate
    }{}%
}



%Page style for TOC
% \tocloftpagestyle{empty} % MAY BE OVERWRITTEN


\newcommand{\addToCLoFLoT}{
    \newpage
    \isPortfolio{}{\begin{SplitColumnsInTwo}}% do not apply to portfolio
        \updateRibbons{\textbf{TOC LOF LOT}}{}
        \hypertarget{contents}{}
        % \section*{Contents}
        \addToC%
        \addLoF%
        \addLoT%
        \addLoTextBoxes%
    \isPortfolio{}{\end{SplitColumnsInTwo}}
    %%%
}




\newcommand{\addToC}{
    \ifthenelse{\boolean{isIncludeToC}}{
        %%% TABLE OF CONTENTS (TOC)
        \hypertarget{toc}{}
        \renewcommand{\contentsname}{\vspace*{-40pt}} % remove title of ToC
        \section*{\listTitleStyle\TEXTtoc}
        % change TOC depth
        \setcounter{tocdepth}{5}
        \begingroup % start a TeX group
            % these apply to all, they more targeted  changes are done elsewhere with the \cft commands
            % \myContentsTextFont
            % \color{myGrayDark}% or whatever color you wish to use
            \tableofcontents%
        \endgroup   % end of TeX group
    }{}
}



\newcommand{\addLoF}{
    \isPortfolio{%
        \newcommand{\myFigCounter}{1} %%% Print in ANY case since 1<>0
    }
    {
        \newcommand{\myFigCounter}{%
            \totvalue{totalFiguresInArticle:\myarticleKeyCore}}
    }
    \ifthenelse{\myFigCounter=0}{ %%% Include only if there are figures
        % No figures to display.
    }{
        \ifthenelse{\boolean{isIncludeLoF}}{
            %%% LIST OF FIGURES (LoF)
            \hypertarget{lof}{}
            \setcounter{lofdepth}{2} % we want subfigures in the list of figures

            \renewcommand{\listfigurename}{\vspace*{-40pt}} % remove title of LoF
            \section*{\listTitleStyle\TEXTlof}
            % change TOC depth
            % \setcounter{tocdepth}{2}
            \listoffigures
            % Total number of figures in this document: \total{myFigCounter}
        }{}
    }

}



\newcommand{\addLoT}{
    \isPortfolio{%
        \newcommand{\myTabCounter}{1} %%% Print in ANY case since 1<>0
    }
    {
        \newcommand{\myTabCounter}{%
            \totvalue{totalTablesInArticle:\myarticleKeyCore}}
    }
    \ifthenelse{\myTabCounter=0}{ %%% Include only if there are tables
        % No tables to display.
    }{
    \ifthenelse{\boolean{isIncludeLoT}}{
        %%% LIST OF TABLES (LoT)
        \hypertarget{lot}{}
        \setcounter{lotdepth}{1}
        \renewcommand{\listtablename}{\vspace*{-40pt}} % remove title of LoT
        \section*{\listTitleStyle\TEXTlot}
        \listoftables
        % Total number of figures in this document: \total{totalTablesInArticle:\myarticleKeyCore}
    }{}
    }
}


\newcommand{\addLoTextBoxes}{
    \isPortfolio{%
        \newcommand{\myTextboxCounter}{1} %%% Print in ANY case since 1<>0
    }
    {
        \newcommand{\myTextboxCounter}{%
            \totvalue{totalTextboxesInArticle:\myarticleKeyCore}}
    }
    \ifthenelse{\myTextboxCounter=0}{ %%% Include only if there are textboxes
        % No tables to display.
    }{
    \ifthenelse{\boolean{isIncludeLoTB}}{
        %%% custom LIST OF Textboxes (LoTextBoxes)
        \hypertarget{loTB}{}
        % \setcounter{secnumdepth}{0}
        \renewcommand{\listtextboxname}{\vspace*{-40pt}} % remove title of LoT
        \section*{\listTitleStyle\TEXTlotb}
        \begingroup % start a TeX group
        % these apply to all elements so targeted adjustments should be done elsewhere
            %\myContentsTextFont
            % \color{red}% or whatever color you wish to use
            \listoftextbox%
        \endgroup   % end of TeX group
        % Total number of figures in this document: \total{totalTablesInArticle:\myarticleKeyCore}
    }{}
    }
}

% styling toc/lof/loc/lotb titles
\newcommand{\listTitleStyle}{%
    \color{myColorPrimary}\myContentsTextFont
}

  \ifthenelse{\boolean{isPrintVersion}}%
{% IF YES
    % PRINTABLE VERSION
    \setboolean{@twoside}{true}
    \ifthenelse{\equal{\@documentclass}{book}}{\setboolean{@openright}{true}}{} %%% only applies to book class
    \loadgeometry{print}
    \isPortfolio
    {%
        \tocloftpagestyle{plain}%
        \pagestyle{plain}%
    }
    {%
        \tocloftpagestyle{stylePrint-NoPage}%
        \pagestyle{stylePrint-NoPage}
    }
}
{% ELSE
    %DIGITAL VERSION
    \setboolean{@twoside}{false} %% should it be two side still, so that fancy footer placements work ???
    \ifthenelse{\equal{\@documentclass}{book}}{\setboolean{@openright}{false}}{} %%% only applies to book class
    \loadgeometry{digital}
    \pagestyle{styleDigital-NoPage}
    \isPortfolio
    {%
        \tocloftpagestyle{plain}%
        \pagestyle{plain}%
    }
    {%
        \tocloftpagestyle{styleDigital-NoPage}%
        \pagestyle{styleDigital-NoPage}
    }
}



\newcommand{\pageStyleBody}{%
    \isPrint%
    {
        \pagestyle{stylePrint}
    }
    {
        \pagestyle{styleDigital}
    }%
}



  

% \setcounter{myFootnoteCounter}{0} % change counter value to a starting specific value


\newcommand{\myFN}[1]{%
    \ifthenelse{\boolean{isIncludeFootnotes}}{%
        \stepcounter{totalFootnotesInArticle:\myarticleKeyCore}% increment counter
        \stepcounter{totalFootnotesAltogether}%
        \footnote{#1}%
    }{}%
}%

  % Package for splitting pages into columns.
% Multicol is great to split within a page into 2 or more columns, but it is does not allow floats (unless using things like figure*)
% \usepackage{multicol}


% Environment to conditionally split into two columns using the base twocolumn functionality (cannot do in-page differences)
\NewDocumentEnvironment{SplitColumnsInTwo}{
    O{isSplitInTwoColumns}% can be overridden optionally
}
{%
    \ifthenelse{\boolean{#1}}
    {\twocolumn}
    {}%
}
{%
    \ifthenelse{\boolean{#1}}
    {\onecolumn}
    {}%
}
% Space between columnx
\setlength{\columnsep}{1.4cm}

% Vertical line between columns
\setlength{\columnseprule}{0.4pt}

% Change color of vertical line
\newcommand{\latexcolumnseprulecolor}{\color{wrtxGrayMed}}
%
\makeatletter
\patchcmd\@outputdblcol{% find
  \normalcolor\vrule
}{% and replace by
  \latexcolumnseprulecolor\vrule
}{% success
}{% failure
  \@latex@warning{Patching \string\@outputdblcol\space failed}%
}
\makeatother


  \newcommand{\wrtxArticleTitleLinker}{}

\newcommand{\tocTitleBgColor}{}
\isDraft{\renewcommand{\tocTitleBgColor}{draftBgColor}}{\renewcommand{\tocTitleBgColor}{bgColor}}%
\newcommand{\setArticleTitle}[1]{%
  \renewcommand{\wrtxArticleTitleLinker}{%
    #1:articleHeader%
  }
  \phantomsection % ensures linking with hyperref to exact page
  \sectionmark{#1}
  \hypertarget{#1}{}
  %%% TOC ENTRY MANUALLY
  \isPortfolio
  {
    % CONTROL SPACE ABOVE LINE
    \addtocontents{toc}{\protect\vspace*{1ex}}
    % ADD TARGET TO BE ABLE TO REACH IT
    \addtocontents{toc}{
      \protect\hypertarget%
      {\wrtxArticleTitleLinker}{}%
        % \par% "par" only needed in single articles for some reason
    }%
  }%
  { %
    % CONTROL SPACE ABOVE LINE
    \addtocontents{toc}{\protect\vspace*{0ex}}
    % ADD TARGET TO BE ABLE TO REACH IT
    \addtocontents{toc}{
      \protect\hypertarget%
      {\wrtxArticleTitleLinker}{}%
        \par% "par must be included, so it a newline is create. Use vspacer above to compensate.
  }%
  }
  % ADD LINE
  \addcontentsline{toc}{section}{%
    {%
      % \colorbox{\tocTitleBgColor}{%
        \noindent%
        % \parbox[][1cm]% box height
        %   [c]% c: centered, t: top, b: bottom
        %   {0.90\linewidth}{% Not
          % \large% Uncomment this to change font size
          \protect\hyperref[\wrtxArticleTitleLinker]{%
            \textbf{Article body:} \wrtxCiteEntry{#1}{title}% ToC entry
          }
        % }
      % }
    }%
  }
  %
  %
  %
  %
  {%
    %%% SECTION TITLE
    \section*{%
      \hyperlink{\wrtxArticleTitleLinker}
      {\Huge\setTitleFont\color{wrtxGrayDark}\wrtxCiteEntry{#1}{title}%
      }% Document entry
    }%
    \label{\wrtxArticleTitleLinker}%
    \isDraftDebugger{%
      \begin{center}%
        Title ref: \wrtxArticleTitleLinker%
      \end{center}}%
      {
        \vspace{-5ex}% remove empty space if needed
      }
  }%
}

\newcommand{\setArticleSubtitle}[1]
{%
      %%% SECTION subTITLE
      {%
      \begin{center}%
        \vspace{-2mm}%
        \setTitleFont%
        \color{wrtxGrayDark}%
        \large%
        \wrtxCiteEntry{#1}%
        {subtitle}%
        \vspace{-4mm}%
      \end{center}%
      }%
}

\newenvironment{bodyEnvironment}
{
    %%%%%%%%%%%%%%%%%%%%%%%%%%%%%%%%%%%
    %%% Article subsection style
    \titlespacing*{\subsection}{0pt}{30pt}{5pt}
    \newfontfamily\articlesectionfont[Color=wrtxGrayMed]{\wrtxTitleFont}
    \titleformat{\subsection}
    {%
    \wrtxFloatBarrier
    \Large\articlesectionfont%
    }
    {\thesubsection}
    {0.1em}
    {}
    %%%%%%%%%%%%%%%%%%%%%%%%%%%%%%%%%%%
    %%% Article subsubsection style
    \titlespacing*{\subsubsection}{0pt}{15pt}{3pt}
    \newfontfamily\articlesubsectionfont[Color=wrtxGrayMed]{\wrtxTitleFont}
    \titleformat{\subsubsection}
    {%
    % \wrtxFloatBarrier
    \large\articlesubsectionfont%
    }
    {\thesubsubsection}
    {0.1em}
    {}
    %%%%%%%%%%%%%%%%%%%%%%%%%%%%%%%%%%%
    %%% Article paragraph style (used as "sub"-subsubsection)
    \titlespacing*{\paragraph}{0pt}{10pt}{0pt}
    \newfontfamily\articleparagraphfont[Color=wrtxGrayMed]{\wrtxTitleFont}
    \titleformat{\paragraph}
    {%
    % \wrtxFloatBarrier
    \normalsize\articleparagraphfont%
    }
    {\theparagraph}
    {0.1em}
    {} % Adds the line break

    %begin code
    \isPrint
    {\newpage\pagestyle{empty}\cleardoublepage}
    {\newpage}
    \draftVersionOnly{
      \linenumbers %activate to add lines
    }
    \isPortfolio
    {
      % IF PORTFOLIO, DO THIS:
      % THIS IS THE PORTFOLIO
    }
    {
      % IF SINGLE ARTICLE, DO THIS:
      % Reset page counter
      \setcounter{page}{1}
    }
    \pageStyleBody %%% RUN LATE TO ENSURE IT IS EFFECTIVE
      % \pagestyle{plain}
    \begin{SplitColumnsInTwo}% conditionally split in two
    %%%%%% TITLE
    \setArticleTitle{\wrtxarticleKey}
    %%%%%% SUBTITLE
    \setArticleSubtitle{\wrtxarticleKey}%
    %%%%%% ILLUSTRATIVE IMAGE
    \printTitleImg
    %%%%%% DECLARE MAIN SOURCE
    \printMainSource
    % \begin{multicols}{2}% to split article body into columns
}
{% ========= END CODE
    % \end{multicols}
    %%%%%% CREDITS
    \printCredits
    \end{SplitColumnsInTwo}% conditionally split in two
    %%%%%% MOVED TEXTBOXES
    \postponeTextBoxPrintTillHere
    %%%% Store number of last article page
    % \totalArticles

    \setcounter{lastPageInArticle:\wrtxarticleKeyCore}{\thepage}
    \newpage
}




% Article section references
\newcommand{\articleSectionRef}{\wrtxarticleKey:secAnchor%
\the\value{totalSectionsInArticle:\wrtxarticleKeyCore}%
}
\newcommand{\articleSubsectionRef}{\wrtxarticleKey:subSecAnchor%
\the\value{totalSubsectionsInArticle:\wrtxarticleKeyCore}%
}
\newcommand{\articleSubsubsectionRef}{\wrtxarticleKey:subsubSecAnchor%
\the\value{totalSubsubsectionsInArticle:\wrtxarticleKeyCore}%
}

\ifthenelse{\boolean{isIncludeToC}}
{ % if TOC included, link to it
  %%%%%%%%%%%%%%%%%%%%%%%%%%%%%%%%%%%%%%%%%%%%
  \newcommand{\linkSectionConditionally}[1]
  {
    \protect\hyperlink
    {\articleSectionRef}
    {#1}
    \isDraftDebugger{\ Section key: \articleSectionRef}{}
  }
  %%%%%%%%%%%%%%%%%%%%%%%%%%%%%%%%%%%%%%%%%%%%
  \newcommand{\linkSubsectionConditionally}[1]
  {
    \protect\hyperlink
    {\articleSubsectionRef}
    {#1}
    \isDraftDebugger{\ Subsection key: \articleSubsectionRef}{}
  }
  %%%%%%%%%%%%%%%%%%%%%%%%%%%%%%%%%%%%%%%%%%%%
  \newcommand{\linkSubsubsectionConditionally}[1]
  {
    \protect\hyperlink
    {\articleSubsubsectionRef}
    {#1}
    \isDraftDebugger{\ Subsubsection key: \articleSubsubsectionRef}{}
  }
}
{ % if TOC not included, link to title instead
  \newcommand{\linkSectionConditionally}[1]
  {
    #1
    \isDraftDebugger{\ Section key: LINK IS INACTIVE}{}
  }
  \newcommand{\linkSubsectionConditionally}[1]
  {
    #1
    \isDraftDebugger{\ Subsection key: LINK IS INACTIVE}{}
  }
  \newcommand{\linkSubsubsectionConditionally}[1]
  {
    #1
    \isDraftDebugger{\ Subsubsection key: LINK IS INACTIVE}{}
  }
}

%%%%%%%%%%%%%%%%%%%%%%%%%%%%%%%%%%%%%%%%%%%%%%%%%%%%%%%%%%%%%%%%%%
%%%%% SECTION
\NewDocumentCommand{\secLabel}
{m}{%
  sec:\detokenize{#1}% detokenize to treat special characters as plain text
}
\NewDocumentCommand{\wrtxArticleSection}
{
  m %
  O{} % optional alternative label for ease of referencing
}
{%
  \stepcounter{totalSectionsInArticle:\wrtxarticleKeyCore} % increase counter
  \refstepcounter{totalArticleSectionsAltogether} % increase counter
  \addtocontents{toc}{
    \protect\hypertarget
    {\articleSectionRef}
    {}% text to add to TOC (leave empty)
    }
  \subsection[#1]{% style defined elsewhere with \titleformat. Section are actually subsections!
    \linkSectionConditionally
    {#1}% (Sub)Section title
  }%
  \ifthenelse{\equal{#2}{}}{}{\label{\secLabel{#2}}}%
  \wrtxLogger{START SECTION NUMBER \the\value{totalSectionsInArticle:\wrtxarticleKeyCore} \wrtxarticleKeyCore}
}

%%%%%%%%%%%%%%%%%%%%%%%%%%%%%%%%%%%%%%%%%%%%%%%%%%%%%%%%%%%%%%%%%%
%%%%% SUBSECTION
\NewDocumentCommand{\subsecLabel}
{m}{%
  subsec:\detokenize{#1}% detokenize to treat special characters as plain text
}
\NewDocumentCommand{\wrtxArticleSubsection}
{
  m %
  O{} % optional alternative label for ease of referencing
}
{%
  \stepcounter{totalSubsectionsInArticle:\wrtxarticleKeyCore} % increase counter
  \refstepcounter{totalArticleSubsectionsAltogether} % increase counter
  \addtocontents{toc}{
    \protect\hypertarget
    {\articleSubsectionRef}
    {}% text to add to TOC (leave empty)
    }
  \subsubsection[#1]{% style defined elsewhere with \titleformat. Subsection are actually subsubsections!
      \linkSubsectionConditionally{#1}
    }%
  \ifthenelse{\equal{#2}{}}{}{\label{\subsecLabel{#2}}}%
  \wrtxLogger{START SUBSECTION NUMBER \the\value{totalSubsectionsInArticle:\wrtxarticleKeyCore} \wrtxarticleKeyCore}
}

%%%%%%%%%%%%%%%%%%%%%%%%%%%%%%%%%%%%%%%%%%%%%%%%%%%%%%%%%%%%%%%%%%
%%%%% SUBSUBSECTION
\NewDocumentCommand{\subsubsecLabel}
{m}{%
  subsubsec:\detokenize{#1}% detokenize to treat special characters as plain text
}
\NewDocumentCommand{\wrtxArticleSubsubsection}
{
  m %
  O{} % optional alternative label for ease of referencing
}
{%
  \stepcounter{totalSubsubsectionsInArticle:\wrtxarticleKeyCore} % increase counter
  \refstepcounter{totalArticleSubsubsectionsAltogether} % increase counter
  \addtocontents{toc}{
    \protect\hypertarget
    {\articleSubsubsectionRef}
    {}% text to add to TOC (leave empty)
    }
  \paragraph[#1]{% style defined elsewhere with \titleformat. Subsection are actually subsubsections!
      \linkSubsubsectionConditionally{#1}
    }%
  \ifthenelse{\equal{#2}{}}{}{\label{\subsubsecLabel{#2}}}%
  \wrtxLogger{START SUBSUBSECTION NUMBER \the\value{totalSubsubsectionsInArticle:\wrtxarticleKeyCore} \wrtxarticleKeyCore}
}





\newcommand{\refDraftMessage}[1]{%
  \ifthenelse{\boolean{isDraft}}{%
    \color{wrtxColorWarning}%
    {\texttt{(#1)}}%
  }%
  {%
    \color{wrtxGrayMed}%
  }%
}

%%% Classic ref, which includes the label number/text
\NewDocumentCommand{\wrtxSecRef}
{%
    m % section reference prefix: sec or subsec or subsubsec
    m % reference key
    O{Section} % reference prefix
}{%
  \begingroup%
  \refDraftMessage{#1:#2}%
  #3\ \ref{#1:#2}% REFER TO IT BY NUMBER
  % #3\ \nameref{#1:#2}% REFER TO IT BY NAME (full text)
  \endgroup%
}


%%% Hyper ref, which links a piece of text
\NewDocumentCommand{\wrtxSecHyperref}
{%
    O{} % section reference prefix: "part" or "chapter" ir "sec" or "subsec" or "subsubsec"
    O{} % reference key
    m % reference prefix
}{%
  \begingroup%
  \refDraftMessage{#1:#2}%
  \hyperref[#1:#2]{#3}%
  \endgroup%
}


\newcommand{\printMainSource}{%
  \ifthenelse{\equal{\mainSourceKey}{}}% check if source is empty
  {%
    % empty main source
  }%
  {%
    \textcolor{wrtxGrayMed}{\textbf{\TEXTmainSource:}}%
    \wrtxCiteEntry{\mainSourceKey}{title}\wrtxCite{\mainSourceKey}%
  }%
}%


\newcommand{\printCredits}{%
  \ifthenelse{\boolean{isCreditsInArticleBody}}% check if source is empty
  {%
  \vspace{1cm}
  \begin{wrtxListMeta}
    [
      align=left,
      leftmargin=30pt,
      labelsep=5pt,
      itemsep=5pt,
    ]
    %%% START WITH EMPTY ITEM
    \item [\setTitleFont\color{wrtxGrayDark}\textbf{Contributions}]
    \hiddenitem % EMPTY ITEM NECESSARY TO AVOID ERROR
    \noindent\wrtxCiteEntryWithLabel{\wrtxarticleKey}{author}%
    \noindent\wrtxCiteEntryWithLabel{\wrtxarticleKey}{translator}%
    \noindent\wrtxCiteEntryWithLabel{\wrtxarticleKey}{illustrator}%
    \noindent\wrtxCiteEntryWithLabel{\wrtxarticleKey}{reviewer}%
    \noindent\wrtxCiteEntryWithLabel{\wrtxarticleKey}{thank}%
  \end{wrtxListMeta}
  }%
  {%
  }%
}%

\newcommand{\printTitleImg}{
  \ifthenelse{\boolean{isIncludeArticleCoverImgInBody}}{
  \begin{center}
    \includegraphics[width=0.75\textwidth]{\wrtxMainImg}%
    \isDraftDebugger{\\MAIN IMAGE}{}%
  \end{center}
  }
  {
    \isDraftDebugger{
      \begin{center}
    EXCLUDE MAIN IMAGE FROM HERE
    \end{center} }{}%
  }
}
  \usepackage{wrapfig}
\usepackage{subfloat} % for subfigures
\setcounter{lofdepth}{2} % we want subfigures in the list of figures

\usepackage{caption}
\usepackage[list=true,listformat=simple]{subcaption} % for subfigure captions
\captionsetup[figure]{font=scriptsize,labelfont=scriptsize}
\captionsetup[figure]{labelformat=empty} % remove label numbering from figure captions
\renewcommand{\figurename}{Fig.} % change label from Figure to Fig. (in text references)
\renewcommand{\subfigurename}{Subfig.} % change label from Subfigure to Subfig. (in text references)
\subcaptionsetup[figure]{labelformat=empty} % remove label numbering from figure subcaptions

\newcommand{\stepGeneralFigureCounter}{%%% Use this macro as the general counter to increment the counter whenever a figure is added
  \stepcounter{totalFiguresInArticle:\myarticleKeyCore}%
  \stepcounter{totalFiguresAltogether}%
}

\newcommand{\refcounterWithoutIncrementing}[1]
{%%% Macro to set a counter  as label reference without incrementing it
  \addtocounter{#1}{-1}\refstepcounter{#1}% subtract 1 then add 1
}

\makeatletter
\newcommand\myFigCaption{% my caption style, linking to the TOF
  \@dblarg\@myFigCaption}
  \newcommand\myFigCaptionLinker{image:\myarticleKeyCore:\the\value{totalFiguresInArticle:\myarticleKeyCore}}% key to identify individual figures
\def\@myFigCaption[#1]#2{%
  \caption[\protect\hypertarget{\myFigCaptionLinker}{#1}]%
      {%
        \ifthenelse{\boolean{isIncludeLoF}}
        {% as hyperlink
          \hyperlink{\myFigCaptionLinker}{#2}
        }%
        {% as simple text
          #2%
        }%
      }%
    }
\makeatother
%%%%

\NewDocumentCommand{\figLabel}
{%
  m % #1 label id
}{%
  fig:\detokenize{#1}% detokenize to treat special characters as plain text
}


\NewDocumentCommand{\myFigRef}
{
    m % reference key
    O{Figure} % reference prefix
}{%
  \begingroup%
  \ifthenelse{\boolean{isDraft}}{%
    \color{myColorWarning}%
    {\texttt{(\figLabel{#1})}}%
  }%
  {%
    \color{myGrayMed}%
  }
  #2\ \ref{\figLabel{#1}}%
  \endgroup%
}


%%%% THE GENERAL FIGURE ENVIRONMENT WHERE INDIVIDUAL FIGURES WILL BE PLACED
\NewDocumentEnvironment{myFigEnv}
{
  O{}% #1: h(ere approx.), b(ottom), t(op), p(age alone), H(ERE PRECISELY). Leaving it empty lets latex choose
  O{width=0.475\linewidth} % #2: default figure size
  m % #3: figure caption.
  m % #4: figure caption extra.
  O{} % #5: optional argument to add a custom label for the figure environment
}
{%%% START ENVIRONMENT
  \begin{figure}[#1]%
    \stepGeneralFigureCounter
    \stepcounter{totalFigureEnvsInArticle:\myarticleKeyCore}%
    \stepcounter{totalFigureEnvsAltogether}%
    \centering%
    \isDraftDebugger{\myBreakMessage{START FIGURE ENVIRONMENT}}{}%
}
{%%%% END ENVIRONMENT
    %%% Caption:
    \captionsetup{#2}% Caption settings
    \myFigCaption[#3]{%
    \textcolor{myGrayMed}{\textbf{Figure\ \the\value{totalFiguresInArticle:\myarticleKeyCore}:\ }}% Figure number label
    \textbf{#3.} #4% Figure caption
    }%
    \isDraftDebugger{\myBreakMessage{END FIGURE ENVIRONMENT}}{}%
    \label{fig:env:#5} %optional label for environment
  \end{figure}
}

%%%% THE GENERAL WRAPFIGURE ENVIRONMENT WHERE INDIVIDUAL WRAP FIGURES WILL BE PLACED
\NewDocumentEnvironment{myWrapFigEnv}
{
  O{10}% #1: How many lines to wrap
  O{R}% #2: r(ight), l(eft), i(nside edge), o(utside edge). If uppercase, figure is a float. Lowercase means exactly here.
  O{0.33\textwidth} % #3: default wrapfigure width. Here "width=" cannot be included in contrast to myFigEnv
  m % #4: figure caption.
  m % #5: figure caption extra.
  O{} % #6: optional argument to add a custom label for the figure environment
}
{%%% START ENVIRONMENT
  \wrapfigure[#1]{#2}{#3}%%% Start wrapfigure
    \stepGeneralFigureCounter%
    \stepcounter{totalWrapfigureEnvsInArticle:\myarticleKeyCore}%
    \stepcounter{totalWrapfigureEnvsAltogether}%
    \centering%
    \isDraftDebugger{%
    % START WRAPFIGURE ENVIRONMENT\\%
    %
    \tiny wrap #1 lines;\ %
    \tiny position: #2;\ %
    \tiny width:\detokenize{#3}\\%
    }{}%
}
{%%%% END ENVIRONMENT
    %%% Caption:
    \myFigCaption[#4]{%
    \textcolor{myGrayMed}{\textbf{Figure\ \the\value{totalFiguresInArticle:\myarticleKeyCore}:\ }}% Figure number label
    \textbf{#4.} #5% Figure caption
    }%
    % \isDraftDebugger{END WRAPFIGURE ENVIRONMENT\\}{}%
    \label{fig:env:#6} %optional label for environment
  \endwrapfigure%%% End wrapfigure
}


\renewcommand\thesubfigure{\alph{subfigure}}% customize type of subfigure (e.g. arabic, alpha) counter display
%%%% COMMAND FOR SUBFIGURES (using subfloat, which is more up to date than subfigure)
\NewDocumentCommand{\mySubfig}
{%
  O{width=1\linewidth}% automatically adjusts to width of figure
  m % #2: figure caption.
  m % #3: figure caption extra.
  m% #4: contents of subfloat.
  O{}% #5: Optional label (cannot set it through the contents for unkown reason)
}%
{%
  \stepcounter{totalSubfiguresInArticle:\myarticleKeyCore}%
  \stepcounter{totalSubfiguresAltogether}%
  \captionsetup[subfigure]{#1}%
  \subfloat%
  [\ifstrempty{#2}{\isDraftDebugger{Captionless subfigure}{}}{#2}]%%% LOT ENTRY (if empty, pass an empty space so that is is still included in LOT)
  [%%% CAPTION OF SUBFLOAT
    \textcolor{myGrayMed}{\textbf{%
      Subfigure\ % custom text before counter
      \the\numexpr\value{totalFiguresInArticle:\myarticleKeyCore}+0\relax% Figure number label (if needed, add 1 so it matches correctly before the actual Figure counter is incremented)
      \thesubfigure% subfigure counter
    \ifstrempty{#2}{}{:\ }% add colon if caption is not empty
    }}%
    \ifstrempty{#2}{}{% check if provided caption is empty
      \textbf{#2.} #3% Figure caption
      }%
  ]%
  {%%%   CONTENT OF SUBFLOAT
    #4%
    \label{fig:#5}% label, which cannot be set through the contents placed in the environment, for an unkown reason
  }%
}%



%%%% MACRO TO INCLUDE A RASTER IMAGE
\NewDocumentCommand{\myFigGraphics}
{
  O{width=1\linewidth} % #1: default figure size attribute. Can be overwritten
  O{%
    {./}%
    {./assets/figures/}% lone article
    {./../../articles_common_files/assets/}% lone article
    {../../../articles/\myarticleKeyCore/assets/figures/}%portfolio
    {./../../../articles_common_files/assets/}% portfolio
  } % #2: default filepath(s). Can be overwritten
  m % #3: file name
  O{jpg} % #4: default extension. Can be overwritten
}
{%
    \stepcounter{totalGraphicsInArticle:\myarticleKeyCore}%
    \stepcounter{totalGraphicsAltogether}%
    \refcounterWithoutIncrementing{totalFiguresAltogether}
    %%% Figure path: is it really needed ?
    \graphicspath{#2}%
    %%% Figure as TiKz to add overlays:
      \begin{tikzpicture}%
        % Image
        \node[
          anchor=center,
          % minimum width=5cm,
          % minimum height=5cm,
          %%% Add border frame around pictures
          % draw=myColorPrimary, % border color
          % line width=2mm, % border width
          %%% Different ways to add shadows
          % drop shadow={opacity=1, shadow scale=1, shadow xshift=.7ex, shadow yshift=-.7ex, fill=red, path fading={circle with fuzzy edge 15 percent}},
          % blur shadow={shadow blur steps=5, shadow scale=1.2, shadow xshift=0.1em,shadow yshift=#1}
          ] at (0,0) {%
          %ADD FIGURE
          % \detokenize{#2#3.#4}
          \includegraphics[#1]{#3.#4}%
          %
          };
        % Rectangle
        % \node [draw, thick, shape=rectangle, minimum width=1\linewidth, minimum height=1cm, anchor=center, fill=myColorPrimary, fill opacity=0.9] at (0,0) {rectangular overlay};
        % Data over image
        \isDraftDebugger{
          % Figure label
          \node[fill=black!90, fill opacity=0.75, anchor=south]
          at (current bounding box.south)
          {Ref key: \figLabel{#3}};
          % Image extension
          \node[fill=black!90, fill opacity=0.75, anchor=north east]
          at (current bounding box.north east)
          {.\detokenize{#4 }};
          % %Image width
          \node[fill=black!90, fill opacity=0.75, anchor=north west]
          at (current bounding box.north west)
          {\detokenize{#1}};
        }{}%
      \end{tikzpicture}%
      %%% Label:
      \label{\figLabel{#3}}% fig:filename
      % Figure label: \figLabel{#3}
      % Figure counter: \the\value{totalFiguresInArticle:\myarticleKeyCore}
}

%%%% MACRO TO INCLUDE A TIKZ IMAGE
\NewDocumentCommand{\myFigTikz}
{
  O{scale=1} % #1: default figure scale attribute (fraction). Can be overwritten, probably necessary to adjust manually on a per img basis.
  O{%
  {./}%
  {./assets/figures/}% lone article
  {./../../articles_common_files/assets/}% lone article
  {../../../articles/\myarticleKeyCore/assets/figures/}% portfolio
  {./../../../articles_common_files/assets/}% portfolio
  } % #2: default filepath(s). Can be overwritten
  m % #3: file name
}
{%
    \stepcounter{totalTikZPicturesInArticle:\myarticleKeyCore}%
    \stepcounter{totalTikZPicturesAltogether}%
    \refcounterWithoutIncrementing{totalFiguresAltogether}
    %%% Figure path:
    \graphicspath{#2}
    %%% TiKz settings
    \tikzset{
      background grid/.style={
          % thick,
          thin,
          draw=myGrayLight,
          step=.5cm
        },
      background rectangle/.style={
          % rounded corners,
          % double, % for double border
          ultra thick,
          draw=myGrayDark,
          fill=white,
          % top color=blue,
          % bottom color= pink
        }
    }
    %%% Figure as TiKz:
    \isMinimal
    {%%% Load stored PDF to be faster
      \isPortfolio
      {%
        \includegraphics[#1]{../../../articles/\myarticleKeyCore/assets/tikz/#3/auxiliary_files/standalone.pdf}%
      }
      {%
        \includegraphics[#1]{assets/tikz/#3/auxiliary_files/standalone.pdf}%
      }
    }
    {%%% Fresh compilation of Tikz
    \isDraft
    {%
      \begin{tikzpicture}
        [#1,% passed as optional argument
        show background rectangle,
        show background grid
        ]
    }
    {%
      \begin{tikzpicture}
        [#1,% passed as optional argument
        ]
    }

        \isPortfolio
        {% Relative address for portfolio
          \expandafter\input\expandafter{../../../articles/\myarticleKeyCore/assets/tikz/#3/content.tex}%
        }{% Relative address for article
        \expandafter\input\expandafter{assets/tikz/#3/content.tex}%
        }%
        % Place a node at the center using current bounding box
        \isDraftDebugger{%
          \node[fill=black!90, fill opacity=0.75, anchor=north] at (current bounding box.south) {fig key: \normalsize\detokenize{#3}};%
        }{};
      \end{tikzpicture}
    }
    %%% Label:
    \label{\figLabel{#3}}% fig:filename
    % Figure label: \figLabel{#3}
    % Figure counter: \the\value{totalFiguresInArticle:\myarticleKeyCore}
}

  \captionsetup[table]{font=scriptsize,labelfont=scriptsize}
\captionsetup[table]{labelformat=empty} % remove label numbering from figure captions
\renewcommand{\tablename}{Tab.} % change label from Figure to Fig.
\subcaptionsetup[table]{labelformat=empty} % remove label numbering from figure subcaptions


\makeatletter
\newcommand\wrtxTabCaption{% wrtx caption style, linking to the TOF
  \stepcounter{totalTablesInArticle:\wrtxarticleKeyCore}% increment counter and ref it%
  \stepcounter{totalTablesAltogether}%
  \@dblarg\@wrtxTabCaption}
  \newcommand\wrtxTabCaptionLinker{table:\wrtxarticleKeyCore:\the\value{totalTablesInArticle:\wrtxarticleKeyCore}}% key to identify individual tables
\def\@wrtxTabCaption[#1]#2{%
  \caption[\protect\hypertarget{\wrtxTabCaptionLinker}{#1}]%
    {
      \ifthenelse{\boolean{isIncludeLoT}}
      {% as hyperlink
        \hyperlink{\wrtxTabCaptionLinker}{#2}
      }%
      {% as simple text
        #2
      }%
    }
    }
\makeatother
%%%%


\newcommand{\tabLabel}[1]{%
  tab:\detokenize{#1}% detokenize to treat special characters as plain text
}

% wriTeX Table Environment
\NewDocumentEnvironment{wrtxArticleTableEnv}
{
m % #1: Table caption
m % #2: table caption extra
O{} % #3: table key
}
{
    \begin{table}
        \centering
        %%% Caption:
        \wrtxTabCaption[#1]{%
            \textcolor{wrtxGrayMed}{\textbf{Table\ \the\value{totalTablesInArticle:\wrtxarticleKeyCore}:\ }}% Table number label
            \textbf{#1.} #2% Table caption
            \isDraftDebugger{\\ Ref key: \tabLabel{#3}}{}%% Print reference key in draft mode
        }
        %%% Label (must be inside table env, after caption):
        \label{\tabLabel{#3}} % tab:filename
}
{
    \end{table}
}

\NewDocumentCommand{\wrtxTabRef}
{
    m % reference key
    O{Table} % reference prefix
}{%
  \begingroup%
  \ifthenelse{\boolean{isDraft}}{%
    \color{wrtxColorWarning}%
    {\texttt{(\tabLabel{#1})}}%
  }%
  {%
    \color{wrtxGrayMed}%
  }
  #2\ \ref{\tabLabel{#1}}%
  \endgroup%
}

  
    %%%% Margin notes
    \usepackage{marginnote}

    %%% add margin notes at a specific point in the document with:
    % \marginpar[left text]{right text}


    %%% Typesettings margin notes
    % \marginparwidth: % determines the width of margin notes and thus the length of lines typeset in the margin note.
    % \marginparsep: % sets the gap (distance) between margin notes and the text of your document.
    % \marginparpush: % defines the minimum separation (vertical) distance between \marginpar notes.

  \ifthenelse{\boolean{isSplitInTwoColumns}}
{
  \newcommand{\textboxXmargin}{5pt}
  \newcommand{\textboxwidth}{0.98\linewidth}
}
{
  \newcommand{\textboxXmargin}{35pt}
  \newcommand{\textboxwidth}{0.95\linewidth}
}

% Define default box style
\tcbset{
  myTcbBaseStyle/.style={
    center, % enlarges bounding box on both sides to fill he line completely
    colback=myGrayLight!50, % Background color
    colframe=myGrayMed, % Outline color
    coltitle=myGrayLight, % Title background color
    sharp corners=south, % Square bottom corners
    rounded corners=north, % Rounded top corners
    boxrule=1pt, % Outline thickness
    top=15pt, % Inner top margin
    bottom=20pt, % Inner bottom margin
    left=\textboxXmargin, % Inner left margin
    right=\textboxXmargin, % Inner right margin
    arc=5pt, % Corner rounding radius
    fonttitle=\bfseries, % Title font
    width=\textboxwidth, % Box width
    before skip=20pt, % Vertical space before the box
    after skip=20pt, % Vertical space after the box
    title after break, % Keep title on broken pages
    breakable, % Allow breaking across pages
    enhanced, % Enable advanced TikZ features
    titlerule=0mm, % Optional title line
    % title={\strut}, % Dummy title; replace with actual text
  },
}

\newtcolorbox[]{myTextBoxBase}[2][]{%
myTcbBaseStyle,
% float, % allow to float ("nofloat" also possible)
% floatplacement=tbp, % default positioning: "{t}op", "{b}ottom", "{p}age" or "{h}ere"
drop fuzzy shadow southeast=myGrayDark,
title after break={#2 -- cont.}, % repeat title after each page break
title={#2},
#1% optional arguments
}

%%% List of Textboxes
\newcommand{\listtextboxname}{My list of Textboxes}
\newlistof{textbox}{loTB}{\listtextboxname}

%%% Textboxes command
\newcommand\myTBLinker{textbox:\the\value{totalTextboxesInArticle:\myarticleKeyCore}}% key to identify individual textboxes
\NewDocumentCommand{\myTextBox}
{
  O{}%optional arguments
  m% title
  m% content
}
{%
\ifthenelse{\boolean{isIncludeTextBoxes}}{%
  \ifthenelse{\boolean{isMoveTextBoxesToEndOfArticle}}
  {%%% Move textbox to end
    \moveTextboxToEnd{%
      \textboxFrame{#2}{#3}
      [nofloat]

      %%% List of Textboxes entry:
      \addLineToToTB{#2}
    }
  }
  {%%% ELSE add textbox as a float
      \textboxFrame{#2}{#3}
      [% /!\ NO SPACES HERE
        float, % allow to float ("nofloat" also possible)
        floatplacement=tbp, % default positioning: "{t}op", "{b}ottom", "{p}age" or "{h}ere"
        #1%
      ]%
    %%% List of Textboxes entry:
    \addLineToToTB{#2}
  }
}{}
}

\newcommand{\addLineToToTB}[1]
{%
  \isPortfolio{%
    %%%% PORTFOLIO
    \addcontentsline{loTB}{textbox}
    {%
      \protect\myContentsTextFont% change font
      \protect%
      \hypertarget{\myTBLinker}%
      {\indent\protect\numberline{%
        \thechapter.% Add chapter nb
        \the\value{totalTextboxesInArticle:\myarticleKeyCore}}%
        \ \ \ #1}%
    }
  }
  {%
    %%%% SINGLE ARTICLE
    \addcontentsline{loTB}{textbox}
    {%
      \protect\myContentsTextFont% change font
      \protect%
      \hypertarget{\myTBLinker}%
      {\indent\protect\numberline{%
        \the\value{totalTextboxesInArticle:\myarticleKeyCore}}%
        \ \ \ #1}%
    }
  }%
  \par%
}

\newcommand{\textboxTitle}[1]{%
  \textcolor{myGrayDark}{\TEXTtextbox\ \the\value{totalTextboxesInArticle:\myarticleKeyCore}:} \detokenize{#1}%
}

\NewDocumentCommand{\textboxFrame}
{
  m % textbox title
  m % textbox content
  O{} % optional arguments
}
{
  \stepcounter{totalTextboxesAltogether}%
  \refstepcounter{totalTextboxesInArticle:\myarticleKeyCore}%
  \begin{myTextBoxBase}[#3]
    {% TEXTBOX TITLE
        {\phantomsection%
        \ifthenelse{\boolean{isIncludeLoTB}}
        {% as hyperlink
          \hyperlink{\myTBLinker}{%
            \textboxTitle{#1}%
          }%
        }%
        {% as simple text
          \textboxTitle{#1}%
        }%
        }
    }%
    \label{\myTBLinker}% increment counter and ref it
    \isDraftDebugger{Ref key: \myTBLinker\\}{}%% Print reference key in draft mode
    % TEXTBOX CONTENTS
    % \onehalfspacing
    #2
  \end{myTextBoxBase}
}


\NewDocumentCommand{\myTextboxRef}
{
    m % reference key
    O{\TEXTtextbox} % reference prefix
}{%
  \begingroup%
  \ifthenelse{\boolean{isDraft}}{%
    \color{myColorWarning}%
    \texttt{(textbox:#1)}%print ref for debugging
  }
  {%
    \color{myGrayMed}%
  }
  %Print ref
  #2\ \ref{textbox:#1}%
  \endgroup%
}

  
%%% COUNT TOTAL ARTICLES
\newtotcounter{totalArticlesAltogether}
% \setcounter{totalArticlesAltogether}{0}
%%% COUNT TOTAL CITATIONS
\newtotcounter{totalCitationsAltogether}
%%% COUNT TOTAL GLOSSARY ENTRIES
\newtotcounter{totalGlossaryEntriesAltogether}
%%% COUNT TOTAL ABREVIATIONS ENTRIES
\newtotcounter{totalAbreviationsAltogether}
%%% COUNT TOTAL ARTICLE SECTIONS ENTRIES
\newtotcounter{totalArticleSectionsAltogether}
%%% COUNT TOTAL ARTICLE SUBSECTIONS ENTRIES
\newtotcounter{totalArticleSubsectionsAltogether}
%%% COUNT TOTAL ARTICLE SUBSUBSECTIONS ENTRIES
\newtotcounter{totalArticleSubsubsectionsAltogether}
%%% COUNT TOTAL FIGURES ENTRIES (FIGURES + WRAPFIGURES: only figures that are counted)
\newtotcounter{totalFiguresAltogether}
%%% COUNT TOTAL FIGURE ENVIRONMENTS ENTRIES
\newtotcounter{totalFigureEnvsAltogether}
%%% COUNT TOTAL WRAPFIGURE ENVIRONMENTS ENTRIES
\newtotcounter{totalWrapfigureEnvsAltogether}
%%% COUNT TOTAL SUBFIGURES ENTRIES
\newtotcounter{totalSubfiguresAltogether}
%%% COUNT TOTAL GRAPHICS ENTRIES (input images)
\newtotcounter{totalGraphicsAltogether}
%%% COUNT TOTAL TIKZ ENTRIES
\newtotcounter{totalTikZPicturesAltogether}
%%% COUNT TOTAL TABLES ENTRIES
\newtotcounter{totalTablesAltogether}
%%% COUNT TOTAL TEXTBOXES ENTRIES
\newtotcounter{totalTextboxesAltogether}
%%% COUNT TOTAL FOOTNOTES ENTRIES
\newtotcounter{totalFootnotesAltogether}
%%% COUNT TOTAL APPENDIX ENTRIES
\newtotcounter{totalAppendixItemsAltogether}

%%%%%%%%%%%% PORTFOLIO ONLY COUNTERS
%%% COUNT TOTAL PORTFOLIO PARTS
\newtotcounter{totalPortfolioPartsAltogether}




%%%%%% CREATING COUNTERS
\AfterPreamble{
    \isPortfolio{}{ %%% if individual article
        \newcommand{\articleList}{\wrtxarticleKeyCore}
    }
    %%% GENERATE KEYS
    \foreach \article in \articleList {
        %%% GENERAL COUNTERS
        \stepcounter{totalArticlesAltogether}
        %%% ARTICLE SPECIFIC COUNTERS
        \expanded{\noexpand\newtotcounter{lastPageInArticle:\article}}%
        \expanded{\noexpand\newtotcounter{totalSectionsInArticle:\article}}%
        \expanded{\noexpand\newtotcounter{totalSubsectionsInArticle:\article}}%
        \expanded{\noexpand\newtotcounter{totalSubsubsectionsInArticle:\article}}%
        \expanded{\noexpand\newtotcounter{totalCitationsInArticle:\article}}%
        \expanded{\noexpand\newtotcounter{totalFiguresInArticle:\article}}% (FIGURES + WRAPFIGURES: only figures that are counted)
        \expanded{\noexpand\newtotcounter{totalFigureEnvsInArticle:\article}}%
        \expanded{\noexpand\newtotcounter{totalWrapfigureEnvsInArticle:\article}}%
        \expanded{\noexpand\newtotcounter{totalSubfiguresInArticle:\article}}%
        \expanded{\noexpand\newtotcounter{totalGraphicsInArticle:\article}}%
        \expanded{\noexpand\newtotcounter{totalTikZPicturesInArticle:\article}}%
        \expanded{\noexpand\newtotcounter{totalTablesInArticle:\article}}%
        \expanded{\noexpand\newtotcounter{totalTextboxesInArticle:\article}}%
        \expanded{\noexpand\newtotcounter{totalGlossaryEntriesInArticle:\article}}%
        \expanded{\noexpand\newtotcounter{totalAbreviationsInArticle:\article}}%
        \expanded{\noexpand\newtotcounter{totalFootnotesInArticle:\article}}%
        % \expanded{\noexpand\newtotcounter{totalAppendixItemsInArticle:\article}}% USELESS FOR NOW
    }
}

%%% command to convert a number to an roman character
\newcommand*\makeAlph[1]{\symbol{\numexpr64+#1}}% 96 for lowercase alpha, 64 for uppercase

\NewDocumentCommand{\wrtxListLabelStyle}
    {
        m
        O{wrtxColorSecondary}
    }
    {%
        \textcolor{#2}%
        {%
            \textbf{%
                #1%
            }%
        }%
    }%

\newcommand{\wrtxTotalCount}[1]{%
    % \textcolor{green}{\total{#1}}% TEST
    \ifthenelse{\totvalue{#1}=0}{%
    %%% IF VALUE IS 0
        \wrtxListLabelStyle{%
            \total{#1}%
        }%
        [wrtxColorPrimary!50]%
    }%
    {%%%% IF VALUE IS NOT 0
        \wrtxListLabelStyle{%
            \total{#1}%
        }%
    }%
}%

\usepackage{zref-totpages}
\usepackage{lastpage}

\newcommand{\countersList}{
    \subsection*{Counters}
    %%%%%%%%%%%%%%%%%%%%%%%%%%%%%%%%%%%%%%%%%%%%%%%%%%%%%%%%%%%%%%%
    %%%%% FIRST COLUMN
    \begin{minipage}[t]{0.40\textwidth} % Adjust width as needed
        \begin{wrtxListMeta}[
            % leftmargin=120pt,
            labelsep=0pt
            ]
            \item[\wrtxTotalCount{totalFiguresInArticle:\wrtxarticleKeyCore}]\ Figure(s).% simple, wrapped or with subfigures (each group of subfigures counts as 1 here)
            %
            \item[\wrtxTotalCount{totalFigureEnvsInArticle:\wrtxarticleKeyCore}]\ Figure Environment(s).% Figure environments
            %
            \item[\wrtxTotalCount{totalWrapfigureEnvsInArticle:\wrtxarticleKeyCore}]\ Wrapfigure Environment(s).% Wrapfigure environments
            %
            \item[\wrtxTotalCount{totalSubfiguresInArticle:\wrtxarticleKeyCore}]\ Subfigure(s).% subfigures inside figures
            %
            \item[\wrtxTotalCount{totalGraphicsInArticle:\wrtxarticleKeyCore}]\ Graphic input(s).% through \wrtxFigGraphics macro
            %
            \item[\wrtxTotalCount{totalTikZPicturesInArticle:\wrtxarticleKeyCore}]\ TiKz picture(s).% throught \wrtxFigTikZ macro
            %
            \item[\wrtxTotalCount{totalTablesInArticle:\wrtxarticleKeyCore}]\ Table(s).
            %
            \item[\wrtxTotalCount{totalTextboxesInArticle:\wrtxarticleKeyCore}]\ Textbox(es).
            %
        \end{wrtxListMeta}
    \end{minipage}
    % \hfill % This adds space between the minipages
    %%%%%%%%%%%%%%%%%%%%%%%%%%%%%%%%%%%%%%%%%%%%%%%%%%%%%%%%%%%%%%%
    %%%%% SECOND COLUMN
    \begin{minipage}[t]{0.23\textwidth}
        \begin{wrtxListMeta}[
            leftmargin=0pt,
            labelsep=0pt
            ]
            % \item[\wrtxTotalCount{wrtxGlossCounterAlt}]\ Glossary Entri(es).
            %
            \item[\wrtxTotalCount{totalGlossaryEntriesInArticle:\wrtxarticleKeyCore}]\ Glossary Entry calls.
            %
            \isPortfolio{}{\item[\wrtxTotalCount{totalGlossaryEntriesAltogether}]\ Glossary Entri(es) (unique).}
            %
            \item[\wrtxTotalCount{totalAbreviationsInArticle:\wrtxarticleKeyCore}]\ Abreviation calls.
            %
            \isPortfolio{}{\item[\wrtxTotalCount{totalAbreviationsAltogether}]\ Abreviation(s) (unique).}
            %
            \item[\wrtxTotalCount{totalCitationsInArticle:\wrtxarticleKeyCore}]\ Citation(s) (repeats included).
            %
            \item[\wrtxTotalCount{totalFootnotesInArticle:\wrtxarticleKeyCore}]\ Non-citation Footnote(s).
            %
        \end{wrtxListMeta}
    \end{minipage}
    %%%%%%%%%%%%%%%%%%%%%%%%%%%%%%%%%%%%%%%%%%%%%%%%%%%%%%%%%%%%%%%
    %%%%% THIRD COLUMN
    \begin{minipage}[t]{0.33\textwidth}
        \begin{wrtxListMeta}[
            leftmargin=0pt,
            labelsep=0pt
            ]
            \isPortfolio{}{\item[\wrtxListLabelStyle{\ztotpages}]\ total page(s) (body + extras).}
            %
            \item[\wrtxTotalCount{lastPageInArticle:\wrtxarticleKeyCore}]\ is the last page in the article.
            %
            \item[%
            \wrtxListLabelStyle{%
            \totalPagesInArticleBody
            }]\ total pages in article.
            %
            \item[\wrtxTotalCount{totalSectionsInArticle:\wrtxarticleKeyCore}]\ Article Section(s).
            %
            \item[\wrtxTotalCount{totalSubsectionsInArticle:\wrtxarticleKeyCore}]\ Article Subsection(s).
            %
            \item[\wrtxTotalCount{totalSubsubsectionsInArticle:\wrtxarticleKeyCore}]\ Article Subsubsection(s).
            %
            \isPortfolio{}{\item[\wrtxTotalCount{totalAppendixItemsAltogether}]\ Appendix Entri(es).}
            %
        \end{wrtxListMeta}
    \end{minipage}
}

\usepackage{refcount}
\newcommand{\pagedifference}[2]{%
  \number\numexpr#1+1-#2\relax%
}

\newcommand{\totalPagesInArticleBody}{
    \pagedifference%
    {\totvalue{lastPageInArticle:\wrtxarticleKeyCore}}% Last page in article
    {\getpagerefnumber{\wrtxarticleKey:article_header}}% title page in article
}

\newcommand{\subtactOne}[1]
{
    \the\numexpr#1-1\relax
}

% Absolute current page number, regardless of numbering
\newcommand\abspagenumber{\inteval{\ReadonlyShipoutCounter+1}}

  \RequirePackage{l3benchmark}
\ExplSyntaxOn
\AfterEndDocument { \benchmark_toc: }
\use:n
  {
    \ExplSyntaxOff
    % PRINT COMPILE DURATION TO LOG (look for "(l3benchmark) + TOC" in .log file)
    %%% print correspond to one TeX run only
    \benchmark_tic:
  }


%%%% LOAD HYPERREF
\usepackage[hidelinks]{hyperref} % clickable refs. should be loaded late - with a few notable exceptions, it should be loaded last
\urlstyle{default} % Set the URL style typewriter(tt),same as surrounding text (same), bold (bf), italic (it), smallcaps (sc), default (default)
%%%% LOAD AFTER HYPERREF

%\usepackage{atbegshi} % add custom frame to article pages
\newcommand{\ribbonTextLeft}{Ribbon text (Left)}
\newcommand{\ribbonTextRight}{Ribbon text (Right)}
\newcommand{\ribbonShiftToCenter}{0.5} % cm
\newcommand{\ribbonWidth}{0.75} % cm
\definecolor{ribbonBgColorStart}{gray}{0.7}
\definecolor{ribbonBgColorEnd}{gray}{0.5}
\definecolor{ribbonTextColor}{gray}{0.1}

\newcommand{\updateRibbons}[2]{
      \renewcommand{\ribbonTextLeft}{#1}
      \renewcommand{\ribbonTextRight}{#2}
}

\newcommand{\ribbonSpacer}{\hspace{5cm}}

\usetikzlibrary{shapes.arrows,positioning}

\tikzset{
    myarrow/.style={
        draw=myGrayDark,
        fill=myGrayLight,
        single arrow,
        rotate=90,
        minimum height=10mm,
        anchor=center
    }
}


\AtBeginShipout{\AtBeginShipoutUpperLeft{
      \ifthenelse{\boolean{isDrawRibbons}}{
            %%% WARNING: HYPERLINKS DO NOT WORK IN ROTATED TEXT)
            \begin{tikzpicture}[remember picture,overlay]%
                  %%% LEFTSIDE
                  \fill[%black!5
                        % for gradient instead:
                        left color=ribbonBgColorStart,
                        right color=ribbonBgColorEnd,
                        shading = axis,
                        shading angle = 90
                        ]
                        ($(current page.north west)+(\ribbonShiftToCenter,0)$)rectangle ++(\ribbonWidth,-\paperheight);
                  \node[rectangle,rotate=90] (leftNode) at ($(current page.west)+(\ribbonShiftToCenter+\ribbonWidth/2, 0)$)
                  {\small\textcolor{ribbonTextColor}{\ribbonTextLeft}};
                  %
                  %
                  %%% RIGHTSIDE
                  \fill[%black!5
                        % for gradient instead:
                        left color=ribbonBgColorStart,
                        right color=ribbonBgColorEnd,
                        shading = axis,
                        shading angle = -90
                        ]
                        ($(current page.north east)-(\ribbonShiftToCenter,0)$) rectangle ++(-\ribbonWidth,-\paperheight);
                  \node[rectangle,rotate=90] (rightNode) at ($(current page.east)-(\ribbonShiftToCenter+\ribbonWidth/2, 0)$)
                  {\normalfont\textcolor{ribbonTextColor}{\ribbonTextRight}};
                  %%%% LINK TO portfolio cover
                  \node (leftLink) at ($(leftNode)+(0, 0.45\pageheight)$)
                  {\hyperlink{start}{\textbf{\textcolor{myColorSecondary}{start}}}};
                  \node[myarrow] at ([yshift=5mm]leftLink.north) {};
                  %%%% LINK TO TOC
                  \node (rightLink) at ($(rightNode)+(0, 0.45\pageheight)$)
                  {\hyperlink{toc}{\textbf{\textcolor{myColorSecondary}{TOC}}}};
                  \node[myarrow] at ([yshift=5mm]rightLink.north) {};
            \end{tikzpicture}%
      }{}
}}

% \usepackage{datatool}% http://ctan.org/pkg/datatool

\newcommand{\wrtxGlossaryBullet}{\bullet}
\newcommand{\wrtxGlossaryKeyColor}{wrtxColorPrimary}

% Glossary counter
\newtotcounter{wrtxGlossCounterAlt}
\setcounter{wrtxGlossCounterAlt}{0} % change counter value to a starting specific value

%%% SORTING ALPHABETICALLY
%%% Source: https://tex.stackexchange.com/questions/121489/alphabetically-display-the-items-in-itemize/121492#121492

\newcommand{\sortitem}[2][\relax]{%
  \DTLnewrow{list}% Create a new entry
  \ifx#1\relax
    \DTLnewdbentry{list}{sortlabel}{#2}% Add entry sortlabel (no optional argument)
  \else
    \DTLnewdbentry{list}{sortlabel}{#1}% Add entry sortlabel (optional argument)
  \fi%
  \DTLnewdbentry{list}{description}{#2}% Add entry description
}


\newenvironment{sortedlist}{%
  \DTLifdbexists{list}{\DTLcleardb{list}}{\DTLnewdb{list}}% Create new/discard old list
}{%
  \DTLsort*{sortlabel}{list}%  Sort list (replace the \DTLsort command by \DTLsort* for case-insensitive comparison)
  %
  \begin{itemize}[label=\wrtxGlossaryBullet]%
    \DTLforeach*{list}{\theDesc=description}{%
      \item \theDesc}% Print each item
  \end{itemize}%
}


\newcommand{\wrtxGlossary}[1]{
    % \total{wrtxGlossCounterAlt}
    \ifthenelse{\totvalue{wrtxGlossCounterAlt}=0}{ %%% Include only if there are glossary items
      % No glossary items in \wrtxLanguage
    }
    {
      \newpage
      \wrtxSectionTitle{Glossary}
      \begin{sortedlist} % order items alphabetically
          \input{#1}
      \end{sortedlist}
      % Total number of glossary items in this document: \total{wrtxGlossCounterAlt}
    }
}



\NewDocumentCommand{\glossaryItemAlt}{O{en} m m}{
    \glossaryItemAltDebug{#1} % line to check if matches are ok
    \ifthenelse{\equal{\wrtxLanguage}{#1}}{
        \stepcounter{wrtxGlossCounterAlt} % increment counter
        \sortitem[#2]{
            \normalsize\textcolor{\wrtxGlossaryKeyColor}{#2}: #3
            }
    }{}
}


\newcommand{\glossaryItemAltDebug}[1]{
    \ifthenelse{\boolean{isDraft}}
    {%
    % \text{%
      \tiny%
      \mbox{
        \noindent%
        \nobreak\thewrtxGlossCounterAlt: \wrtxLanguage\ = #1 >\ %
        \ifthenelse{\equal{\wrtxLanguage}{#1}}{\textcolor{wrtxColorSuccess}{T}}{\textcolor{wrtxColorWarning}{F}}%
      }%
    % }
    }
    {}
  }

\usepackage[
  acronym,
  section, % to avoid page break if placed after chapter heading
  nopostdot=true, % add/remove "." after each definition
  nogroupskip=false, % add/remove skip after each group
  nonumberlist % add/remove references to appearances by page number
]{glossaries}%import glossary as well as acronym features

% Change glossary/acronym nsame font
\renewcommand{\glsnamefont}[1]{\textsf{\textcolor{wrtxColorPrimary}{#1}}}
% Change glossary/acronym page number style (if "nonumberlist" is off)
%\renewcommand{\glossaryentrynumbers}[1]{\textcolor{wrtxColorPrimary}{#1}}
% Change glossary/acronym alphabetic navigation line style
\renewcommand*{\glslistnavigationitem}[1]
{\item \textcolor{wrtxColorPrimary}{\textbf{#1}}}
% Change glossary/acronym alphabetic letter grouping style
\renewcommand*{\glslistgroupheaderfmt}[1]
{\textcolor{wrtxColorPrimary}{\textbf{#1}}}

\makenoidxglossaries

%%% GLOSSARY STYLE
\newcommand{\glossaryStyle}{}
\ifthenelse{\boolean{isPrintVersion}}
{
  \renewcommand{\glossaryStyle}{listgroup}
}
{
  \renewcommand{\glossaryStyle}{listhypergroup}
}
\setglossarystyle{\glossaryStyle}
%%% Style options:
% list. Writes the defined term in boldface font
% altlist. Inserts newline after the term and indents the description.
% listgroup. Group the terms based on the first letter.
% listhypergroup. Adds hyperlinks at the top of the index.


\NewDocumentCommand{\glossaryItem}
{
  O{en} % LANGUAGE
  m % KEY
  O{} % NAME
  m % DESCRIPTION
  O{} % ABBREVIATION
}
{
    \ifthenelse{\equal{\wrtxLanguage}{#1}}{
        \stepcounter{totalGlossaryEntriesAltogether} % increment counter

        %%% CREATE NEW GLOSSARY ENTRY
        % If #3 is empty, use #2 as name
        \ifthenelse{\equal{#3}{}}
        {% IF THEN
          \newglossaryentry{#2}
          {
              name={#2}, % Reuse entry
              description={\wrtxGlossaryDescription{#2}{#4}{#5}}
          }
        }
        %%%%%%%%%%%%%%%%%%%%%%%%%%%%%%%%%%%%%%%%%%%%%%%%%%%%%%%%%%%%%%%%%%%%%%
        {% ELSE
          \newglossaryentry{#2}
          {
              name={#3},
              description={\wrtxGlossaryDescription{#2}{#4}{#5}}
          }
        }
    }{}
}

\NewDocumentCommand{\wrtxGlossaryDescription}{
  m % glossary key
  m % Description
  m % Abreviation key
}{%
\isDraftDebugger{[KEY: \detokenize{#1}\ ]}{}
#2 \includeAbrevInGlossary{#3}%
}

\NewDocumentCommand{\includeAbrevInGlossary}
{
  m % abbreviation key
}{%
  \ifthenelse{\equal{#1}{}}% check if empty
  {\isDraftDebugger{no abbreviation}{}}
  {Abbreviated as \acrshort{#1}.}
}



\NewDocumentCommand{\abreviationsItem}
{
  O{en} % LANGUAGE
  m % KEY
  O{} % NAME
  m % MEANING
}
{
    \ifthenelse{\equal{\wrtxLanguage}{#1}}{
        \stepcounter{totalAbreviationsAltogether} % increment counter
        %%% CREATE NEW GLOSSARY ENTRY
        % If #3 is empty, use #2 as name
        \ifthenelse{\equal{#3}{}}
        {% IF THEN
        \newacronym{#2}{#2}{\isDraftDebugger{[abrev. key: #2]}{}[red]#4}
        }
        {% ELSE
        \newacronym{#2}{#3}{\isDraftDebugger{[abrev. key: #2]}{}[red]#4}
        }
    }{}
}

% FORMAT GLOSSARY/ABBREVIATION LINK IN TEXT
\ifthenelse{\boolean{isDraft}}
{ % DRAFT MODE
  \renewcommand*{\glstextformat}[1]%
  {%
    \texttt{%
    \colorbox{wrtxColorSuccess}%
    {%
      \textcolor{wrtxColorWarning}{#1}%
    }%
    }%
    }%
}
{ % NORMAL MODE
  \renewcommand*{\glstextformat}[1]
  {%
  \ifthenelse{
    \boolean{isHighlightGlossaryAndAbreviations}
    % \AND
    % \(
    % \boolean{isIncludeGlossary}
    % \OR
    % \boolean{isIncludeAbreviations}
    % \)
    %
    }%
  {%
    \textbf{%
      \textcolor{wrtxGrayDark}
      {%
        #1%
      }%
      }%
    }%
  }{}%
}





\newcommand{\addAbreviations}{

  \ifthenelse{
    \boolean{isIncludeAbreviations}
    \AND
      \(
      \boolean{isPrintUnusedAbreviations}
      \OR
      \totvalue{totalAbreviationsInArticle:\wrtxarticleKeyCore}>0
      \)
    \AND
    \totvalue{totalAbreviationsAltogether}>0
    }{
    \newpage
    \begin{SplitColumnsInTwo}%[true]
      \updateRibbons{\textbf{Abreviations}}{}
      {
        \singlespacing
        \wrtxSectionTitle{Abreviations and Acronyms}
        % Print Abreviations
        \printnoidxglossary[
          type=\acronymtype,
          title={},
          % toctitle=List of terms
          ]
        % Add even unused entries
        \ifthenelse{\boolean{isPrintUnusedAbreviations}}{
          \glsaddallunused[\acronymtype]
        }{}
      }
    \end{SplitColumnsInTwo}
  }
  {}
}



\newcommand{\addGlossary}{
  \ifthenelse{
    \boolean{isIncludeGlossary}
    \AND
      \(
      \boolean{isPrintUnusedGlossary}
      \OR
      \totvalue{totalGlossaryEntriesInArticle:\wrtxarticleKeyCore}>0
      \)
    \AND
    \totvalue{totalGlossaryEntriesAltogether}>0
  }{
    \newpage
    \begin{SplitColumnsInTwo}%[true]
      \updateRibbons{\textbf{Glossary}}{}
      {
        \singlespacing
        \wrtxSectionTitle{Glossary}
        % Print glossary
        \printnoidxglossary[
          type=main,
          title={},
          %  toctitle=List of terms
          ]
          % Add even unused entries
          \ifthenelse{\boolean{isPrintUnusedGlossary}}{
            \glsaddallunused[main]
          }{}
      }
    \end{SplitColumnsInTwo}
  }
  {}
}



\newcommand{\addAbreviationsPORTFOLIO}{
  \ifthenelse{\boolean{isIncludeAbreviations} \and 1>0}{
    % \newpage
    \cleardoublepage
    \begin{SplitColumnsInTwo}
    \wrtxPortfolioChapter{Abreviations}
    {
      \phantomsection
      \singlespacing
      %%%
      % Print Abreviations
      \renewcommand*{\glsclearpage}{} % change default break behavior before glossary
      \printnoidxglossary[
        type=\acronymtype,
        % style=listgroup,
        title={}
        % toctitle={}% Dont comment out
      ]
      % Add even unused entries
      \ifthenelse{\boolean{isPrintUnusedAbreviations}}{
        \glsaddallunused[\acronymtype,]
      }{}
    }
    \end{SplitColumnsInTwo}
  }
  {}
}


\newcommand{\addGlossaryPORTFOLIO}{
  \ifthenelse{\boolean{isIncludeGlossary} \and 1>0}{
    % \newpage
    \cleardoublepage
    \begin{SplitColumnsInTwo}
    \wrtxPortfolioChapter{Glossary}
    {%
      \phantomsection
      \singlespacing
      %%%
      % Print glossary
      \renewcommand*{\glsclearpage}{} % change default break behavior before glossary
      \printnoidxglossary[
        type=main,
        % style=listgroup,
        title={}
        % toctitle={}% Dont comment out
      ]
      % Add even unused entries
      \ifthenelse{\boolean{isPrintUnusedGlossary}}{
        \glsaddallunused[main]
      }{}
    }
    \end{SplitColumnsInTwo}
  }
  {}
}

\NewDocumentCommand{\wrtxGLS}%
{%
  m% glossary key
  O{}% optional text to be used instead of name from glossary
}{%
  {%
    \ifthenelse{{\equal{\wrtxarticleKeyCore}{}}}% check if key is still empty
    {}% if so, do nothing, since it means the article has not yet started
    {% else, step counter
      \stepcounter{totalGlossaryEntriesInArticle:\wrtxarticleKeyCore}%
    }%
    \ifthenelse{\boolean{isIncludeGlossary}}%
    {%
      %GLOSSARY INCLUDED
      \isDraftDebugger{[gloss. key: \detokenize{#1}]}%
      {%
        \ifthenelse{%
          \equal{#2}{}%
          }%
          {\gls{#1}}% if no optional text was provided
          {\glslink{#1}{#2}}% if optional text was provided
      }%
    }%
    {%
      %GLOSSARY NOT INCLUDED
      \isDraftDebugger{[gloss. key: \detokenize{#1}]}%
      {%
        \ifthenelse{%
          \equal{#2}{}%
          }%
          {\gls*{#1}}% if no optional text was provided
          {\glslink*{#1}{#2}}% if optional text was provided
      }%
    }%
  }%
}

\NewDocumentCommand{\wrtxAbrev}%
{%
  m% abreviation key
  O{}% optional text to be used instead of name from abreviation
}{%
  {%
    \ifthenelse{{\equal{\wrtxarticleKeyCore}{}}}% check if key is still empty
    {}% if so, do nothing, since it means the article has not yet started
    {% else, step  counter
      \stepcounter{totalAbreviationsInArticle:\wrtxarticleKeyCore}%
    }%
    \ifthenelse{\boolean{isIncludeAbreviations}}%
    {%
      %GLOSSARY INCLUDED
      \isDraftDebugger{[abrev. key: \detokenize{#1}]}%
      {%
        \ifthenelse{%
          \equal{#2}{}%
          }%
          {\gls{#1}}% if no optional text was provided
          {\glslink{#1}{#2}}% if optional text was provided
      }%
    }%
    {%
      %GLOSSARY NOT INCLUDED
      \isDraftDebugger{[abrev. key: \detokenize{#1}]}%
      {%
        \ifthenelse{%
          \equal{#2}{}%
          }%
          {\gls*{#1}}% if no optional text was provided
          {\glslink*{#1}{#2}}% if optional text was provided
      }%
    }%
  }%
}



% commands to move textboxes on end of document/chapter/setion/etc

\newcommand{\moveTextboxToEnd}
[1]
{%
    \ifthenelse{\boolean{isMoveTextBoxesToEndOfArticle}}
    {%
        \immediate\write\writeenditems{%
        \detokenize{#1}}%
    }%
    {%
        #1%
    }%
}



\AtBeginDocument{%
  \newwrite\writeenditems
  \immediate\openout\writeenditems=wrtxfile.tmp
  }


\newcommand{\printMovedContents}
{
    \immediate\closeout\writeenditems

    % This branch may need some tests s.a. \IfFileExists{wrtxfile.tmp}{True}{False}
    \clearpage
    \newread\readenditems
    \immediate\openin\readenditems=wrtxfile.tmp
    \loop
    \immediate\read\readenditems to\linein
    \linein
    \ifeof\readenditems
    \else\repeat
    \immediate\closein\readenditems
}

% To print at the VERY END of the document
% \AtEndDocument{%
%     \printMovedContents{}
% }

% To print where this command is placed
\newcommand{\postponeTextBoxPrintTillHere}{%
    \noindent\isDraftDebugger{\wrtxBreakMessage[purple][green]{TEXTBOX CONTAINER WHEN MOVED TO END (start)}}{}% visually mark float barrier in pdf
    \FloatBarrier % Do not let floats past here UNCONDITIONALLY (do not use wrtxFloatBarrier)
    %
    %
    \ifthenelse{\boolean{isMoveTextBoxesToEndOfArticle}}
    {%
        \printMovedContents{}%
    }
    {}
    %
    %
    \FloatBarrier % Do not let floats past here UNCONDITIONALLY
    \noindent\isDraftDebugger{\wrtxBreakMessage[purple][green]{TEXTBOX CONTAINER WHEN MOVED TO END (finish)}}{}%
}
% To use markdown in a LATEX document
\usepackage[
  fancyLists=true
]{markdown}

\newcommand{\wrtxInputREADME}{%
  \begin{tcolorbox}[
      breakable,
      title={README contents},
      width={0.975\pagewidth},
      center,
      % show bounding box % to show where box is placed
    ]
      \isPortfolio{%
        \markdownInput{../../../README.md}%
      }%
      {%
        \markdownInput{../../README.md}%
      }%
  \end{tcolorbox}
}

% \AfterPreamble{
%   %%%% NOT WORKING !!!!
%   % directory where the auxiliary latex file will be created
%   \markdownOptionOutputDir{hello}
%   % filename
%   \markdownOptionInputTempFileName{jjjj}
% }







\NewDocumentCommand{\myAppendix}
{
    % m
}
{   \ifthenelse{
        \boolean{isIncludeAppendix}%
        \AND%
        \not\equal{\appendixList}{}% if appendix is empty, return false
    }
    {
        \updateRibbons{\textbf{Appendix}}{}
        %%%%%%%%%%%%%%%%%%%%%%%%%%%%%%%%%%%%%%%%%%%%%%%%%%
        % APPENDIX
        \appendix
        % https://latex-tutorial.com/latex-appendix/
        % \clearpage
        \pagestyle{plain}
        \pagenumbering{Roman}
        % % counter for appendix items
        % \cleardoublepage


        \foreach \item [count=\n] in \appendixList {%
            \ifthenelse{\equal{\item}{}}%
            {}% if appendix item is empty, skip it
            {%
                \cleardoublepage%
                \refstepcounter{totalAppendixItemsAltogether}%
                \expandafter\addAppendixItem\expandafter{\item}%
            }%
        }%
    }%
    {}%
}%


\NewDocumentCommand{\addAppendixItem}
{
    m % appendix key
}
{%
    % \detokenize{./elements/appendix/#1/content.tex}
    \isPortfolio
    {%
        \wrtxAppendixTitle{Appendix Title B}[ref-b-appendix]


\lipsum[1-2]%
    }%
    {%
        \wrtxAppendixTitle{Appendix Title B}[ref-b-appendix]


\lipsum[1-2]%
    }%
}%

\NewDocumentCommand{\myAppendixTitle}
{
    m % Title text
    O{} % optional label for referencing
}{%
    \isPortfolio
    {%
        \titlespacing{\chapter}{0pt}{0pt}{10pt} % hacking way to bring content UP
        \chapter[#1]{%
            \color{myColorPrimary}%
            \setTitleFont%
            \linkAppendixConditionally{#1}%
            }%
    }
    {%
        \section[#1]{%
            \color{myColorPrimary}%
            \setTitleFont%
            \linkAppendixConditionally{#1}
        }%
        \ifthenelse{\equal{#2}{}}
        {}
        {%
            \label{\appendixLabel{#2}}%
            %
            \isDraftDebugger{\ Appendix key: \appendixLabel{#2}}{}%
        }%
    }%
    \hypertarget{\autoAppendixTargetId}{\isDraftDebugger{\\Auto ref: \autoAppendixTargetId}{}}%
}


\NewDocumentCommand{\appendixLabel}
{m}{%
  appendix:\detokenize{#1}% detokenize to treat special characters as plain text
}

\NewDocumentCommand{\myAppendixRef}{
    m
    O{Appendix}
}{\mySecRef{appendix}{#1}[#2]}




\ifthenelse{\boolean{isIncludeToC}}
{ % if TOC included, link to it
  \newcommand{\linkAppendixConditionally}[1]
  {
    \protect\hyperlink
    {toc}% link to TOC
    {#1}%
  }
}
{ % if TOC not included, link to title instead
  \newcommand{\linkAppendixConditionally}[1]
  {%
    #1%
  }
}

% automatically generated target for hyperlink, to compliment the manual one
\newcommand{\autoAppendixTargetId}{%
    appendix:target:\thetotalAppendixItemsAltogether
}

%
%
%
%
% LOAD EXTRA SETTINGS
% lua function to transform a string into a command
\begin{luacode}
    function fontCommand(word)
        return "\\"..word
    end
\end{luacode}
% Define a wrapper command in LaTeX
\newcommand{\fontCmd}[1]{%
    \directlua{tex.print(fontCommand("#1"))}%
}


\begin{luacode}
    ---------------------------------------------------------------------------
    -- variables to hold included/excluded font families
    fontFamiliesExcluded={}
    fontFamiliesIncluded={}
    ---------------------------------------------------------------------------
    -- functions to print length of included/excluded lists
    function fontFamiliesExcludedLength()
        tex.print(#fontFamiliesExcluded)
    end
    function fontFamiliesIncludedLength()
        tex.print(#fontFamiliesIncluded)
    end
    ---------------------------------------------------------------------------
    -- functions to add into included/excluded lists
    function addToExcluded(s)
        if s ~= "" then -- if not empty
            fontFamiliesExcluded[#fontFamiliesExcluded + 1] = s
        end
    end
    function addToIncluded(s)
        if s ~= "" then -- if not empty
            fontFamiliesIncluded[#fontFamiliesIncluded + 1] = s
        end
    end

    ---------------------------------------------------------------------------
    -- function to loop over a list and add commas in between
    function loopOverList(l)
        list={}
        -- Print the list contents
        for i, item in pairs(l) do
            -- if the current item is empty
            if item == "" then
                break
            end
            -- if the current item is the last element
            if i ~= #l then
                list[#list + 1] = item .. ","
            else
                list[#list + 1] = item
            end
        end
        return list
    end

    ---------------------------------------------------------------------------
    -- functions to print included/excluded lists
    function printFamiliesIncluded()
        local list=loopOverList(fontFamiliesIncluded)
        if #list == 0 then
            tex.print("none")
        else
            tex.print(list)
        end
    end

    function printFamiliesExcluded()
        local list=loopOverList(fontFamiliesExcluded)
        if #list == 0 then
            tex.print("none")
        else
            tex.print(list)
        end
    end

\end{luacode}

% lua function wrappers
\NewDocumentCommand{\addToIncluded}
{m%
}{%
    \directlua{addToIncluded("#1")}%
}

\NewDocumentCommand{\addToExcluded}
{m%
}{%
    \directlua{addToExcluded("#1")}%
}

%%%% Filter font families based on whether they are available on the computer or not
\newcommand{\filterFontFamilies}{%
    \foreach \fam [count=\n] in \FontFamilies {%
        \expandafter\IfFontExistsTF\expandafter{\fam}
        {\addToIncluded{\fam}}
        {\addToExcluded{\fam}}
    }%
}

%%% Function to apply a passed string as a style for the following text
\NewDocumentCommand{\applyProperty}
{
    m% parameter to be applied
}
{%
    % check if if passed parameter is a font. Since the font family has already been filtered, if it is a font, it must exist. If it does not exists, it must be a font size or style
    \IfFontExistsTF{#1}%
    {% if it is a font, set it
        \setmainfont{#1}%
    }%
    {%% if it is not a font, apply it
        \fontCmd{#1}%
    }%
}%

% function to style labels
\newcommand{\wrtxLabel}[1]{\textcolor{black!75}{#1}}%%% styling labels

% function to add horizontal line separators between outer loop turns
\newcommand{\outerSeparator}{%
    \noindent\rule{\textwidth}{.4pt}\\[\dimexpr-\baselineskip+1mm+2pt]
    \rule{\textwidth}{2pt}
}

%
%
%
%
% LOAD NECESSARY BIB FILES
\loadBibIfExists{./../../biblatex_files/bibliography_template.bib}
\loadBibIfExists{./../../biblatex_files/myArticles_template.bib}
\loadBibIfExists{./../../biblatex_files/bibliography.bib}
\loadBibIfExists{./../../biblatex_files/myArticles.bib}
%
%
%
%
%%%%%%%%%%%%%%%%%%%%%%%%%%%%%%%%%%%%%%%%%%%%%%%%%%%%%%%%%%%%%%%%%%%%%%%%%%%%%%%
%%%%%%%%%%%%%%%%%%%%%%%%%%%%%%%%%%%%%%%%%%%%%%%%%%%%%%%%%%%%%%%%%%%%%%%%%%%%%%%
%%%%%%%%%%%%%%%%%%%%%%%%%%%%%%%%%%%%%%%%%%%%%%%%%%%%%%%%%%%%%%%%%%%%%%%%%%%%%%%

%%%%%%%%%%% PICK FONT FAMILIES TO BE DEMOED
\newcommand{\FontFamilies}
    {% empty items get ignored (e.g. trailing commas)
        %%% TEST USER DEFINED FONTS IN ARTICLES
        \myMainFont/Main font,
        \myMainFontBackup/Main backup font,
        % \myDraftFont/Draft font,
        % \myDraftFontBackup/Draft backup font,
        % \myTitleFont/Title font,
        % \myTitleFontBackup/Title backup font,
        % \mySubtitleFont/Subtitle font,
        % \mySubtitleFontBackup/Subtitle backup font,
        %%% TEST FONTS DIRECTLY (case insensitive)
        % Comic Sans MS/,
        % Chalkboard/,
        % Cambria/,
        Didot/,
        % Futura/,
        % Herculanum/,
        % Impact/,
        % Phosphate/,
        % Rockwell/,
        % Savoye LET/,
        % Snell Roundhand/,
        % Zapfino/,
    }%

% Identify fonts to include or exclude based on availability
\filterFontFamilies


%%%%%%%%%%% PICK FONT STYLES TO BE DEMOED
\newcommand{\FontStyles}
    {% empty items get ignored (e.g. trailing commas)
        mdseries/Medium,%
        bfseries/Bold,%
        lsstyle/test,%
        itshape/Italic,%
        slshape/Slated,%
        upshape/Upright,%
        scshape/Smallcaps,%
        color{red}/Color red,%
    }%

%%%%%%%%%%% PICK FONT SIZES TO BE DEMOED
\newcommand{\FontSizes}
    {% empty items get ignored (e.g. trailing commas)
        tiny/smallest font,%
        scriptsize/,%
        footnotesize/,%
        small/,%
        normalsize/,%
        large/,
        Large/,%
        huge/,%
        Huge/,%
        HUGE/Largest font,%
    }%

%%%%%%%%%%% DEFINE WITH OF EACH COOLUMN
\def\widthHeadI{2cm}
\def\widthHeadII{3cm}
\def\widthHeadGap{0.25cm}
\def\widthHeadIII{% adjust automatically or add explicitly
    %7cm
    \dimexpr\linewidth-% auto adjust
    \widthHeadI-\widthHeadII-\widthHeadGap\relax%
    -0.2cm% slight adjustment
    }


%%%%%%%%%%% PICK TEXT TO BE DISPLAYED WITH FONTS
\newcommand{\fontDemoText}{%
    % Demonstration text%
    \myCiteEntry{\myarticleKey}{title}
    %
    %%% PANGRAMS
    % The quick brown fox jumps over a lazy dog. % pangram english
    % Portez ce vieux whisky au juge blond qui fume. % pangram french
    % Falsches Üben von Xylophonmusik quält jeden größeren Zwerg. % pangram german
    % El veloz murciélago hindú comía feliz cardillo y kiwi. La cigüeña tocaba el saxofón detrás del palenque de paja. % pangram spanish
    % Gazeta publica hoje no jornal uma breve nota de faxina na quermesse. % pangram portuguese
}

%%%%%%%%%%% Choose which list corresponds  to which layer in the loop
%
%%%%%%%%%%%%%%%%%%%%%%%%%%%%%
% be sure to update labels
\def\aLoopHeadI{}% first header label
\def\aLoopHeadII{Family Macro}% second header label
\def\aLoopHeadIII{}% third header label
\edef\aLoop{% Outer loop
% Pick list to use in the loop:
    \directlua{printFamiliesIncluded()}% available fonts only
    % \FontStyles%
%
}%
%
%%%%%%%%%%%%%%%%%%%%%%%%%%%%%
% be sure to update labels
\def\bLoopHeadI{}% first header label
\def\bLoopHeadII{Size Macro}% second header label
\def\bLoopHeadIII{}% third header label
\edef\bLoop{% Middle loop
% Pick list to use in the loop:
    \FontSizes%
%
}%
%
%%%%%%%%%%%%%%%%%%%%%%%%%%%%%
% be sure to update labels
\def\cLoopHeadI{Effect}% first header label
\def\cLoopHeadII{Style Macro}% second header label
\def\cLoopHeadIII{}% third header label
\edef\cLoop{% Inner loop
% Pick list to use in the loop:
    \FontStyles%
    % \directlua{printFamiliesIncluded()}%
%
}%

\geometry{
          left=0.2cm,
          right=0.2cm,
          top=0.2cm,
          bottom=0.2cm,
          bindingoffset=0cm
          }
\begin{document}
    %%%%%%%%%%%%%%%
    %%%%%%%%%%%%%%%%%%%%%
    %%%%%%%%%%%%%%%%%%%%%%%%%%%
%%%%%%%%%%%%%%%%%%%%%%%%%%%%%%%%%%%%%%%%%%%%%%%%%%%%%%%%%%%%%%%%%%%%%%%%%%%%%%%
    \pagecolor{black!7.5}
    \section*{Fonts Demonstration}
    \singlespacing

    %%%%%%%%%%%% INCLUDED FONTS
    \noindent\textcolor{myColorSuccess!50!black}{Available fonts} :% label
    \ \directlua{printFamiliesIncluded()}% elements
    \ (\directlua{fontFamiliesIncludedLength()})% list length

    %%%%%%%%%%%% EXCLUDED FONTS
    \noindent\textcolor{myColorDanger!50!black}{Unavailable fonts} :% label
    \ \directlua{printFamiliesExcluded()}% elements
    \ (\directlua{fontFamiliesExcludedLength()})% list length


    \iftrue
    %Loop structure:
    \noindent\aLoopHeadI \myLabel{\ (outer loop)}\\
    $\rightarrow$ \bLoopHeadI \myLabel{\ (middle loop)}\\
    ---$\rightarrow$bLoopHeadI \myLabel{\ (inner loop)}\\
    %
    %
    %
    % Initialize the length counter
    \newcounter{aListLength}
    \setcounter{aListLength}{0}
    % Count the length of the list
    \foreach \itemA in \aLoop {
        \stepcounter{aListLength}
    }
    %------------------- INNER LOOP
    \foreach \itemA/\labelA [count=\nA] in \aLoop {%
        \ifthenelse{\equal{\itemA}{}}% check if empty
        {}
        {%
        \subsection*{%
        \myLabel{\aLoopHeadII:} \itemA
        \ \myLabel{%
            (\ifthenelse{\equal{\labelA}{}}{}{\labelA,\ }% label
            \nA\ out of \theaListLength) % current position in loop
        }%
        }%
        %%% Check if font family exists
        %
        %------------------- MIDDLE LOOP
        \foreach \itemB/\labelB  [count=\nB] in \bLoop
        {%
            \ifthenelse{\equal{\itemB}{}}% check if empty
            {}
            {%
            \paragraph*{%
                \dotfill%
                \myLabel{\bLoopHeadII:} \itemB%
                \ifthenelse{\equal{\labelB}{}}{}
                    {\myLabel{\ (\labelB)}}%
                \dotfill%
            }%
            %
            \begin{myListMeta}[
                leftmargin=0pt,
                labelsep=0pt
                ]
                %%% Header
                \item[] %
                    % First head
                    \makebox[\widthHeadI][r]{
                        \myLabel{\cLoopHeadI}%
                    }%
                    % Second head
                    \makebox[\widthHeadII][r]{
                        \myLabel{\cLoopHeadII}%
                    }%
                    \makebox[\widthHeadGap]{}% GAP
                    % Third head
                    \myLabel{\cLoopHeadIII}%
                    \myLabel{\hrule}
                %------------------- INNER LOOP
                \foreach \itemC/\labelC [count=\nC] in \cLoop
                {%
                    \ifthenelse{\equal{\itemC}{}}% check if empty
                    {}
                    {%
                    \item[]%
                    %%%%%% FIRST COLUMN
                    \makebox[\widthHeadI][r]{%
                        \ifthenelse{\equal{\labelC}{}}{}
                        {\myLabel{\labelC}}%
                    }%
                    %%%%%% SECOND COLUMN
                    \makebox[\widthHeadII][r]{%
                        \expandafter\detokenize\expandafter{\itemC}%
                    }%
                    \makebox[\widthHeadGap]{}% Gap
                    %%%%%% Apply style/size/family/etc
                    \fcolorbox{black!10}{white}{\parbox{\widthHeadIII}{%
                        \applyProperty{\itemA}%
                        \applyProperty{\itemB}%
                        \applyProperty{\itemC}%
                        %%%%%% DEMO TEXT
                        \fontDemoText%
                    }}
                    }
                }%end of innerloop
            \end{myListMeta}%
            }
        }}% end of middle loop
        \vspace{1cm}
        %%%% END RULE
        \outerSeparator
    }% end of outer loop
    \fi

%%%%%%%%%%%%%%%%%%%%%%%%%%%%%%%%%%%%%%%%%%%%%%%%%%%%%%%%%%%%%%%%%%%%%%%%%%%%%%%
\end{document}