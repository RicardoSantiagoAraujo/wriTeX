%%%%% BIBLIOGRAPHY
%% Troubleshooting: check README
\usepackage{csquotes}

\ifthenelse{\boolean{isIncludeCitationsInFootnotes}}
{%
    \newcommand{\myCiteStyle}{authoryear}%
    \newcommand{\myBibStyle}{authoryear}%
}
{%
    \newcommand{\myCiteStyle}{numeric}%
    \newcommand{\myBibStyle}{numeric}%
}



\usepackage[
    %sorting=none,
    sorting=nty, %name, title, year
    %%% CITATION STYLES: "style applies to both citations in text and display in printed bibliography, citestyle and bibstyles splits
    % Style option examples: numeric, authoryear, authortitle, verbose
    % style=authoryear,
    citestyle=\myCiteStyle,
    bibstyle=\myBibStyle,
    backend=biber,
    datamodel=mydatamodels,
    doi=false,
    isbn=false,
    url=false,
    eprint=false
    % maxcitenames=2
    % maxbibnames=1,
    % minbibnames=3
]{biblatex} %Imports biblatex package
%Import the bibliography file(s)
%%%% bibliographic sources
\newcommand{\loadBibIfExists}[1]%%% load bib file only if it exists
{\IfFileExists{#1}
    {\addbibresource{#1}}%
    {}%
}
\isPortfolio
{
    \loadBibIfExists{../../../articles_common_files/biblatex_files/bibliography.bib}% FOR PORTFOLIO
}
{
    \isArticle{%%% make sure it really is an article, and not e.g. a standalone file
        \loadBibIfExists{../../articles_common_files/biblatex_files/bibliography.bib}% FOR SINGLE ARTICLES
    }{}
}


\DeclareBibliographyCategory{myMediationArticles}
\addtocategory{myMediationArticles}{\myarticleKey}


%%% Include all biblatex items in bibliography, even if not cited in the text
% \nocite{*}

%%% MY OWN TYPE AND RESPECTIVE FIELDS
\input{../../../../articles_common_files/settings_and_packages/biblatex_settings/my_biblatex_types.tex}


\AtDataInput{\stepcounter{%
    totalCitationsAltogether%
    }} % Increment with each citation
% AtDataInput{} is triggered at the instant of each citation, whereas AtEveryBibitem{} counts only after Bib printing




% Command to add bibliography section
\newcommand{\addBibliography}{
    \isPortfolio{%
    \newcommand{\myCitationCounter}{1} %%% Print in ANY case since 1<>0
    }
    {
    \newcommand{\myCitationCounter}{%
        \totvalue%
        {totalCitationsInArticle:\myarticleKeyCore}%
        }
    }

    \ifthenelse{\boolean{isIncludeBiblio} \and \myCitationCounter>0}{
        \newpage
        \begin{SplitColumnsInTwo}%[true]
        \updateRibbons{\textbf{\TEXTbibliography}}{}
        \mySectionTitle{\TEXTbibliography}
        % This document contains \total{totalCitationsInArticle:\myarticleKeyCore}\ citation(s).
        \printbibliography[
            heading=none, % "bibintoc" adds the title to the table of contents. "none" to exclude.
            % title={My bibliography title} % Add title above bibliography
            % type=report,
            notcategory=myMediationArticles% Exclude my articles
            ]
        \end{SplitColumnsInTwo}
    }{}
}




\newcommand{\myCite}[1]{%
    \ifthenelse{\boolean{isIncludeCitations}}{% whether or not to include citations in article
    \stepcounter{%
    totalCitationsInArticle:\myarticleKeyCore% COUNTS REPEATS!
    }%
        \ifthenelse{\boolean{isIncludeCitationsInFootnotes}}{% whether to show citations in footnotes or not
            \footcite{#1}%
        }{%
            \cite{#1}%
        }%
    }{}
}



% raise inline text of different size to be aligned vertically
\newcommand*\raiseup[2]{%
        \begingroup%
        \setbox0\hbox{#1\strut #2}%
        \leavevmode%
        % Change formula to adjust height
        \raise\dimexpr (\ht\strutbox - \ht0)/3 \box0%
        \endgroup%
}

% Change missing reference message for Biblatex
\usepackage{xpatch}
\newcommand{\myUnknownRefSymbol}{????}
\makeatletter%
\def\abx@missing@entry#1{%
\raiseup{\tiny}{%
    \textcolor{myColorDanger}{%
        \abx@missing{[\myUnknownRefSymbol\ #1 \myUnknownRefSymbol]}%
    }%
    }%
}
\makeatother%


%%% Custom cite commands

% command to apply to prenotes and custom inputs
\newcommand{\genericPrenote}[1]
{\textcolor{myGrayMed}{#1\addcolon\space}}
% custom field format
\DeclareFieldFormat{myLabelFormat}{\genericPrenote{\titlecap{#1}}}
%
% field format for prenotes
\DeclareFieldFormat{prenote}
{\genericPrenote{#1}}
%
% custom empty entry format
\DeclareFieldFormat{myEntrymptyEntry}{\textcolor{red}{#1}}
%
% field format for DOI specifically (auto applies)
\DeclareFieldFormat{doi}{%
%   \mkbibacro{DOI} % prints label by default
  \ifhyperref
    {\href{https://doi.org/#1}{\nolinkurl{#1}}}
    {\nolinkurl{#1}}}
%
% field format for URL specifically (auto applies)
\DeclareFieldFormat{url}{%
% \mkbibacro{URL} % prints label by default
\url{#1}}
%
% field format for ISSN specifically (auto applies)
\DeclareFieldFormat{issn}{%
% \mkbibacro{ISSN} % prints label by default
#1}
% Define separator between citation and postnote
\renewcommand{\postnotedelim}{\space}

% \usepackage{natbib}
% \setcitestyle{comma}

% Macro to check if an entry is empty, and print something if TRUE or FALSE
\newcommand{\checkIfNoEntryFound}[2]{%
    \iffieldundef{\myEntry}% IS IT A FIELD ?
    {%
        \ifnameundef{\myEntry}% IS IT A NAME ?
        {%
            \iflistundef{\myEntry}% IS IT A LIST ?
            {#1}%
            {#2}%
        }%
        {#2}%
    }%
    {#2}%
}%
\newcommand{\checkIfNoEntryFoundConditional}[2]{%
    \ifthenelse{\boolean{isIncludeMissingBibEntries}}
    {%
    % \textcolor{myColorSuccess}{TRUE}
    #2%
    }%
    {%
        \checkIfNoEntryFound{#1}{#2}
    }%
}%


\DeclareCiteCommand{\myCiteCommand}%
    {% PRENOTE
    % \textcolor{orange}{\myEntry}
    \checkIfNoEntryFound{%
            % \textcolor{red}{Missing entry!}
        }{%
            \renewcommand{\genericPrenote}[1]{#1}% to remove formatting from prenote
            \usebibmacro{prenote}%
        }%
    }
    {
        \iffieldundef{\myEntry}% IS IT A FIELD ?
        {%
            \ifnameundef{\myEntry}% IS IT A NAME ?
            {%
                \iflistundef{\myEntry}% IS IT A LIST ?
                {%
                    % \ifthenelse{\boolean{isIncludeMissingBibEntries}}{%
                    \isDraftDebugger{
                        \printtext[myEmptyEntry]
                        {\na}
                        }{}%
                        % If it is none of the below
                    % }{}%
                }%
                {%
                    \printlist{\myEntry}% if it is a biblatex list
                }%
            }%
            {%
                \printnames{\myEntry}% if it is a biblatex name
            }%
        }%
        {%
            \printfield{\myEntry}% if it is a biblatex field
        }%
    }
    {}
    {% POSTNOTE
    \checkIfNoEntryFound{%
            % Missing entry!
        }{%
            \usebibmacro{postnote}%
        }%
    }

\DeclareCiteCommand{\myCiteWithLabelCommand}%
    {%
        \checkIfNoEntryFoundConditional{%
            % Missing entry!
        }{%
            \item[%
                \iffieldundef{prenote}%
                {%
                    \printtext[myLabelFormat]{\myEntry:}%
                    % \setunit{\prenotedelim}
                }%
                {%
                    \usebibmacro{prenote}%
                }%
            ]
        }%
    }%
    {%
        \iffieldundef{\myEntry}% IS IT A FIELD ?
        {%
            \ifnameundef{\myEntry}% IS IT A NAME ?
            {%
                \iflistundef{\myEntry}% IS IT A LIST ?
                {%
                    \ifthenelse{\boolean{isIncludeMissingBibEntries}}{%
                        \isDraftDebugger{
                            \printtext[myEmptyEntry]{\na}%
                            }{}%
                    }{}%
                }%
                {%
                    \printlist{\myEntry}% if it is a biblatex list
                }%
            }%
            {%
                \printnames{\myEntry}% if it is a biblatex name
            }%
        }%
        {%
            \printfield{\myEntry}% if it is a biblatex field
        }%
    }%
    {%
        \multicitedelim%
    }%
    {%
        \checkIfNoEntryFoundConditional{%
            % Missing entry!
        }{%
            \usebibmacro{postnote}%
        }%
    }%


% placeholder command to hold entry label
\newcommand{\myEntry}{}
% entrypoint commands for the citation that picks on the above cite command to generalize it
\NewDocumentCommand{\myCiteEntryWithLabel}
{
    m% #1 citation key
    m% #2 citation entry key
    O{}% #3 prenote
    O{}% #4 postnote
}{%
    \renewcommand{\myEntry}{#2}%
    \myCiteWithLabelCommand[#3][#4]{#1}%
}%

% entrypoint command for the citation that picks on the above cite command to generalize it
\NewDocumentCommand{\myCiteEntry}
{
    m% #1 citation key
    m% #2 citation entry key
    O{}% #3 prenote
    O{}% #4 postnote
}{%
    \renewcommand{\myEntry}{#2}%
    % \textcolor{red}{#2}
    \myCiteCommand[#3][#4]{#1}%
}%
