\usepackage{wrapfig}
\usepackage{subfloat} % for subfigures
\setcounter{lofdepth}{2} % we want subfigures in the list of figures

\usepackage{caption}
\usepackage[list=true,listformat=simple]{subcaption} % for subfigure captions
\captionsetup[figure]{font=scriptsize,labelfont=scriptsize}
\captionsetup[figure]{labelformat=empty} % remove label numbering from figure captions
\renewcommand{\figurename}{Fig.} % change label from Figure to Fig. (in text references)
\renewcommand{\subfigurename}{Subfig.} % change label from Subfigure to Subfig. (in text references)
\subcaptionsetup[figure]{labelformat=empty} % remove label numbering from figure subcaptions

\newcommand{\stepGeneralFigureCounter}{%%% Use this macro as the general counter to increment the counter whenever a figure is added
  \stepcounter{totalFiguresInArticle:\myarticleKeyCore}%
  \stepcounter{totalFiguresAltogether}%
}

\newcommand{\refcounterWithoutIncrementing}[1]
{%%% Macro to set a counter  as label reference without incrementing it
  \addtocounter{#1}{-1}\refstepcounter{#1}% subtract 1 then add 1
}

\makeatletter
\newcommand\myFigCaption{% my caption style, linking to the TOF
  \@dblarg\@myFigCaption}
  \newcommand\myFigCaptionLinker{image:\myarticleKeyCore:\the\value{totalFiguresInArticle:\myarticleKeyCore}}% key to identify individual figures
\def\@myFigCaption[#1]#2{%
  \caption[\protect\hypertarget{\myFigCaptionLinker}{#1}]%
      {%
        \ifthenelse{\boolean{isIncludeLoF}}
        {% as hyperlink
          \hyperlink{\myFigCaptionLinker}{#2}
        }%
        {% as simple text
          #2%
        }%
      }%
    }
\makeatother
%%%%

\NewDocumentCommand{\figLabel}
{%
  m % #1 label id
}{%
  fig:\detokenize{#1}% detokenize to treat special characters as plain text
}


\NewDocumentCommand{\myFigRef}
{
    m % reference key
    O{Figure} % reference prefix
}{%
  \begingroup%
  \ifthenelse{\boolean{isDraft}}{%
    \color{myColorWarning}%
    {\texttt{(\figLabel{#1})}}%
  }%
  {%
    \color{myGrayMed}%
  }
  #2\ \ref{\figLabel{#1}}%
  \endgroup%
}


%%%% THE GENERAL FIGURE ENVIRONMENT WHERE INDIVIDUAL FIGURES WILL BE PLACED
\NewDocumentEnvironment{myFigEnv}
{
  O{}% #1: h(ere approx.), b(ottom), t(op), p(age alone), H(ERE PRECISELY). Leaving it empty lets latex choose
  O{width=0.475\linewidth} % #2: default figure size
  m % #3: figure caption.
  m % #4: figure caption extra.
  O{} % #5: optional argument to add a custom label for the figure environment
}
{%%% START ENVIRONMENT
  \begin{figure}[#1]%
    \stepGeneralFigureCounter
    \stepcounter{totalFigureEnvsInArticle:\myarticleKeyCore}%
    \stepcounter{totalFigureEnvsAltogether}%
    \centering%
    \isDraftDebugger{\myBreakMessage{START FIGURE ENVIRONMENT}}{}%
}
{%%%% END ENVIRONMENT
    %%% Caption:
    \captionsetup{#2}% Caption settings
    \myFigCaption[#3]{%
    \textcolor{myGrayMed}{\textbf{Figure\ \the\value{totalFiguresInArticle:\myarticleKeyCore}:\ }}% Figure number label
    \textbf{#3.} #4% Figure caption
    }%
    \isDraftDebugger{\myBreakMessage{END FIGURE ENVIRONMENT}}{}%
    \label{fig:env:#5} %optional label for environment
  \end{figure}
}

%%%% THE GENERAL WRAPFIGURE ENVIRONMENT WHERE INDIVIDUAL WRAP FIGURES WILL BE PLACED
\NewDocumentEnvironment{myWrapFigEnv}
{
  O{10}% #1: How many lines to wrap
  O{R}% #2: r(ight), l(eft), i(nside edge), o(utside edge). If uppercase, figure is a float. Lowercase means exactly here.
  O{0.33\textwidth} % #3: default wrapfigure width. Here "width=" cannot be included in contrast to myFigEnv
  m % #4: figure caption.
  m % #5: figure caption extra.
  O{} % #6: optional argument to add a custom label for the figure environment
}
{%%% START ENVIRONMENT
  \wrapfigure[#1]{#2}{#3}%%% Start wrapfigure
    \stepGeneralFigureCounter%
    \stepcounter{totalWrapfigureEnvsInArticle:\myarticleKeyCore}%
    \stepcounter{totalWrapfigureEnvsAltogether}%
    \centering%
    \isDraftDebugger{%
    % START WRAPFIGURE ENVIRONMENT\\%
    %
    \tiny wrap #1 lines;\ %
    \tiny position: #2;\ %
    \tiny width:\detokenize{#3}\\%
    }{}%
}
{%%%% END ENVIRONMENT
    %%% Caption:
    \myFigCaption[#4]{%
    \textcolor{myGrayMed}{\textbf{Figure\ \the\value{totalFiguresInArticle:\myarticleKeyCore}:\ }}% Figure number label
    \textbf{#4.} #5% Figure caption
    }%
    % \isDraftDebugger{END WRAPFIGURE ENVIRONMENT\\}{}%
    \label{fig:env:#6} %optional label for environment
  \endwrapfigure%%% End wrapfigure
}


\renewcommand\thesubfigure{\alph{subfigure}}% customize type of subfigure (e.g. arabic, alpha) counter display
%%%% COMMAND FOR SUBFIGURES (using subfloat, which is more up to date than subfigure)
\NewDocumentCommand{\mySubfig}
{%
  O{width=1\linewidth}% automatically adjusts to width of figure
  m % #2: figure caption.
  m % #3: figure caption extra.
  m% #4: contents of subfloat.
  O{}% #5: Optional label (cannot set it through the contents for unkown reason)
}%
{%
  \stepcounter{totalSubfiguresInArticle:\myarticleKeyCore}%
  \stepcounter{totalSubfiguresAltogether}%
  \captionsetup[subfigure]{#1}%
  \subfloat%
  [\ifstrempty{#2}{\isDraftDebugger{Captionless subfigure}{}}{#2}]%%% LOT ENTRY (if empty, pass an empty space so that is is still included in LOT)
  [%%% CAPTION OF SUBFLOAT
    \textcolor{myGrayMed}{\textbf{%
      Subfigure\ % custom text before counter
      \the\numexpr\value{totalFiguresInArticle:\myarticleKeyCore}+0\relax% Figure number label (if needed, add 1 so it matches correctly before the actual Figure counter is incremented)
      \thesubfigure% subfigure counter
    \ifstrempty{#2}{}{:\ }% add colon if caption is not empty
    }}%
    \ifstrempty{#2}{}{% check if provided caption is empty
      \textbf{#2.} #3% Figure caption
      }%
  ]%
  {%%%   CONTENT OF SUBFLOAT
    #4%
    \label{fig:#5}% label, which cannot be set through the contents placed in the environment, for an unkown reason
  }%
}%



%%%% MACRO TO INCLUDE A RASTER IMAGE
\NewDocumentCommand{\myFigGraphics}
{
  O{width=1\linewidth} % #1: default figure size attribute. Can be overwritten
  O{%
    {./}%
    {./assets/figures/}% lone article
    {./../../articles_common_files/assets/}% lone article
    {../../../articles/\myarticleKeyCore/assets/figures/}%portfolio
    {./../../../articles_common_files/assets/}% portfolio
  } % #2: default filepath(s). Can be overwritten
  m % #3: file name
  O{jpg} % #4: default extension. Can be overwritten
}
{%
    \stepcounter{totalGraphicsInArticle:\myarticleKeyCore}%
    \stepcounter{totalGraphicsAltogether}%
    \refcounterWithoutIncrementing{totalFiguresAltogether}
    %%% Figure path: is it really needed ?
    \graphicspath{#2}%
    %%% Figure as TiKz to add overlays:
      \begin{tikzpicture}%
        % Image
        \node[
          anchor=center,
          % minimum width=5cm,
          % minimum height=5cm,
          %%% Add border frame around pictures
          % draw=myColorPrimary, % border color
          % line width=2mm, % border width
          %%% Different ways to add shadows
          % drop shadow={opacity=1, shadow scale=1, shadow xshift=.7ex, shadow yshift=-.7ex, fill=red, path fading={circle with fuzzy edge 15 percent}},
          % blur shadow={shadow blur steps=5, shadow scale=1.2, shadow xshift=0.1em,shadow yshift=#1}
          ] at (0,0) {%
          %ADD FIGURE
          % \detokenize{#2#3.#4}
          \includegraphics[#1]{#3.#4}%
          %
          };
        % Rectangle
        % \node [draw, thick, shape=rectangle, minimum width=1\linewidth, minimum height=1cm, anchor=center, fill=myColorPrimary, fill opacity=0.9] at (0,0) {rectangular overlay};
        % Data over image
        \isDraftDebugger{
          % Figure label
          \node[fill=black!90, fill opacity=0.75, anchor=south]
          at (current bounding box.south)
          {Ref key: \figLabel{#3}};
          % Image extension
          \node[fill=black!90, fill opacity=0.75, anchor=north east]
          at (current bounding box.north east)
          {.\detokenize{#4 }};
          % %Image width
          \node[fill=black!90, fill opacity=0.75, anchor=north west]
          at (current bounding box.north west)
          {\detokenize{#1}};
        }{}%
      \end{tikzpicture}%
      %%% Label:
      \label{\figLabel{#3}}% fig:filename
      % Figure label: \figLabel{#3}
      % Figure counter: \the\value{totalFiguresInArticle:\myarticleKeyCore}
}

%%%% MACRO TO INCLUDE A TIKZ IMAGE
\NewDocumentCommand{\myFigTikz}
{
  O{scale=1} % #1: default figure scale attribute (fraction). Can be overwritten, probably necessary to adjust manually on a per img basis.
  O{%
  {./}%
  {./assets/figures/}% lone article
  {./../../articles_common_files/assets/}% lone article
  {../../../articles/\myarticleKeyCore/assets/figures/}% portfolio
  {./../../../articles_common_files/assets/}% portfolio
  } % #2: default filepath(s). Can be overwritten
  m % #3: file name
}
{%
    \stepcounter{totalTikZPicturesInArticle:\myarticleKeyCore}%
    \stepcounter{totalTikZPicturesAltogether}%
    \refcounterWithoutIncrementing{totalFiguresAltogether}
    %%% Figure path:
    \graphicspath{#2}
    %%% TiKz settings
    \tikzset{
      background grid/.style={
          % thick,
          thin,
          draw=myGrayLight,
          step=.5cm
        },
      background rectangle/.style={
          % rounded corners,
          % double, % for double border
          ultra thick,
          draw=myGrayDark,
          fill=white,
          % top color=blue,
          % bottom color= pink
        }
    }
    %%% Figure as TiKz:
    \isMinimal
    {%%% Load stored PDF to be faster
      \isPortfolio
      {%
        \includegraphics[#1]{../../../articles/\myarticleKeyCore/assets/tikz/#3/auxiliary_files/standalone.pdf}%
      }
      {%
        \includegraphics[#1]{assets/tikz/#3/auxiliary_files/standalone.pdf}%
      }
    }
    {%%% Fresh compilation of Tikz
    \isDraft
    {%
      \begin{tikzpicture}
        [#1,% passed as optional argument
        show background rectangle,
        show background grid
        ]
    }
    {%
      \begin{tikzpicture}
        [#1,% passed as optional argument
        ]
    }

        \isPortfolio
        {% Relative address for portfolio
          \expandafter\input\expandafter{../../../articles/\myarticleKeyCore/assets/tikz/#3/content.tex}%
        }{% Relative address for article
        \expandafter\input\expandafter{assets/tikz/#3/content.tex}%
        }%
        % Place a node at the center using current bounding box
        \isDraftDebugger{%
          \node[fill=black!90, fill opacity=0.75, anchor=north] at (current bounding box.south) {fig key: \normalsize\detokenize{#3}};%
        }{};
      \end{tikzpicture}
    }
    %%% Label:
    \label{\figLabel{#3}}% fig:filename
    % Figure label: \figLabel{#3}
    % Figure counter: \the\value{totalFiguresInArticle:\myarticleKeyCore}
}
